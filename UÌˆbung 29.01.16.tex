\documentclass[11pt]{memoir}

\usepackage[utf8]{inputenc}
\usepackage[ngerman]{babel}
\usepackage{mathrsfs} %hübsche Buchstaben
\usepackage{amssymb}
\usepackage{ntheorem}
\everymath{\displaystyle}
\usepackage{amsmath, amssymb} %Damit geht dann cases
%\usepackage[all, error]{onlyamsmath} %Wohl sinnvoll, damit Befehle immer mathematisch genommen 
\usepackage{ulsy} %Für Widerspruchspfeil
\usepackage{mathtools} %Für Rightarrow mit Wort darüberunderbrace
\usepackage{extpfeil} %für = mit Wort drüber
\usepackage{fdsymbol} %varrightwavearrow - blockiert Underbrace: muss nach mathtools kommen, sonst hässliche Balken.
\usepackage{braket} % Für schönere Mengen. Oder so. Hier ist z. B. \Set drin und die vertikalen Balken sind schön.


\newcommand{\dom}{\partial\Omega}
\newcommand{\pax}{\partial x}
\newcommand{\pay}{\partial y}
\newcommand{\pap}{\partial\phi}
\begin{document}


\title{\textbf{Analysis III}\\ Übung am 29.01.16}
\author{Tamar Mirbach}
\date{Wintersemester 2015/16\\ Karlsruher Institut für Technologie}

\maketitle


\emph{Mengen mit $\mathscr C^1$-Rand}: 
\par
$\Omega \subset \mathbb R^n$ heißt $\mathscr C^k$-regulär, falls für jedes $p \in \partial\Omega \; \exists ($nach Rotation und Translation$), U \subset \mathbb R^{n-1}, I \subset \mathbb R$ offen, $u: U \rightarrow I, u \in \mathscr C^k (U)$ 
$$\Omega \cap U \times I = \Set{(y,r) \in U \times I | r < u(y)}$$

$\exists O \in SO(n), q \in \mathbb R^n$, 
$$O(\Omega - p) \cap U \times I = \Set{(y,r) \in U \times I | r < u(y)}$$

\par

\textbf{Wichtig}: \\
$\partial\Omega$ ist eine $\mathscr C^{(k)}$-Mannigfaltigkeit. Falls $\Omega \subset \mathbb R^n$ und $\partial\Omega$ ist eine $\mathscr C^{(k)}$-Mannigfaltigkeit $$\nRightarrow \Omega \; \mathscr C^k \text{-regulär}.$$
z.B $B_2\backslash \partial B_1$

\par

\textbf{Satz von Gauß}: \\
$\Omega \subset \mathbb R^n$ mit $\mathscr C^1$-Rand. $X \in \mathscr C^1(\overline{\Omega}, \mathbb R^n)$, dann gilt:
$$\int_\Omega \text{div} (X)  = \int_{\partial\Omega} X\cdotp \nu \;d\mu_{\partial\Omega}$$
\emph{Randbemerkung}: \\
Wichtiges Mittel für den Beweis war die Partition der $1$. \\
\textbf{$1.$ Schritt:} 
$$\Omega = \Set{(x, r) | r < u(x)}$$
$$x \in U,\; U \subset \mathbb R^{n-1},\; u \in \mathscr C^1(U \mathbb R^n)$$
$$\partial\Omega \;\cap\; (U \times \mathbb R) = \Set{u(x) = r} \Rightarrow T_{(x, u(x))}  \text{Graph}_u = \text{span}\Set{e_i + \dfrac{\partial u}{\partial  x_i}  e_n | i = 1, ..., n-1}$$
$$X \in \mathscr C^1(U, \mathbb R^n) \text{ mit spt } (X) \subset K \times \mathbb R$$
$$\int_\Omega \sum\limits_{i=1}^n \dfrac{\partial x^i}{x ^i} \, dx = \int_\Omega \text{div}(X) = \int_{\partial \Omega \cap U \times \mathbb R} X \nu \, d\mu_{\partial \Omega}$$
$$ = \int_U \left\langle X(x, u(x), \text{ Vektor }-\bigtriangledown u \text{ und }1 \right\rangle \, dx $$
$$= \int_U \left\langle X(x, u(x)), selber Vektor \right\rangle \; dx$$
\textbf{$2.$ Schritt:} Partition der Eins
\par
Im Falle von $\mathbb R^3$-Bemerkung: \\
Sei $\Omega$ ein $\mathscr C^1$-reguläre Menge und $\phi: U \rightarrow \mathbb R^3$ mit $U \subset \mathbb R^2$ offen und $\phi(U)$ ist ein lokale Parametrisierung von $\partial \Omega$
$$\nu \| \text{ span }\Set{\dfrac{\partial\phi}{\partial x}, \dfrac{\partial \phi}{\partial y}}^\bot \Rightarrow \nu = \lambda \cdotp \dfrac{\partial \phi}{\partial x} \times \dfrac{\partial \phi}{\partial y}; \; \lambda = \pm \dfrac{1}{\left| \dfrac{\partial \phi}{\partial x} \times \dfrac{\partial \phi}{\partial y}\right|}$$
$$\left\lvert\dfrac{\partial \phi}{\partial x} \times \dfrac{\partial \phi}{\partial y}\right\rvert^2 = \left\lvert\dfrac{d \phi}{d x} \times \left(  \dfrac{\pap}{\pay} - \left\langle\dfrac{ \dfrac{\pap}{\pax}}{\left\lvert\dfrac{\pap}{\pax}\right\rvert^2}, \dfrac{\pap}{\pay}\right\rangle \dfrac{ \dfrac{\pap}{\pax}}{\left\lvert\dfrac{\pap}{\pax}\right\rvert}\right) \right\rvert^2$$ 
$$= \left|\dfrac{\pap}{\pax}\right|^2 \left(\left|\dfrac{\pap}{\pay} - \left\langle\dfrac{ \dfrac{\pap}{\pax}}{\left|\dfrac{\pap}{\pax}\right|}, \dfrac{\pap}{\pay}\right\rangle \dfrac{\pap}{\pax}\right|\right)^2$$
$$= \left|\dfrac{\pap}{\pax}\right|^2\left|\dfrac{\pap}{\pay}\right|^2  - \left\langle\dfrac{\partial \phi}{\partial x} , \dfrac{\partial \phi}{\partial y}\right\rangle^2 = \det(g)$$

\par\bigskip

$g$ Gramsche Matrix
$$\int_{\partial \Omega \cap \phi(U)} X \nu \, d\mu_{\partial\Omega} = \sigma \int_U \left\langle X(\phi(x)), \dfrac{\partial \phi}{\partial x} \times \dfrac{\partial \phi}{\partial y}\right\rangle \, dx \, dy,$$
wobei $\sigma \in \Set{\pm 1}$ je nach Orientierung von $\phi$.

\par\bigskip
\textbf{1. Beispiel}: \\
Sei $\Omega \subset \mathbb R^n \, \mathscr C^1$-regulär, dann gilt:
$$L^n(\Omega) = \dfrac{1}{n} \int_{\partial \Omega} \left\langle x - x_0, \nu\right\rangle \; d\mu_{\partial\Omega} \quad \forall x_0 \in \mathbb R^n \quad (1)$$
$$\Rightarrow L^n(\Omega) \leq \dfrac{1}{n}\text{ diam }(\Omega) \mu(\partial\Omega) \quad (2)$$

\par\bigskip
\textbf{Theorem}: 
\par
Sei $u: \overline{B_1} \rightarrow \overline{B_1}$ stetig differenzierbar auf $\overline{B_1}$, dann hat $u$ (mindestens) einen Fixpunkt, d.h.
$$\exists x \in \overline{B_1} \text{ mit }u(x) = x$$
\par\bigskip

\textbf{Lemma 1}: (Null Lagrangions) 
$L: \mathbb R^{n \times n} \rightarrow \mathbb R$ heißt \textbf{Null-Lagrangion}, falls gilt:
$$\forall u \in \mathscr C^1(U, \mathbb R^n): \quad \dfrac{\partial}{\partial x} \left( \dfrac{\partial L}{\partial p_{i,j}}\right) = 0 \quad \forall j = 1,... n$$
Wenn $L$ diese Eigenschaft hat:
$$\int_U L(\bigtriangledown u) = \int_U L(\bigtriangledown v) \quad \forall u, v \in \mathscr C^1(U, \mathbb R^n) \text{ mit }u = v \text{ auf } \dom$$


\textbf{Lemma 2}: 
$L(P) = \det (P)$ ist ein Null-Lagrangian.

\end{document}