\documentclass[11pt]{memoir}

\usepackage[utf8]{inputenc}
\usepackage[ngerman]{babel}
\usepackage{mathrsfs}
\usepackage{amssymb}
\usepackage[standard]{ntheorem}
\everymath{\displaystyle}

\newcommand{\cara}{Carathéodory-Fortsetzung}

\begin{document}

\title{\textbf{Analysis III}\\ Skript zur Vorlesung von Prof. Dr. Tobias Lamm}
\author{Tamar Mirbach\\ Younis Bensalah}
\date{Wintersemester 2015/16\\ Karlsruher Institut für Technologie}

\maketitle

\chapter{Maße und messbare Funktionen}
\section{$\sigma$-Algebren und Maße}
Notation: $X$ Menge, \\ $P(X) =$ \textbraceleft Teilmengen von $X$\textbraceright  {} Potenzmenge \\
$A$ $\subset$  $P(X)$ Mengensystem
\begin{Definition}
Ein Mengensystem $\mathscr{A} \subset P(X)$ heißt \underline{$\sigma$-Algebra}, falls
\begin{enumerate}
	\item $X \in \mathscr{A}$
	\item $A \in \mathscr{A} \Rightarrow X \backslash A \in \mathscr{A}$
	\item $A_i \in \mathscr{A}$, $i \in$ $\mathbb{N} \Rightarrow \bigcup_{i = 1}^{\infty} A_i$
\end{enumerate}
Das System $(X, \mathscr{A})$ heißt \underline{messbarer Raum}.
\end{Definition}

\begin{Bemerkung} %mit doppeltem $$ große Caps, sonst kleine mit Grenzen rechts nebendran. Groß passt nicht in Fließtext
\begin{enumerate}
	\item $A_i \in \mathscr{A}$, $i \in \mathbb{N} \Rightarrow $ $$\bigcap_{i = 1}^{\infty} A_i$$, denn:
	$$\bigcap_{i=1}^{\infty} A_i $$ $= X\backslash ($ $$\bigcup_{i=1}^{\infty} (X \backslash A_i))$$
	\item $\emptyset \in \mathscr{A}$
	\item $A, B \in \mathscr{A} \Rightarrow A \backslash B \in \mathscr{A}$, denn $A \backslash B = A \cap (X \backslash B)$
\end{enumerate}
\end{Bemerkung}

\begin{Beispiel}
\begin{enumerate}
	\item $P(X)$ trivial
	\item \textbraceleft$ \emptyset, X $\textbraceright{} trivial
\end{enumerate}
\end{Beispiel}

\begin{Satz}
Jeder Durchschnitt von (endlich oder $\infty$-vielen) $\sigma$-Algebren auf derselben Menge $X$ ist eine $\sigma$-Algebra auf $X$.
\end{Satz}

\begin{Definition}
Für ein Mengensystem $E \subset P(X)$ heißt \\
$\sigma(E):=\bigcap$ \textbraceleft $\mathscr{A}: \mathscr{A}$ ist $\sigma$-Algebra in X und E $ \subset \mathscr{A}$\textbraceright {} die \underline{ von $E$ erzeugte $\sigma$-Algebra}.
\end{Definition}

\begin{Bemerkung}
\begin{enumerate}
	\item $P(X)$ $\sigma$-Algebra mit $E \subset P(X)$
	\item Satz 1 $\Rightarrow \sigma(E)$ ist $\sigma$-Algebra
	\item $\mathscr{A}$ $\sigma$-Algebra mit $E \subset \mathscr{A} \Rightarrow \sigma(E) \subset 		\mathscr{A}$ "kleinste"{} $\sigma$-Algebra, die $E$ enthält.
\end{enumerate}
\end{Bemerkung}

\begin{Beispiel}
\begin{enumerate}
	\item $E \subset X$ $\Rightarrow $ $\sigma(E)$ = \textbraceleft $X$, $E$, $\emptyset$, $X \setminus E$\textbraceright
	\item $(X, d)$ sei metrischer Raum. $\mathscr{O}:=$ \textbraceleft offene Teilmenge von $X$ \\
	$\sigma(\mathscr{O})=:\mathscr{B(O)}$ \underline{Borel $\sigma$-Algebra}. Elemente darin heißen \underline{Borelmengen} \\
	$X = \mathbb{R}^{n}: \mathscr{B(O)} = \mathscr{B}^{n}$ \\
	$\overline{\mathbb{R}} := \mathbb{R}$  $\cup$ \textbraceleft$\pm\infty$\textbraceright {}($a < b$ Anordnung)
\end{enumerate}
\end{Beispiel}


\begin{Definition}
Eine Folge $(s_{k}) \subset \overline{\mathbb{R}}$ \underline{konvergiert} gegen $s \in \mathbb{R}$, falls eine der folgenden Alternativen gilt:
\begin{itemize}
	\item  $s \in \mathbb{R}$ und $\forall \epsilon > 0$ gilt $s_{k} \in (s-\epsilon, s+\epsilon)$ $\forall k$ groß genug
	\item $s = \infty$ und $\forall r \in \mathbb{R}: s_{k} \in (r, \infty]$ $\forall k$ groß genug
	\item $s = -\infty$ und $\forall r \in \mathbb{R}: s_k \in (-\infty, r)$ $\forall k$ groß genug
\end{itemize}

\end{Definition}



%Satz I.11, VL 06.11.15
\begin{Satz}
$\lambda: R \rightarrow [0, \infty], {}R \subset P(X)$ (Prämaß auf Ring), $\forall {}E \subset X:$ \\
\begin{center}
$\mu(E):= inf\{ \sum\limits_{i=1}^{\infty} \lambda(A_i) | E \subset \bigcup\limits_{i=1}^{\infty} A_i, A_i \in \mathbb{R}\}$
\end{center}
ist Fortsetzung von $\lambda$.
\end{Satz}

\begin{Lemma}
Sei $\mu$ Carathéodory-Fortsetzung des Prämaß $\lambda$ auf $R$. Sei $\tilde{\mu}$

\end{Lemma}


\chapter{Das Lebesgue -Maß und -Integral}
\section{Das Lebesgue-Maß}

Halbring $\mathcal{Q}^n$ Inhalt.\\
$vol^n : \mathcal{Q}^n \rightarrow [0, \infty]$

\begin{Satz}
	Prämaß auf Halbring\\
	$\Rightarrow$ \cara{} ist äußeres Maß auf $P(X)$.
\end{Satz}

\begin{Lemma}
	$vol^n : \mathcal{Q}^n \rightarrow [0, \infty]$ ist ein Prämaß.
\end{Lemma}

\begin{Definition}
	Das (äußere) Lebesguemaß einer Menge $E \subset \mathbb{R}^n$ ist definiert durch\\
	$$ \mathscr{L}^n(E) := inf\{ \sum_{i = 1}^{\infty} vol^n(P_i) \mid P_i \in \mathscr{Q}^n, E \subset \cup_{i = 1}^{\infty} P_i \} $$
\end{Definition}


\end{document}
