\documentclass[11pt]{report}
\usepackage[applemac]{inputenc}
\usepackage[ngerman]{babel}
\usepackage{mathrsfs}
\usepackage{amssymb}
\usepackage{ntheorem}
\newtheorem{defi}{Definition}
\newtheorem{lemma}{Lemma}
\newtheorem{satz}{Satz}

\begin{document}

\title{Analysis III}
\author{Dr. Tobias Lamm}
%\date{}
\maketitle
\chapter{I. Ma�e und messbare Funktionen}
\section{ $\sigma$-Algebren und Ma�e} 
Notation: $X$ Menge, \\ $P(X) =$ \textbraceleft Teilmengen von $X$\textbraceright  {} Potenzmenge \\
$A$ $\subset$  $P(X)$ Mengensystem
\begin{defi}
Ein Mengensystem $\mathscr{A} \subset P(X)$ hei�t \underline{$\sigma$-Algebra}, falls
\begin{enumerate}
	\item $X \in \mathscr{A}$
	\item $A \in \mathscr{A} \Rightarrow X \backslash A \in \mathscr{A}$
	\item $A_i \in \mathscr{A}$, $i \in$ $\mathbb{N} \Rightarrow \bigcup_{i = 1}^{\infty} A_i$
\end{enumerate}
Das System $(X, \mathscr{A})$ hei�t \underline{messbarer Raum}.
\end{defi}

\subsubsection{Bemerkung}
\begin{enumerate}
	\item $A_i \in \mathscr{A}$, $i \in \mathbb{N} \Rightarrow \bigcap_{i = 1}^{\infty} A_i$, denn:
	$\bigcap_{i=1}^{\infty} A_i = X\backslash (\bigcup_{i=1}^{\infty} (X \backslash A_i))$
	\item $\emptyset \in \mathscr{A}$
	\item $A, B \in \mathscr{A} \Rightarrow A \backslash B \in \mathscr{A}$, denn $A \backslash B = A \cap (X \backslash B)$
\end{enumerate}

\subsubsection{Beispiel}
\begin{enumerate}
	\item $P(X)$ trivial
	\item \textbraceleft$ \emptyset, X $\textbraceright{} trivial 
\end{enumerate}

um irgendwas hinzuzuf�gen

\end{document}