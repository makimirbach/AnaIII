\documentclass[11pt]{memoir}

\usepackage[utf8]{inputenc}
\usepackage[ngerman]{babel}
\usepackage{mathrsfs}
\usepackage{amssymb}
\usepackage{ntheorem}
\everymath{\displaystyle}
\usepackage{amsmath, amssymb} %Damit geht dann cases
%\usepackage{array}
\usepackage{mathabx} %für #

\theoremstyle{break}

\newtheorem{Definition}{Definition}[chapter]
\newtheorem{Bemerkung}{Bemerkung}[chapter]
\newtheorem{Beispiel}{Beispiel}[chapter]
\newtheorem{Lemma}{Lemma}[chapter]
\newtheorem{Satz}{Satz}[chapter]

\newcommand{\cara}{Carathéodory-Fortsetzung}

\begin{document}

\title{\textbf{Analysis III}\\ Skript zur Vorlesung von Prof. Dr. Tobias Lamm}
\author{Tamar Mirbach\\ Younis Bensalah}
\date{Wintersemester 2015/16\\ Karlsruher Institut für Technologie}

\maketitle

\chapter{Maße und messbare Funktionen}
\section{$\sigma$-Algebren und Maße}


\emph{Notation}: $X$ Menge, \\ $P(X) =$ \textbraceleft Teilmengen von $X$\textbraceright  {} Potenzmenge \\
$A$ $\subset$  $P(X)$ Mengensystem
%Def. I.1
\begin{Definition}
Ein Mengensystem $\mathscr{A} \subset P(X)$ heißt \underline{$\sigma$-Algebra}, falls
\begin{enumerate}
	\item $X \in \mathscr{A}$
	\item $A \in \mathscr{A} \Rightarrow X \backslash A \in \mathscr{A}$
	\item $A_i \in \mathscr{A}$, $i \in$ $\mathbb{N} \Rightarrow \bigcup_{i = 1}^{\infty} A_i$
\end{enumerate}
Das System $(X, \mathscr{A})$ heißt \underline{messbarer Raum}.
\end{Definition}

%Bem.
\begin{Bemerkung}
\begin{enumerate}
	\item $A_i \in \mathscr{A}$, $i \in \mathbb{N} \Rightarrow \bigcap_{i = 1}^{\infty} A_i$, denn:
	$\bigcap_{i=1}^{\infty} A_i  = X \backslash (\bigcup_{i=1}^{\infty} (X \backslash A_i))$
	\item $\emptyset \in \mathscr{A}$
	\item $A, B \in \mathscr{A} \Rightarrow A \backslash B \in \mathscr{A}$, denn $A \backslash B = A \cap (X \backslash B)$
\end{enumerate}
\end{Bemerkung}

%Bsp.
\begin{Beispiel}
\begin{enumerate}
	\item $P(X)$ trivial
	\item $\{ \emptyset, X \}${} trivial
\end{enumerate}
\end{Beispiel}

%Satz I.1
\begin{Satz}
Jeder Durchschnitt von (endlich oder $\infty$-vielen) $\sigma$-Algebren auf derselben Menge $X$ ist eine $\sigma$-Algebra auf $X$.
\end{Satz}

%Def I.2
\begin{Definition}
Für ein Mengensystem $E \subset P(X)$ heißt \\
$\sigma(E):=\bigcap \{ \mathscr{A}: \mathscr{A}$ ist $\sigma$-Algebra in $X$ und $E$ $\subset \mathscr{A} \}$ die \underline{von $E$ erzeugte $\sigma$-Algebra}.
\end{Definition}

%Bem.
\begin{Bemerkung}
\begin{enumerate}
	\item $P(X)$ $\sigma$-Algebra mit $E \subset P(X)$
	\item Satz 1 $\Rightarrow \sigma(E)$ ist $\sigma$-Algebra
	\item $\mathscr{A}$ $\sigma$-Algebra mit $E \subset \mathscr{A} \Rightarrow \sigma(E) \subset 		\mathscr{A}$ "kleinste"{} $\sigma$-Algebra, die $E$ enthält.
\end{enumerate}
\end{Bemerkung}

%Bsp.
\begin{Beispiel}
\begin{enumerate}
	\item $E \subset X$ $\Rightarrow $ $\sigma(E)$ = \textbraceleft $X$, $E$, $\emptyset$, $X \setminus E$\textbraceright
	\item $(X, d)$ sei metrischer Raum. $\mathscr{O}:=$ \textbraceleft offene Teilmenge von $X$ \\
	$\sigma(\mathscr{O})=:\mathscr{B(O)}$ \underline{Borel $\sigma$-Algebra}. Elemente darin heißen \underline{Borelmengen} \\
	$X = \mathbb{R}^{n}: \mathscr{B(O)} = \mathscr{B}^{n}$ \\
	$\overline{\mathbb{R}} := \mathbb{R}$  $\cup$ \textbraceleft$\pm\infty$\textbraceright {}($a < b$ Anordnung)
\end{enumerate}
\end{Beispiel}

%Def I.3
\begin{Definition}
Eine Folge $(s_{k}) \subset \overline{\mathbb{R}}$ \underline{konvergiert} gegen $s \in \mathbb{R}$, falls eine der folgenden Alternativen gilt:
\begin{itemize}
	\item  $s \in \mathbb{R}$ und $\forall \epsilon > 0$ gilt $s_{k} \in (s-\epsilon, s+\epsilon)$ $\forall k$ groß genug
	\item $s = \infty$ und $\forall r \in \mathbb{R}: s_{k} \in (r, \infty]$ $\forall k$ groß genug
	\item $s = -\infty$ und $\forall r \in \mathbb{R}: s_k \in (-\infty, r)$ $\forall k$ groß genug
\end{itemize}
\end{Definition}

%Bem.
\begin{Bemerkung}
\begin{enumerate}
	\item $(a_n)$ monoton wachsende Folge
	$\Rightarrow \lim a_n \in \mathbb{\overline{R}}$
	\item $\sum\limits_{n=1}^{\infty} a_n \in \mathbb{\overline{R}}$, $a_n \geq 0$ $ \forall n \in \mathbb{N}$
	\item offene Teilmengen von $\overline{\mathbb{R}}$: \\
	$U \subset \overline{\mathbb{R}} $ offen $\Leftrightarrow U \cap \mathbb{R}$ offen und falls $\{+\infty\}$ oder $\{-\infty\}$ in $M$ liegt, existiert ein $a \in \mathbb{R}$ mit $(a,{} \infty ]\subset U$ bzw. $[-\infty, a) \subset U$
\end{enumerate}
\end{Bemerkung}
 % Hier fehlt noch eine kleine Tabelle

 %Def I.4
\begin{Definition}
$\mathscr{A} \subset P(X)$ $\sigma$-Algebra. Eine nicht negative Mengenfunktion $\mu: \mathscr{A} \rightarrow [0, \infty]$ heißt \underline{Maß} auf $\mathscr{A}$, wenn
\begin{enumerate}
	\item $\mu(\infty) = 0$
	\item für paarweise disjunkte Mengen $A_i \in \mathscr{A}, i \in \mathbb{N}$ gilt: \\
	$\mu(\bigcup\limits_{i=1}^{\infty}A_i) = \sum\limits_{i=1}^{\infty} \mu(A_i)$ \underline{$\sigma$-Additivität}
\end{enumerate}
Das Tripel $(X, \mathscr{A}, \mu)$ wird als Maßraum bezeichnet.
\end{Definition}

%Bem
\begin{Bemerkung}
Monotonie: $A \subset B \Rightarrow \mu(A) \leq \mu(B)$\\
$\mu(B) = \mu(A \cup B\backslash A) = \mu(A) + \underbrace{\mu(B\backslash A)}_{\geq 0}\geq \mu(A)$
\end{Bemerkung}

%Def I.5
\begin{Definition}
$(X, \mathscr{A}, \mu)$ sei Maßraum: $\mu$ heißt \underline{endlich}, wenn $\mu(A)$ \textless {} $ \infty $ $\forall A \in \mathscr{A}$, $\mu$ heißt \underline{$\sigma$-endlich}, falls eine Folge $(X_i) \subset \mathscr{A}$ mit $\mu(X_i)$ \textless {} $ \infty $ existiert, sodass $X = \bigcup\limits_{i=1}^{\infty} X_i$. \\
Falls $\mu(X) = 1$, so wird $\mu$ \underline{Wahrscheinlichkeitsmaß} genannt.
\end{Definition}


%Bsp.
\begin{Beispiel}
\begin{enumerate}
	\item $X$ Menge, $\mathscr{A} = P(X), x \in X$ \underline{Diracmaß}
	\begin{equation}
		\delta_{x} (A) =
		\begin{cases}
			1 & \text{falls } x \in A\\
		0 & \text{sonst}
		\end{cases}
	\end{equation}
	\begin{itemize}
		\item $\delta_x ( \emptyset) = 0$
		\item Sei $A = \bigcup\limits_{i=1}^{\infty} A_i$ und $A_i$ sind paarweise disjunkt.
		\begin{enumerate}
			\item $x \in A$, $\delta_x(A) = 1 = \sum\limits_{i=1}^{\infty} \delta_x(A_i)$, $x \in A_i$ für genau ein i
			\item $x \notin A \Rightarrow \delta_x(A) = 0 = \sum\limits_{i=1}^{\infty}A_i)$
					\end{enumerate}
		Weil $\delta_x(X) = 1$, ist $\delta_x$ ein Wahrscheinlichkeitsmaß.
	\end{itemize}
	\item $X$ beliebig, $\mathscr{A} = P(X)$: \underline{Zählmaß} card: $P(X) \rightarrow [0, +\infty]$
	\begin{equation}
		card(A) :=
		\begin{cases}
			\varhash \text{ Elemente von} A & A \text{ endlich} \\
			\infty & \text{sonst}
		\end{cases}
	\end{equation}
	$card(\emptyset) = 0$ \\
	$A = \bigcup\limits_{i=1}^{\infty} A_i$ paarweise disjunkt.
	\begin{enumerate}
		\item $A$ endlich $\Rightarrow$ endliche Vereinigung von endlichen Mengen
		\item $A$ nicht endlich
		\begin{enumerate}
			\item ein $A_i$ ist nicht endlich
			\item $\forall A_i$ endlich
		\end{enumerate}
		card endlich $\Leftrightarrow X$ endlich \\
		card ist $\sigma$-endlich $\Leftrightarrow X$ abzählbar
	\end{enumerate}
\end{enumerate}
\end{Beispiel}

%Satz I.2
\begin{Satz}
$(X, \mathscr{A}, \mu)$ sei Maßraum. Dann gelten für $A_i \in \mathscr{A}, i \in \mathbb{N}$ die Aussagen
\begin{enumerate}
	\item $A_1 \subset A_2 \subset ...  A_i \subset A_{i+1} \subset .. $\\
	$\Rightarrow \mu(\bigcup\limits_{i=1}^{\infty} A_i) = \lim_{i \rightarrow \infty} \mu(A_i)$
	\item Aus $A_1 \supset A_2 \supset ... \supset A_i \supset A_{i+1} \supset ...$ mit $\mu(A_1)$ \textless  {} $ \infty$ folgt\\
	$\mu(\bigcap\limits_{i=1}^{\infty} A_i) = \lim_{i \rightarrow \infty} \mu(A_i)$
	\item $\mu(\bigcup\limits_{i=1}^{\infty} A_i) \leq \sum\limits_{i=1}^{\infty} \mu(A_i)$ \\
	($A_i$ müssen hier nicht paarweise disjunkt sein)
\end{enumerate}
\end{Satz}

%Bem
\begin{Bemerkung}
\begin{enumerate}
	\item \emph{1.} heißt \underline{Stetigkeit von unten} \\
	\emph{2.} heißt \underline{Stetigkeit von oben} \\
	\emph{3.} \underline{$\sigma$-Subadditivität} des Maßes
	\item  $\mu(A_1$ \textless $\infty)$ in \emph{2.} kann durch $\mu(A_k) $ \textless  $\infty$ für ein $k \in \mathbb{N}$ ersetzt werden.
\end{enumerate}
\end{Bemerkung}

%Bsp.
\begin{Beispiel}
$A_k := \{k, k+1, ... \} \subset \mathbb{N}$, card($A_k$) \textgreater $\infty$ \\
card($\bigcap\limits_{k=1}^{\infty} A_k$) $= 0 \ne 0 \lim_{k \rightarrow \infty} $card$ (A_k) = \infty$
\end{Beispiel}




%Satz I.11, VL 06.11.15
\begin{Satz}
$\lambda: R \rightarrow [0, \infty], {}R \subset P(X)$ (Prämaß auf Ring), $\forall {}E \subset X:$ \\
\begin{center}
$\mu(E):= inf\{ \sum\limits_{i=1}^{\infty} \lambda(A_i) | E \subset \bigcup\limits_{i=1}^{\infty} A_i, A_i \in \mathbb{R}\}$
\end{center}
ist Fortsetzung von $\lambda$.
\end{Satz}

\begin{Lemma}
Sei $\mu$ \cara{} des Prämaß $\lambda$ auf $R$. Sei $\tilde{\mu}$ ein Maß auf $\sigma(R)$ mit $\tilde{\mu} = \lambda$ auf $R$. Dann gilt $\tilde{\mu}(R) \leq \mu(E)$ $\forall E \in \sigma(R)$.
\end{Lemma}


\begin{Satz} \emph{Hopf-Fortsetzung} \\
Sei $\lambda: R \rightarrow [0, \infty]$ Prämaß auf Ring $R \subset P(X)$. Dann existiert ein Maß $\mu$ auf $\sigma(R)$ mit $\mu =\lambda$ auf $R$. Diese Fortsetzung ist \underline{eindeutig}, falls $\lambda$ $\sigma$-endlich ist.
\end{Satz}

\emph{Notation:}
Sei $\mu$ Maß auf $X$ mit der Eigenschaft: $\forall D \subset X: \exists E \in \mathscr{M}(\mu)$ mit $D \subset E$ und $\mu(D) = \mu(E)$. Dann heißt $\mu$ \underline{reguläres} äußeres Maß.

\begin{Beispiel}

Blödes Beispiel
\end{Beispiel}

\begin{Satz}
Sei $\mu$ \cara{} des Prämaßes $\lambda$ auf $R$.
\end{Satz}

\chapter{Das Lebesgue -Maß und -Integral}
\section{Das Lebesgue-Maß}

Halbring $\mathcal{Q}^n$ Inhalt.\\
$vol^n : \mathcal{Q}^n \rightarrow [0, \infty]$

\begin{Satz}
	Prämaß auf Halbring\\
	$\Rightarrow$ \cara{} ist äußeres Maß auf $P(X)$.
\end{Satz}

\begin{Lemma}
	$vol^n : \mathcal{Q}^n \rightarrow [0, \infty]$ ist ein Prämaß.
\end{Lemma}

\begin{Definition}
	Das (äußere) Lebesguemaß einer Menge $E \subset \mathbb{R}^n$ ist definiert durch\\
	$$ \mathscr{L}^n(E) := inf\{ \sum_{i = 1}^{\infty} vol^n(P_i) \mid P_i \in \mathscr{Q}^n, E \subset \cup_{i = 1}^{\infty} P_i \} $$
\end{Definition}


\end{document}
