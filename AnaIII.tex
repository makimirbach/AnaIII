\documentclass[11pt]{memoir}

\usepackage[utf8]{inputenc}
\usepackage[ngerman]{babel}
\usepackage{mathrsfs}
\usepackage{amssymb}
\usepackage{ntheorem}
\everymath{\displaystyle}
\usepackage{amsmath, amssymb} %Damit geht dann cases
%\usepackage{array}
\usepackage{mathabx} %für #
%\usepackage{amsfonts} Wie bekommt man Underrightarrow{bla}?
\usepackage{fdsymbol} %varrightwavearrow
%\usepackage{fontenc} % Für "
\usepackage{ulsy} %Für Widerspruchspfeil
\usepackage{mathtools} %Für Rightarrow mit Wort darüber


\theoremstyle{break}

\newtheorem{Definition}{Definition}[chapter]
\newtheorem{Bemerkung}{Bemerkung}[chapter]
\newtheorem{Beispiel}{Beispiel}[chapter]
\newtheorem{Lemma}{Lemma}[chapter]
\newtheorem{Satz}{Satz}[chapter]

\newcommand{\cara}{Carathéodory-Fortsetzung}

\begin{document}

\title{\textbf{Analysis III}\\ Skript zur Vorlesung von Prof. Dr. Tobias Lamm}
\author{Tamar Mirbach\\ Younis Bensalah}
\date{Wintersemester 2015/16\\ Karlsruher Institut für Technologie}

\maketitle

\chapter{Maße und messbare Funktionen}
\section{$\sigma$-Algebren und Maße}


\emph{Notation}: $X$ Menge, \\ $P(X) =$ \{Teilmengen von $X$\}  {} Potenzmenge \\
$A$ $\subset$  $P(X)$ Mengensystem
%Def. I.1
\begin{Definition}
Ein Mengensystem $\mathscr{A} \subset P(X)$ heißt \underline{$\sigma$-Algebra}, falls
\begin{enumerate}
	\item $X \in \mathscr{A}$
	\item $A \in \mathscr{A} \Rightarrow X \backslash A \in \mathscr{A}$
	\item $A_i \in \mathscr{A}$, $i \in$ $\mathbb{N} \Rightarrow \bigcup_{i = 1}^{\infty} A_i$
\end{enumerate}
Das System $(X, \mathscr{A})$ heißt \underline{messbarer Raum}.
\end{Definition}

%Bem.
\begin{Bemerkung}
\begin{enumerate}
	\item $A_i \in \mathscr{A}$, $i \in \mathbb{N} \Rightarrow \bigcap_{i = 1}^{\infty} A_i$, denn:
	$\bigcap_{i=1}^{\infty} A_i  = X \backslash (\bigcup_{i=1}^{\infty} (X \backslash A_i))$
	\item $\emptyset \in \mathscr{A}$
	\item $A, B \in \mathscr{A} \Rightarrow A \backslash B \in \mathscr{A}$, denn $A \backslash B = A \cap (X \backslash B)$
\end{enumerate}
\end{Bemerkung}

%Bsp.
\begin{Beispiel}
\begin{enumerate}
	\item $P(X)$ trivial
	\item $\{ \emptyset, X \}${} trivial
\end{enumerate}
\end{Beispiel}

%Satz I.1
\begin{Satz}
Jeder Durchschnitt von (endlich oder $\infty$-vielen) $\sigma$-Algebren auf derselben Menge $X$ ist eine $\sigma$-Algebra auf $X$.
\end{Satz}

%Def I.2
\begin{Definition}
Für ein Mengensystem $E \subset P(X)$ heißt \\
$\sigma(E):=\bigcap \{ \mathscr{A}: \mathscr{A}$ ist $\sigma$-Algebra in $X$ und $E$ $\subset \mathscr{A} \}$ die \underline{von $E$ erzeugte $\sigma$-Algebra}.
\end{Definition}

%Bem.
\begin{Bemerkung}
\begin{enumerate}
	\item $P(X)$ $\sigma$-Algebra mit $E \subset P(X)$
	\item Satz 1 $\Rightarrow \sigma(E)$ ist $\sigma$-Algebra
	\item $\mathscr{A}$ $\sigma$-Algebra mit $E \subset \mathscr{A} \Rightarrow \sigma(E) \subset 		\mathscr{A}$ "kleinste"{} $\sigma$-Algebra, die $E$ enthält.
\end{enumerate}
\end{Bemerkung}

%Bsp.
\begin{Beispiel}
\begin{enumerate}
	\item $E \subset X$ $\Rightarrow $ $\sigma(E)$ = \textbraceleft $X$, $E$, $\emptyset$, $X \setminus E$\textbraceright
	\item $(X, d)$ sei metrischer Raum. $\mathscr{O}:=$ \textbraceleft offene Teilmenge von $X$ \\
	$\sigma(\mathscr{O})=:\mathscr{B(O)}$ \underline{Borel $\sigma$-Algebra}. Elemente darin heißen \underline{Borelmengen} \\
	$X = \mathbb{R}^{n}: \mathscr{B(O)} = \mathscr{B}^{n}$ \\
	$\overline{\mathbb{R}} := \mathbb{R}$  $\cup$ \{t$\pm\infty$\} {}($a < b$ Anordnung)
\end{enumerate}
\end{Beispiel}

%Def I.3
\begin{Definition}
Eine Folge $(s_{k}) \subset \overline{\mathbb{R}}$ \underline{konvergiert} gegen $s \in \mathbb{R}$, falls eine der folgenden Alternativen gilt:
\begin{itemize}
	\item  $s \in \mathbb{R}$ und $\forall \epsilon > 0$ gilt $s_{k} \in (s-\epsilon, s+\epsilon)$ $\forall k$ groß genug
	\item $s = \infty$ und $\forall r \in \mathbb{R}: s_{k} \in (r, \infty]$ $\forall k$ groß genug
	\item $s = -\infty$ und $\forall r \in \mathbb{R}: s_k \in (-\infty, r)$ $\forall k$ groß genug
\end{itemize}
\end{Definition}

%Bem.
\begin{Bemerkung}
\begin{enumerate}
	\item $(a_n)$ monoton wachsende Folge
	$\Rightarrow \lim a_n \in \mathbb{\overline{R}}$
	\item $\sum\limits_{n=1}^{\infty} a_n \in \mathbb{\overline{R}}$, $a_n \geq 0$ $ \forall n \in \mathbb{N}$
	\item offene Teilmengen von $\overline{\mathbb{R}}$: \\
	$U \subset \overline{\mathbb{R}} $ offen $\Leftrightarrow U \cap \mathbb{R}$ offen und falls $\{+\infty\}$ oder $\{-\infty\}$ in $M$ liegt, existiert ein $a \in \mathbb{R}$ mit $(a,{} \infty ]\subset U$ bzw. $[-\infty, a) \subset U$
\end{enumerate}
\end{Bemerkung}
 % Hier fehlt noch eine kleine Tabelle

 %Def I.4
\begin{Definition}
$\mathscr{A} \subset P(X)$ $\sigma$-Algebra. Eine nicht negative Mengenfunktion $\mu: \mathscr{A} \rightarrow [0, \infty]$ heißt \underline{Maß} auf $\mathscr{A}$, wenn
\begin{enumerate}
	\item $\mu(\infty) = 0$
	\item für paarweise disjunkte Mengen $A_i \in \mathscr{A}, i \in \mathbb{N}$ gilt: \\
	$\mu(\bigcup\limits_{i=1}^{\infty}A_i) = \sum\limits_{i=1}^{\infty} \mu(A_i)$ \underline{$\sigma$-Additivität}
\end{enumerate}
Das Tripel $(X, \mathscr{A}, \mu)$ wird als Maßraum bezeichnet.
\end{Definition}

%Bem
\begin{Bemerkung}
Monotonie: $A \subset B \Rightarrow \mu(A) \leq \mu(B)$\\
$\mu(B) = \mu(A \cup B\backslash A) = \mu(A) + \underbrace{\mu(B\backslash A)}_{\geq 0}\geq \mu(A)$
\end{Bemerkung}

%Def I.5
\begin{Definition}
$(X, \mathscr{A}, \mu)$ sei Maßraum: $\mu$ heißt \underline{endlich}, wenn $\mu(A)$ \textless {} $ \infty $ $\forall A \in \mathscr{A}$, $\mu$ heißt \underline{$\sigma$-endlich}, falls eine Folge $(X_i) \subset \mathscr{A}$ mit $\mu(X_i)$ \textless {} $ \infty $ existiert, sodass $X = \bigcup\limits_{i=1}^{\infty} X_i$. \\
Falls $\mu(X) = 1$, so wird $\mu$ \underline{Wahrscheinlichkeitsmaß} genannt.
\end{Definition}


%Bsp.
\begin{Beispiel}
\begin{enumerate}
	\item $X$ Menge, $\mathscr{A} = P(X), x \in X$ \underline{Diracmaß}
	\begin{equation}
		\delta_{x} (A) =
		\begin{cases}
			1 & \text{falls } x \in A\\
			0 & \text{sonst}
		\end{cases}
	\end{equation}
	\begin{itemize}
		\item $\delta_x ( \emptyset) = 0$
		\item Sei $A = \bigcup\limits_{i=1}^{\infty} A_i$ und $A_i$ sind paarweise disjunkt.
		\begin{enumerate}
			\item $x \in A$, $\delta_x(A) = 1 = \sum\limits_{i=1}^{\infty} \delta_x(A_i)$, $x \in A_i$ für genau ein i
			\item $x \notin A \Rightarrow \delta_x(A) = 0 = \sum\limits_{i=1}^{\infty}A_i)$
					\end{enumerate}
		Weil $\delta_x(X) = 1$, ist $\delta_x$ ein Wahrscheinlichkeitsmaß.
	\end{itemize}
	\item $X$ beliebig, $\mathscr{A} = P(X)$: \underline{Zählmaß} card: $P(X) \rightarrow [0, +\infty]$
	\begin{equation}
		card(A) :=
		\begin{cases}
			\varhash \text{ Elemente von } A & A \text{ endlich} \\
			\infty & \text{sonst}
		\end{cases}
	\end{equation}
	$card(\emptyset) = 0$ \\
	$A = \bigcup\limits_{i=1}^{\infty} A_i$ paarweise disjunkt.
	\begin{enumerate}
		\item $A$ endlich $\Rightarrow$ endliche Vereinigung von endlichen Mengen
		\item $A$ nicht endlich
		\begin{enumerate}
			\item ein $A_i$ ist nicht endlich
			\item $\forall A_i$ endlich
		\end{enumerate}
		card endlich $\Leftrightarrow X$ endlich \\
		card ist $\sigma$-endlich $\Leftrightarrow X$ abzählbar
	\end{enumerate}
\end{enumerate}
\end{Beispiel}

%Satz I.2
\begin{Satz}
$(X, \mathscr{A}, \mu)$ sei Maßraum. Dann gelten für $A_i \in \mathscr{A}, i \in \mathbb{N}$ die Aussagen
\begin{enumerate}
	\item $A_1 \subset A_2 \subset ...  A_i \subset A_{i+1} \subset .. $\\
	$\Rightarrow \mu(\bigcup\limits_{i=1}^{\infty} A_i) = \lim_{i \rightarrow \infty} \mu(A_i)$
	\item Aus $A_1 \supset A_2 \supset ... \supset A_i \supset A_{i+1} \supset ...$ mit $\mu(A_1)$ \textless  {} $ \infty$ folgt\\
	$\mu(\bigcap\limits_{i=1}^{\infty} A_i) = \lim_{i \rightarrow \infty} \mu(A_i)$
	\item $\mu(\bigcup\limits_{i=1}^{\infty} A_i) \leq \sum\limits_{i=1}^{\infty} \mu(A_i)$ \\
	($A_i$ müssen hier nicht paarweise disjunkt sein)
\end{enumerate}
\end{Satz}

%Bem
\begin{Bemerkung}
\begin{enumerate}
	\item \emph{1.} heißt \underline{Stetigkeit von unten} \\
	\emph{2.} heißt \underline{Stetigkeit von oben} \\
	\emph{3.} \underline{$\sigma$-Subadditivität} des Maßes
	\item  $\mu(A_1$ \textless $\infty)$ in \emph{2.} kann durch $\mu(A_k) $ \textless  $\infty$ für ein $k \in \mathbb{N}$ ersetzt werden.
\end{enumerate}
\end{Bemerkung}

%Bsp.
\begin{Beispiel}
$A_k := \{k, k+1, ... \} \subset \mathbb{N}$, card($A_k$) \textgreater $ \infty$ \\
card($\bigcap\limits_{k=1}^{\infty} A_k$) $= 0 \ne 0 \lim_{k \rightarrow \infty} $card$ (A_k) = \infty$
\end{Beispiel}

%Def I.6
\begin{Definition}
Sei $(X, \mathscr{A}, \mu)$ Maßraum. Jede Menge $A \in \mathscr{A}$ mit $\mu(A) = 0$ heißt \underline{$\mu$-Nullmenge}. Das Maß heißt \underline{vollständig}, wenn $N \subset A $ für ein $A \subset \mathscr{A}$ mit $\mu(A) = 0 \Rightarrow N \in \mathscr{A}$ und $\mu(N) = 0$.
\end{Definition}

%Bsp.
\begin{Beispiel}
(das blöde Maß): \\
$\mu(A) = 0$ $ \forall A \in \mathscr{A}$ und $\mathscr{A} \ne P(X)$ ist \underline{nicht vollständig}
\end{Beispiel}

%Bem.
\begin{Bemerkung}
Jedes Maß kann vervollständigt werden (siehe Blatt $2$) 
\end{Bemerkung}


\section{Messbare Funktionen}

%Def. I.7
\begin{Definition}
Seien $(X, \mathscr{A}), (Y, \mathscr{C})$ messbare Räume.\\
 Eine Abbildung $f: X \rightarrow Y$ heißt \underline{$\mathscr{A}-\mathscr{C}$-messbar}, wenn $f^{-1}(\mathscr{C}) \subset \mathscr{A}$, d.h. $f^{-1}(C) \in \mathscr{A}$ $ \forall C \in \mathscr{C}$
\end{Definition}

%Bem.
\begin{Bemerkung}
$f$ heißt kurz messbar (bzw. $\mathscr{A}$-messbar), wenn an $\mathscr{A}, \mathscr{C}$ (bzw. $\mathscr{C}$) kein Zweifel besteht.
\end{Bemerkung}

%Bsp
\begin{Beispiel} 

\begin{enumerate}
	\item $(X, \mathscr{A}), (Y, \mathscr{C})$ beliebig. \\
	$f: X \rightarrow Y, f(x) = y_0 \in Y$ \\
	$\forall x \in X, c \in \mathscr{C}$: 
	\begin{equation}
		f^{-1}(c) = 
		\begin{cases}
			x & y_0 \in C \\
			\emptyset & y_0 \notin C
		\end{cases}
	\end{equation}
	\item Für $E \subset X$ beliebig heißt $\chi_E: X \rightarrow \mathbb{R}$, \\
	\begin{equation}
		\chi_E(x) =
		\begin{cases}
			1 & x \in E \\
			0 & \text{sonst}
		\end{cases}
	\end{equation}
	\underline{charakteristische Funktion} von $E$ auf $\mathbb{R}$. \\
	Sei $\mathscr{B}^1$ als $\sigma$-Algebra gegeben. \\
	$\chi_E$ ist $\mathscr{A}$-messbar $\Leftrightarrow$ $\{1\} = \bigcap\limits_k \left(1- \dfrac{1}{k}, 1 + \dfrac{1}{k}\right), E \in \mathscr{A}$
	\item $(X, \mathscr{A}), (Y, \mathscr{C}), (Z, \mathscr{D})$ messbare Räume \\
	$f: X \rightarrow Y$ $ \mathscr{A}-\mathscr{C}$-messbar \\
	$g: Y \rightarrow Z$ $ \mathscr{C}-\mathscr{D}$-messbar \\
	$\xRightarrow{Beh.}$ $  g \circ f: X \rightarrow Z$ $ \mathscr{A}-\mathscr{D}$-messbar, denn \\
	$(g \circ f)^{-1}(\mathscr{D}) = f^{-1}(g^{-1}(\mathscr{D})) \subset f^{-1}(\mathscr{D}) \subset \mathscr{A}$
\end{enumerate}
\end{Beispiel}

% Lemma I.1
\begin{Lemma}
$(X, \mathscr{A}), (Y, \mathscr{C})$ messbare Räume und $f: X \rightarrow Y$ Abbildung. Für beliebige Mengensysteme $\xi \in \mathscr{C}$ gilt $f^{-1}(\sigma(\xi)) = \sigma(f^{-1}(\xi))$
\end{Lemma}

% Lemma I.2
\begin{Lemma}
Seien $(X, \mathscr{A}), (Y, \mathscr{C})$ messbare Räume und $\mathscr{C} = \sigma(\xi)$ für ein Mengensystem $\xi \subset P(Y)$. Jede Abbildung $f: X \rightarrow Y$ mit $f^{-1}(\xi) \subset \mathscr{A}$ ist $\mathscr{A}-\mathscr{C}$-messbar.
\end{Lemma}

%Bsp
\begin{Beispiel}
\begin{enumerate}
	\item Jedes $f: \mathbb{R}^n \rightarrow \mathbb{R}^m$ stetig ist $\mathscr{B}^n-\mathscr{B}^m-$ messbar, denn $f^{-1}(\text{offen})$ ist offen. \\
	\emph{Notation}: $f$ ist \underline{Borel-messbar} 
	\item $X \ne \emptyset$ Menge, $(Y, \mathscr{C})$ messbarer Raum, $f: X \rightarrow Y$ Abbildung. \\
	Blatt $\varrightwavearrow$ $f^{-1}(\mathscr{C})$ $\sigma$-Algebra. Tatsächlich ist es die kleinste $\sigma$-Algebra, die $f: X \rightarrow Y$ messbar macht. $f^{-1}(\mathscr{C})$ heißt die \underline{durch $f$ und $(Y, \mathscr{C})$ induzierte $\sigma$-Algebra}
\end{enumerate}
\end{Beispiel}

%Definition I.8
\begin{Definition}
Sei $(X, \mathscr{A})$ messbarer Raum und $D \in \mathscr{A}$. Eine Funktion $f: D \rightarrow \overline{\mathbb{R}}$ heißt \underline{numerische Funktion}.
\end{Definition}

%Bem
\begin{Bemerkung}
Numerische Funktionen sind $\mathscr{A}$-messbar (auf $D$), wenn $f^{-1}(\overline{\mathbb{B}^1}) \subset \mathscr{A}|_D:= \{A \cap D |A \in \mathscr{A}\}$ \\
($\mathscr A | _D$ ist $\sigma$-Algebra, siehe Übung).
\end{Bemerkung}

%Lemma I.3
\begin{Lemma}
$(X, \mathscr A)$ messbarer Raum, $D \in \mathscr A$, $f: D \rightarrow \overline {\mathbb R}$. Dann sind äquivalent:
\begin{enumerate}
	\item $f$ ist $\mathscr A$-messbar
	\item $\forall U \subset \mathbb R$ offen ist $f^{-1}(U) \in \mathscr A |_D$ und $f^{-1}(\{\infty\}), f^{-1}(\{-\infty\}) \in \mathscr A |_D$
	\item $\{f \leq s\} := \{x \in D: f(x) \leq x\} \in \mathscr A |_D$ $ \forall s \in \mathbb R$
	\item $\{f < s\} := \{x \in D: f(x) < x\} \in \mathscr A |_D$ $ \forall s \in \mathbb R$
	\item $\{f \geq s\} := \{x \in D: f(x) \geq x\} \in \mathscr A |_D$ $ \forall s \in \mathbb R$
	\item $\{f > s\} := \{x \in D: f(x) > x\} \in \mathscr A |_D$ $ \forall s \in \mathbb R$
\end{enumerate}
\end{Lemma}

%Lemma I.4
\begin{Lemma}
$(X, \mathscr A)$ messbarer Raum. $D \in \mathscr a$ und $f, g: D \rightarrow \overline{\mathbb R }$ $\mathscr A$-messbar \\
$ \left( f^{-1}(\overline{\mathscr{B}^1}) \subset \mathscr A |_D, g^{-1}(\overline{\mathscr{B}^1}) \subset \mathscr A |_D\right)$. \\ Dann sind die Mengen $\{f < g\}:= \{x \in D: f(x) < g(x)\}$ und $\{f \leq g\}$ Elemente aus $\mathscr A |_D$.
\end{Lemma}

%Satz I.3
\begin{Satz}
$(X, \mathscr A)$ messbarer Raum. $D \in \mathscr A$, $f_k: D \rightarrow \overline{\mathbb R}$ Folge von $\mathscr A$-messbaren numerischen Funktionen. dann sind $\inf\limits_k f_k$, $\sup\limits_k f_k$, $\liminf\limits_{k\rightarrow \infty} f_k$, $\limsup\limits_{k \rightarrow \infty} f_k$ auch $\mathscr A$-messbar. \\
(Hierbei ist $(\liminf f_k)(x) = \liminf f_k(x)$, ebenso $\limsup$)
\end{Satz}

%Satz I.4
\begin{Satz}
$(X, \mathscr A)$ messbarer Raum, $D \in \mathscr A$, $f, g: D \rightarrow \overline{\mathbb R }$ $\mathscr A$-messbar und $\alpha \in \mathbb R$. Dann sind $f+g, \alpha \cdotp f, f^\pm, \max{\{f, g\}}, \min{\{f, g\}}, \|f\|, f \cdotp g, \dfrac{f}{g}$ $\mathscr A$-messbar. \\
$\left( f^+ = \max{\{f, 0\}} \geq 0, f^- = \max{\{-f, 0\}} \geq 0, f = f^+ - f^-, \|f\| = f^+ + f^- \right)$
\end{Satz}

%Wieso x in M in A?? Was ist dann M?
%Bem
\begin{Bemerkung}
$(X, \mathscr A, \mu)$ Maßraum. Man sagt, die Aussage $A[x]$ ist wahr für \underline{$\mu$-fast alle $x \in M \in \mathscr A$}, falls eine $\mu$-Mullmenge $N$ existiert mit $\{x \in M: A[x]  \text{ falsch}\} \subset N$.\\
$f, g: X \rightarrow \overline{\mathbb R}$ Aussage  "$f(x) \leq g(x)$ für $\mu$-fast alle $x \in X$"  bedeutet: \\
$\exists N \subset \mathscr A$ mit $\mu(N) = 0$, sodass $\forall x \in X\backslash N: f(x) \leq g(x)$ \\
Eine Funktion $h$ ist "\underline{$\mu$-fast überall auf $X$}" definiert, wenn $h$ aus $D \in \mathscr A$ definiert ist und $\mu(X \backslash D) = 0$. \\
Eine Folge von Funktionen $f_k: D \rightarrow \overline{\mathbb R}$ konvergiert punktweise $\mu$-fast überall gegen $f: D \rightarrow \overline{\mathbb R}$, falls eine Menge $M \in \mathscr A$ existiert mit $\mu(N) = 0$ und $\lim\limits_ {k \rightarrow \infty} f_k(x) = f(x)$ $\forall x \in D \backslash N$
\end{Bemerkung}

%Definition I.9
\begin{Definition}
$(X, \mathscr A, \mu)$ Maßraum. Eine auf $D \in \mathscr A$ definierte numerische Funktion $f$ heißt \underline{$\mu$ messbar} (auf $X$), wenn $\mu(X \backslash D) = 0$ und $f$ $\mathscr A$-messbar ist.
\end{Definition}

%bem
\begin{Bemerkung}
Die Relation $f = g$ $\mu$-fast überall ist eine Äquivalenzrelation. Sei $D \in \mathscr A, f: D \rightarrow \overline{\mathbb R} \mu$-messbar. Dann gibt es eine $\mathscr A$-messbare Funkion $g: X \rightarrow \overline{\mathbb R}$ mit $f = g$ auf $D$. \\
\emph{Z.B.} \\
\begin{equation}
	g(x) = 
	\begin{cases}
		f(x) & , x \in D \\
		0 & \text{, sonst}
	\end{cases}
\end{equation} \\
$\rightarrow$ Satz $1.3$ und $1.4$ gelten auch für $\mu$-messbare Funktionen (man muss zusätzlich fordern, dass $f, g, f \cdotp g, \dfrac{f}{g}$ $ \mu$-fast überall definiert sind)
\end{Bemerkung}


%Lemma I.5
\begin{Lemma}
$(X, \mathscr A, \mu)$ vollständiger Maßraum und $f$ $\mu$-messbar auf $X$. Dann ist auch jede Funktion $\tilde f$ mit $\tilde f = f$ $\mu-$fast überall auf $X$ $\mu$-messbar.
\end{Lemma}


%Satz I.5
\begin{Satz}
$(X, \mathscr A, \mu)$ vollständiger Maßraum, $f_k, k \in \mathbb N$ $\mu$-messbar. Falls $f_k$ punktweise $\mu$-fast überall gegen $f$ konvergiert, so ist $f$ $\mu$-messbar.
\end{Satz}

%Satz I.6
\begin{Satz}{\emph{Egorov:}} \\
$(X, \mathscr A, \mu)$ Maßraum. $D \in \mathscr A$ Menge mit $\mu(D) \less \infty$ und $f_k, f$ $ \mu|_D$-messbar, $\mu|_D$ - fast überall endliche Funktionen mit $f_k \rightarrow f$ $\mu|_D$- fast überall. \\
Dann existiert $\forall \epsilon \> 0$ eine Menge $B \subset D, B \in \mathscr A$ mit \\
\begin{enumerate}
	\item $\mu(D\backslash B) \less \epsilon$
	\item $f_n \rightarrow f$ gleichmäßig auf $B$. 
\end{enumerate}
\end{Satz}

\section{Äußere Maße}

%Def I.10
\begin{Definition}
Sei $X$ eine Menge. Eine Funktion $\mu: P(X) \rightarrow [0, \infty]$ mit $\mu(\emptyset) = 0$ und $A \subset \bigcup\limits_{i=1}^\infty A_i \Rightarrow \mu(A) \leq \sum\limits_{i=1}^\infty \mu(A_i)$ heißt \underline{äußeres Maß} auf $X$.
\end{Definition}

%Bem
\begin{Bemerkung}
\begin{enumerate}
	\item Äußeres Maß ist immer auf $P(X)$ definiert.
	\item $\mu$ Maß auf $P(X) \Rightarrow$ $\mu$ \underline{äußeres Maß}. 
	\item Begriffe $\sigma$-Additiv, $\sigma$-endlich, endlich, monoton, Nullmengen sind wie bei Maßen definiert.
	\item Jedes äußere Maß ist monoton. $A \subset B \Rightarrow \mu(A) \leq \mu(B)$
\end{enumerate}
\end{Bemerkung}

%Def I.11
\begin{Definition}
Sei $\mu$ äußeres Maß auf $X$. Die Menge $A \subset X$ heißt \underline{$\mu$-messbar}, falls für alle $S \subset X$: \\
$\mu(S) \geq \mu(S \cap A) + \mu(S \backslash A)$ \\
$\mathscr M(\mu):= \{ A \subset X | A$ $ \mu-messbar \}$
\end{Definition}

%Bem
\begin{Bemerkung}
Es gilt $(S \cap A) \cup (S \backslash A) = S$. Also $\mu(S) \leq \mu(S \cap A) + \mu(S \backslash A)$ \\
$A$ $\mu$-messbar $\Leftrightarrow$ $\mu(S) = \mu(S \cap A) + \mu(S \backslash A)$ $\forall S \subset X$
\end{Bemerkung}

%Satz I.7
\begin{Satz}
Sei $\mathscr Q$ ein System von Teilmengen einer Menge $X$, das die leere Menge enthält und sei $\lambda: \mathscr Q \rightarrow [0, \infty]$ mit $\lambda(\emptyset) = 0$.\\
Definiere $\mu: P(X) \rightarrow [0, \infty]$ für beliebige $E \subset X$ durch \\
 $\mu(E) := \inf \left\lbrace \sum\limits_{i=1}^\infty \lambda(P_i) | P_i \in \mathscr Q \forall i, E \subset \bigcup\limits_{i=1}^\infty P_i \right\rbrace$ \\
Dann ist $\mu$ äußeres Maß. $(\inf \emptyset = + \infty)$
\end{Satz}

%Satz I.8
\begin{Satz}
$\mu: P(X) \rightarrow [0, \infty]$ äußeres Maß auf $X$. Für $M \subset X$ erhält man durch $\mu|_M : P(X) \rightarrow [0, \infty], \mu|_M := \mu(A \cap M)$ ein \underline{äußeres Maß} (Einschränkung von $\mu$ auf $M$) auf $X$. Es gilt: $A$ $ \mu-$messbar $\Rightarrow$  $A$ $ \mu|_M $-messbar
\end{Satz}

%Satz I.9
\begin{Satz}
Sei $\mu$ äußeres Maß auf $X$. Dann gilt:
\begin{enumerate}
	\item $N$ $\mu$-Nullmenge $\Rightarrow N$ $\mu$-messbar 
	\item $N_i, i \in \mathbb N,$ $\mu$-messbar $\Rightarrow \bigcup\limits_{x=1}^\infty N_i$ Nullmengen
\end{enumerate}
\end{Satz}


%Lemma I.6
\begin{Lemma}
$\mu$ äußeres Maß auf $X$, $A_i \in \mathscr M(\mu), i = 1, ... k$ seien paarweise disjunkt. Dann gilt $\forall S \subset X$: \\
$\mu\left(S \cap \bigcup\limits_{i=1}^\infty A_i\right) = \sum\limits_{i=1}^\infty \mu(S \cap A_i)$
\end{Lemma}

%Satz I.10
\begin{Satz}
Sei $\mu: P(X) \rightarrow [0, \infty]$ äußeres Maß. Dann ist $\mathscr M(\mu)$ eine $\sigma$-Algebra und $\mu$ ist ein vollständiges Maß auf $\mathscr M(\mu)$.
\end{Satz}

%Lemma I.7
\begin{Lemma}
Sei $\mu$ äußeres Maß und seien $A_i \in \mathscr M(\mu), i \in \mathbb N$. Dann gelten:
\begin{enumerate}
	\item aus $A_1 \subset A_2 \subset ...$ folgt $\mu\left(\bigcup\limits_{i=1}^\infty A_i\right) = \lim_{i \rightarrow \infty} \mu(A_i)$ 
	\item aus $A_1 \supset A_2 \supset ...$ folgt $\mu\left(\bigcap\limits_{i=1}^\infty A_i\right) = \lim_{i \rightarrow \infty} \mu(A_i)$, falls $\mu(A_1) \less \infty$
\end{enumerate}
\end{Lemma}


%hö, das sollte aber I.5 sein. Was ist wo I.4??
\section{Fortsetzungssatz von Carathéodory}

%Def I.12
\begin{Definition}
Ein Mengensystem $\mathscr A \in P(X)$ heißt \underline{$\cap$-stabil}, \underline{ $\cup$-stabil }, \underline{$\backslash$-stabil}, falls mit $A, B \in \mathscr A$ auch $A \cap B \in \mathscr A, A \cup B \in \mathscr A, A\backslash B \in \mathscr A$.
\end{Definition}

%Bem.
\begin{Bemerkung}
$\cap$-stabil $(\cup$-stabil$)$ $\Rightarrow$ $\cap$-Stabilität $($bzw. $\cup)$ von endlichen Durchschnitten.
\end{Bemerkung}

%Def I.13.
\begin{Definition}
Ein Mengensystem $\mathscr R \in P(X)$ heißt \underline{Ring} über $X$, falls
\begin{enumerate}
	\item $\emptyset \in \mathscr R$
	\item $A, B \in \mathscr R \Rightarrow A\backslash B \in \mathscr R$
	\item $A, B \in \mathscr R \Rightarrow A \cup B \in \mathscr R$
\end{enumerate}
Ist $X \in \mathscr R$, so heißt $\mathscr R$ \underline{Algebra}.
\end{Definition}


%Bsp.
\begin{Beispiel}
\begin{enumerate}
	\item Für $A \subset X$ ist $\{\emptyset, A\}$ ein Ring. Für $A \ne X$ ist $\{\emptyset, A\}$ keine Algebra. $P(X)$ ist Algebra
	\item $\{$endliche Teilmengen von $X\}$ Ring über $X$, ebenso $\{ $abzählbare Teilmengen von $X\}$
\end{enumerate}
\end{Beispiel}

%Bem.
\begin{Bemerkung}
\begin{enumerate}
	\item $A, B \in \mathscr R$ (Ring) $\Rightarrow A \cap B = A \backslash (A \backslash B) \in \mathscr R$
	 \item Satz 1.1 ist auch für Ringe bzw. Algebren richtig; Satz 1.1 $\rightarrow$ erzeugte $\sigma$-Algebra. Also können wir auch erzeugte Ringe und erzeugte Algebren definieren.
\end{enumerate}
\end{Bemerkung}


%Def I.14
\begin{Definition}
Sei $\mathscr R \subset P(X)$ ein Ring. Eine Funktion $\lambda: \mathscr R \rightarrow [0, \infty]$ heißt \underline{Prämaß} auf $\mathscr R$, falls
\begin{enumerate}
	\item $\lambda(\emptyset) = 0$
	\item für paarweise disjunkte $A_i \in \mathscr R, i \in \mathbb N$ mit $\bigcup\limits_{i=1}^\infty A_i \in \mathscr R$: \\
	$\lambda \left(\bigcup\limits_{i=1}^\infty A_i\right) = \sum\limits_{i=1}^\infty \lambda(A_i)$
\end{enumerate}
\end{Definition}

%Bem.
\begin{Bemerkung}
Die Begriffe $\sigma$-subadditiv, $\sigma$-endlich, endlich, monoton, Nullmenge werden wie bei Maßen definiert
\end{Bemerkung}

%Bsp.
\begin{Beispiel}
\begin{enumerate}
	\item $\mathscr R$ Ring über $X$,  
	\begin{equation}
	\lambda(A) =
		\begin{cases}
			0 & A = \emptyset \\
			\infty & \text{sonst}
		\end{cases}
	\end{equation}
	\item $\mathscr R = \{$endliche Teilmengen vom $X\}$, $\lambda = $ $card|_{\mathscr R}$ ist Prämaß
	\item alle äußeren Maße sind Prämaße, wenn man sie einschränkt auf $\mathscr R = \mathscr M(\mu)$
\end{enumerate}
\end{Beispiel}

%Def I.15
\begin{Definition}
Sei $\lambda$ Prämaß auf Ring $\mathscr R \subset P(X)$. Ein äußeres Maß $\mu$ auf $X$ (bzw. ein Maß $\mu$ auf $\mathscr A$) heißt \underline{Fortsetzung} von $\lambda$, falls gilt:
\begin{enumerate}
	\item $\mu|_\mathscr R = \lambda$, d.h. $\mu(A) = \lambda(A)$ $\forall A \in \mathscr R$
	\item $\mathscr R \in \mathscr M(\mu) ($bzw. $\mathscr R \subset \mathscr A)$
\end{enumerate}
\end{Definition}


%Satz I.11, VL 06.11.15
\begin{Satz}
$\lambda: R \rightarrow [0, \infty], {}R \subset P(X)$ (Prämaß auf Ring), $\forall {}E \subset X:$ \\
\begin{center}
$\mu(E):= inf\{ \sum\limits_{i=1}^{\infty} \lambda(A_i) | E \subset \bigcup\limits_{i=1}^{\infty} A_i, A_i \in \mathbb{R}\}$
\end{center}
ist Fortsetzung von $\lambda$. $(\mu$ ist das durch $\lambda$ induzierte äußere Maß$)$
\end{Satz}

%Lemma I.6
\begin{Lemma}
Sei $\mu$ \cara{} des Prämaß $\lambda$ auf $R$. Sei $\tilde{\mu}$ ein Maß auf $\sigma(R)$ mit $\tilde{\mu} = \lambda$ auf $R$. Dann gilt $\tilde{\mu}(R) \leq \mu(E)$ $\forall E \in \sigma(R)$.
\end{Lemma}

%Satz I.12 
\begin{Satz} \emph{Hopf-Fortsetzung} \\
Sei $\lambda: R \rightarrow [0, \infty]$ Prämaß auf Ring $R \subset P(X)$. Dann existiert ein Maß $\mu$ auf $\sigma(R)$ mit $\mu =\lambda$ auf $R$. Diese Fortsetzung ist \underline{eindeutig}, falls $\lambda$ $\sigma$-endlich ist.
\end{Satz}

\emph{Notation:}
Sei $\mu$ Maß auf $X$ mit der Eigenschaft: $\forall D \subset X: \exists E \in \mathscr{M}(\mu)$ mit $D \subset E$ und $\mu(D) = \mu(E)$. Dann heißt $\mu$ \underline{reguläres} äußeres Maß.

\begin{Beispiel}
Blödes Beispiel
\end{Beispiel}

%Satz I.13
\begin{Satz}
Sei $\mu$ \cara{} des Prämaßes $\lambda$ auf $R$. Dann existiert zu jedem $D \subset X$ ein $E \in \sigma(R) \subset \mathscr M(\mu)$ mit $E \supset D$ und $\mu(D) = \mu(E)$
\end{Satz}

%Satz I.14
\begin{Satz}
$\lambda$  $\sigma$-endliches Prämaß auf Ring $\mathscr R$ und $\mu$ sei \cara{} von $\lambda$. Dann ist $\mu|_{\mathscr M(\mu)}$ die \underline{Vervollständigung} von $\mu|_{\sigma(\mathscr R)}$. Insbesondere gibt es genau eine Fortsetzung von $\lambda$ zu einem Maß auf $\mathscr M(\mu)$.
\end{Satz}

%Lemma I.9
\begin{Lemma}
Sei $\lambda: \mathscr R \rightarrow [0, \infty]$ Prämaß auf Ring $\mathscr R$ und $\lambda$ sei $\sigma$-endlich. $\mu$ sei die \cara{} von $\lambda$. $D \subset X$ ist genau dann $\mu$-messbar, falls eine der beiden folgenden Bedingungen gilt:
\begin{enumerate}
	\item $\exists E \in \sigma(\mathscr R)$ mit $E \supset D$ und $\mu(E \backslash D) = 0$
	\item $\exists C \in \sigma(\mathscr R)$ mit $C \subset D$ und $\mu(D \backslash C) = 0$
\end{enumerate}
\end{Lemma}

%Michelle 09.11.15
Prämaß auf Ring kann zu einem \underline{Maß} fortgesetzt werden: \\
\underline{Ziel}: Volumen (Flächeninhalt) einer Menge $\Omega \subset \mathbb R^n$ zu berechnen ($\rightarrow$ Maß einer Menge)


%I.5 Halbringe und Inhalte
\section{Halbringe und Inhalte}
%Def. I.16
\begin{Definition}
Ein Mengensystem $\mathscr Q \subset P(X)$ heißt \underline{Halbring} über $X$, falls
\begin{enumerate}
	\item $\emptyset \in \mathscr Q$
	\item $P, Q \in \mathscr Q \Rightarrow P \cap Q \in \mathscr Q$
	\item $P, Q \in \mathscr Q \Rightarrow P \backslash Q = \bigcup\limits_{i=1}^k P_i$ mit $P_i \in \mathscr Q$ paarweise disjunkt.
\end{enumerate}
\end{Definition}

%Bsp.
\begin{Beispiel}
\begin{enumerate}
	\item $X$ beliebige Menge. $Q := \{\emptyset\} \cup \{\{a\}:  a \in X\}$ ist Halbring 
	\item $I \subset \mathbb R$ heißt \underline{Intervall}, falls $a, b \in \mathbb R$ existieren mit $a \leq b$ und $(a, b) \subset \overline{I} \subset [a, b]$. Quader in $\mathbb R^n$ $ Q:= I_1 \times ... \times I_n$,  $I_1, ... I_n$ Intervalle.
\end{enumerate}
\end{Beispiel}

%Satz I.15
\begin{Satz}
$Q^n := \{Q \in \mathbb R^n: Q $ Quader$\}$ ist Halbring über $\mathbb R^n$
\end{Satz}

%Satz I.16
\begin{Satz}
$\mathscr Q$ Halbring über $X$ und $\mathscr F := \{F = \bigcup\limits_{i=1}^k P_i, k \in \mathbb N, P_i \in \mathscr Q$ $\forall 1 \leq i \leq k\}$. Dann ist $\mathscr F$ der von $\mathscr Q$ erzeugte Ring.
\end{Satz}

%Bsp.
\begin{Beispiel}
\begin{enumerate}
	\item $\mathscr Q = \{\emptyset \} \cup \{\{a\}: a \in X\} \rightarrow \mathscr F = \{$ endliche Teilmengen von $X\}$
	\item $\mathscr Q^n \rightarrow$ E erzeugter Ring $\mathscr F^n:= \{$endliche Vereinigung von Quadern$\}$: Figuren
\end{enumerate}
\end{Beispiel}

%Lemma I.10
\begin{Lemma}
Sei $\mathscr Q$ Halbring über $X$ und $\mathscr F$ der von $\mathscr Q$ erzeugte Ring. Dann gilt: $\sigma(\mathscr Q) = \sigma(\mathscr F)$, $(\sigma(.)$: erzeugte $\sigma$-Algebra$)$
\end{Lemma}

%Lemma I.11
\begin{Lemma}
Sei $\mathscr Q$ Halbring über $X$ und $\mathscr F$ der von $\mathscr Q$ erzeugte Ring. Zu jedem $F \in \mathscr F$ existieren $P_1, ... P_k \in \mathscr Q$ \underline{paarweise disjunkt} mit $F = \bigcup\limits_{i=1}^k P_i$
\end{Lemma}

%Def I.17
\begin{Definition}
Sei $\mathscr Q \subset P(X)$ Halbring über $X$. Eine Funktione $\lambda: \mathscr Q \rightarrow [0, \infty ]$ heißt \underline{Inhalt} auf $\mathscr Q$, falls
\begin{enumerate}
	\item $\lambda(\emptyset) = 0$
	\item $\lambda\left(\bigcup\limits_{i=1}^n	A_i\right) = \sum\limits_{i=1}^n \lambda(A_i)$ für paarweise disjunkte $A_i \in \mathscr Q$ mit $\bigcup\limits_{i=1}^n A_i \in \mathscr Q$
\end{enumerate}
Ein Inhalt heißt \underline{Prämaß} auf $\mathscr Q$, falls $\lambda$ $\sigma$-additiv auf $\mathscr Q$ ist, d.h. für paarweise disjunkte $A_i, i \in \mathbb N$ gilt: $\lambda\left(\bigcup\limits_{i\in \mathbb N} A_i \right) = \sum\limits_{i=1}^\infty \lambda(A_i)$, falls $\bigcup\limits_{i \in \mathbb N} A_i \in \mathscr Q$
\end{Definition}

%Bem.
\begin{Bemerkung}
\begin{enumerate}
	\item Ist $\mathscr Q$ Ring, so stimmt die Definition von Prämaß auf \emph{Def. 1.14} überein
	\item Die Begriffe $\sigma$-subadditiv, $\sigma$-endlich, monoton, Nullmenge übertragen sich direkt auf Inhalte.
\end{enumerate}
\end{Bemerkung}

%Satz I.17
\begin{Satz}
Sei $\lambda$ Inhalt auf Halbring $\mathscr Q$ über $X$ und sei $\mathscr F$ der von $\mathscr Q$ erzeugte Ring. Dann gibt es genau einen Inhalt $\overline{\lambda}: \mathscr F \rightarrow [0, \infty]$ mit $\overline{\lambda}(Q) = \lambda(Q)$ $\forall Q \in \mathscr Q$
\end{Satz}

%Fehlt hier noch was?? Die Vorlesung nach dem 09.11. war über was??
%
%
%


\newpage
%Kapitel 2

\chapter{Das Lebesgue -Maß und -Integral}
\section{Das Lebesgue-Maß}

Halbring $\mathcal{Q}^n$ Inhalt.\\
$vol^n : \mathcal{Q}^n \rightarrow [0, \infty]$

\begin{Satz}
	Prämaß auf Halbring\\
	$\Rightarrow$ \cara{} ist äußeres Maß auf $P(X)$.
\end{Satz}

\begin{Lemma}
	$vol^n : \mathcal{Q}^n \rightarrow [0, \infty]$ ist ein Prämaß.
\end{Lemma}

\begin{Definition}
	Das (äußere) Lebesguemaß einer Menge $E \subset \mathbb{R}^n$ ist definiert durch\\
	$ \mathscr{L}^n(E) := inf \left\{ \sum_{i = 1}^{\infty} vol^n(P_i) \mid P_i \in \mathscr{Q}^n, E \subset \bigcup\limits_{i = 1}^\infty P_i \right\}$
\end{Definition}




\newpage


%Ich brauche Beschäftigung: VL 23.11.15: ab Satz II.5
%31.12.15 spät abends. Noch ist der folgende Satz II.5 im pdf als II.2 gezählt

%Satz II.5 
\begin{Satz}
Sei $U \subseteq \mathbb R^n$ offen und $f: U \rightarrow \mathbb R^n$ lokal Lipschitz $($z.B. $f \in \mathscr C^1(U, \mathbb R^n))$, so gilt:
\begin{enumerate}
	\item $N \subset U$ $\mathscr L^n$-Nullmenge $\Rightarrow$ $f(N)$ $\mathscr L^n$-Nullmenge
	\item $E \subset U$ $\mathscr L^n$-messbar $\Rightarrow$ $f(E)$ $\mathscr L^n$-messbar.
\end{enumerate}
\end{Satz}


%Satz II.6
\begin{Satz}
Für $S \in O(n)$ und $a \in \mathbb R^n$ gilt: 
$\mathscr L^n (S(E) +a) = \mathscr L^n(E)$ $\forall E \subset \mathbb R^n$
\end{Satz}


%Lemma II.5
\begin{Lemma}
$\forall S \in GL(n)$ $\exists$ Diagonalmatrix $\Lambda$ mit Einträgen $\lambda_i > 0$ und $T_1, T_2 \in O(n)$, sodass $S = T_1\cdotp\Lambda\cdotp T_2$ $($Polarzerlegung$)$
\end{Lemma}


%Satz II.7
\begin{Satz}
\emph{Lineare Transformationsformel} \\
Für eine lineare Abbildung $S: \mathbb R^n \rightarrow \mathbb R^n$ gilt: $\mathscr L^n(S(E)) = |\det S| \mathscr L^n(E)$ $\forall E \subset \mathbb R^n$
\end{Satz}


%YES! Erste Matrix! Sogar mit Punkten!! :D
%Bsp Ellipsoid
\begin{Beispiel}
\emph{Ellipsoid} \\
$\lambda_1, ..., \lambda_n = 0$, $E = \left\{ x \in \mathbb R^n | \left( \dfrac{x_1}{\lambda_1} \right)^2 + ... + \left (\dfrac{x_n}{\lambda_n}\right)^2 < 1 \right\}$ \\
Jetzt gilt: $E = \Lambda(\mathscr B_1(0))$, $\Lambda =$
$\begin{pmatrix}
      \lambda_1 & 	 &	&	& 0 \\
      		 &  	\ddots \\
		 &	& \ddots \\
		 &	&	& \ddots \\
		 0 &	&	&	&	\lambda_n
\end{pmatrix}$
$\Rightarrow \mathscr L^n(E) = (\lambda_1 \cdotp ... \cdotp \lambda_n)\mathscr L^n(\mathscr B_1(0))$
\end{Beispiel}


%Bsp Vitali
\begin{Beispiel}
\emph{Vitali 1905} \\
$\exists S \subset [0, 1]$, die nicht $\mathscr L^1$-messbar ist. Auf $[0, 1]$ gibt es die Äquivalenzrelation 
\begin{center}
	$x \sim y \Leftrightarrow x - y \in \mathbb Q$
\end{center}
Auswahlaxiom: $\exists$ Repräsentantensystem $S \subset [0, 1]$ für $\sim$, d.h. $\forall y \in [0, 1]$ $\exists! x \in S: x \sim y$ \\
Sei jetzt $q_1, q_2, ...$ eine Abzählung von $\mathbb Q \cap [-1, 1]$. Dann gilt: $(q_j + S) \cap (q_k +S)  \emptyset $ $\forall j \ne k$ \\
$($Sonst existieren $x_1, x_2 \in S$ mit $q_j +x_1 = q_k + x_2 \Rightarrow x_1-x_2 = q_k -q_j \in \mathbb Q \Rightarrow x_1 \sim x_2  \Rightarrow x_1 = x_2 \Rightarrow q_j = q_k $\blitzd $ )$ \\

\emph{Behauptung:} $[0, 1]$ $\underbrace{\subset}_{1)}$ $\bigcup\limits_{k=1}^\infty
 (q_k + S) \subset [-1, 2]$ \\
 Zu $1)$: zu $y \in (0, 1)$ existiert $x \in S: y - x = q \in \mathbb Q \cap [-1, 1] \Rightarrow$ $\exists k: y \in q_k + S$ \\
\emph{Annahme:} $S$ ist $\mathscr L^1$-messbar mit $\mathscr L^1(S) = \mathscr L^1(q + S)$ $\forall q \in \mathbb Q$ folgt: $\sum\limits_{k=1}^\infty \mathscr L^1(S) = \sum\limits_{k=1}^\infty \mathscr L^1(q_k + S) \underbrace{=}_ {\text{Satz I.10}} \mathscr L^1\left(\bigcup\limits_{k=1}^\infty (q_k + S) \right) \in (1, 3]$ \blitzd
\end{Beispiel}


%27.11.15

%der erste Absatz ist nicht wirklich beschriftet, sondern einfach so da, ich tex ihn mal einfach ab, habe das geschwungene tau aber noch nie in Ana gesehen^^

%
$(X, \mathscr A)$ messbarer Raum \\
$\rightarrow \mathscr A-TF$ \\
$\rightarrow \tau_\mathscr A ^+ = \left\{\phi \in \tau_\mathscr A | \phi \geq 0\right\}$ \\
Jetzt $(X, \mathscr A, \mu)$ Maßraum. Neue Notation: \underline{$\tau^+ (\mu)$} statt $\tau_\mathscr A ^+$ \\
Definiere für $\phi \in \tau+(\mu)$ mit Wertemenge $\{s_1, ... , s_l\} \subset [0, \infty)$ das \emph{Integral}: \\
$I(\phi) = \sum\limits_{i=1}^l s_i \mu(\{\phi = s_i\}) \in [0, \infty]$


%Lemma II.6
\begin{Lemma}
Für $\phi, \psi \in \tau	^+(\mu)$ und $\alpha, \beta \in [0, \infty)$ gelten:
\begin{enumerate}
	\item $I(\alpha\phi + \beta\psi) = \alpha I(\phi) + \beta I(\psi)$
	\item $\phi \leq \psi \Rightarrow I(\phi) \leq I(\psi)$ 
\end{enumerate}
\end{Lemma}


%Def II.4
\begin{Definition}
Sei $(X, \mathscr A, \mu)$ ein Maßraum und $f: X \rightarrow \overline{\mathbb R} \mathscr A$-messbar. Ist $f: X \rightarrow [0, \infty]$, so setzen wir:
$\int f d\mu := \sup\left\{I(\phi) | \phi \in \tau^+(\mu), \phi \leq f\right\}$ $(\phi$ heißt \underline{Unterfunktion} in dieser Definition$)$ \\

Ist $f: X \rightarrow \overline{\mathbb R}$ und sind die Integrale von $f^+ = \max{f, 0}$ und $f^- = \max{-f, 0}$ wie eben definiert und nicht beide unendlich, so setzen wir: \\
$\int f d\mu := \int f^+ d\mu - \int f^- d\mu \in [-\infty, \infty]$ \\
$(f = f^+ - f^-)$ \\
$($Das \underline{Lebesgue-Integral}!$)$
\end{Definition}

%Lemma II.7
\begin{Lemma}
Für $f \in \tau^+ (\mu)$ gilt: $\int f d\mu = I(f)$
\end{Lemma}

%Bsp.
\begin{Beispiel}
$\mathscr L^1$ auf $\mathbb R$; $\mathscr A = \mathscr M(\mathscr L^1), \chi_\mathbb Q \in \tau^+ (\mathscr L^1)$ \\
$\mathbb Q \in \mathscr B^1$ $($Blatt 2$)$ \\
$\int \chi_\mathscr Q d\mathscr L^1 = 0 \cdotp \mathscr L^1(\mathbb R \backslash \mathbb Q) + 1 \cdotp \mathscr L^1(\mathbb Q) = 0$
\end{Beispiel}

%Def II.5
\begin{Definition}
Eine Funktion $f: X \rightarrow \overline{\mathbb R}; ((X, \mathscr A, \mu)$ Maßraum$)$ heißt \underline{integrierbar bzgl. $\mu$}, wenn sie \underline{$\mathscr A$-messbar} ist und wenn gilt: \\
$\int f d\mu \in \mathbb R$ $\Leftrightarrow$ $\int f^+ d\mu + \int f^- d\mu < \infty$
\end{Definition}

%Satz II.9
\begin{Satz}
Sei $(X, \mathscr A, \mu)$ Maßraum und $f, g: X \rightarrow \overline{\mathbb R}$ seien $\mathscr A$-messbar. Ist $f \leq g$ $\mu$-fast unendlich und $\int f d\mu > -\infty$, so existiert auch $\int g d\mu$ und es gilt: \\
$\int f d\mu \leq \int g d\mu$ \\
Analog:  Ist $f \geq g$ $\mu$-fast unendlich und $\int f d\mu < \infty$, so existiert auch $\int g d\mu$ und es gilt: \\
$\int f d\mu \geq \int g d\mu$
\end{Satz}

%Bem.
\begin{Bemerkung}
Sei $\mu$ vollständiges Maß, $f$ $\mu$-messbar und $f = g$ $\mu$-fast überall \\
$\xRightarrow{\text{Lemma I.5}} g$ $\mu$-messbar \\
$\Rightarrow \int f d\mu = \int g d\mu$, falls einer der beiden existiert.
\end{Bemerkung}

%Lemma II.8: Tschebyscheff-Ungleichung
\begin{Lemma}
\emph{Tschebyscheff-Ungleichung} \\
Für eine $\mathscr A$-messbare Funktion $f: X \rightarrow [0, \infty]$ mit $\int f d\mu < \infty$ gilt: \\
\begin{equation}
	\mu(\{f \geq s\}) \leq
	\begin{cases}
		\dfrac{1}{s}\int f d\mu & s \in (0, \infty) \\
		0 & s = \infty
	\end{cases}
\end{equation}
\end{Lemma}

%Lemma II.9
\begin{Lemma}
$f: X \rightarrow \overline{\mathbb R}$ sei $\mathscr A$-messbar. 
\begin{enumerate}
	\item Ist $\int f d\mu < \infty \Rightarrow \{f = \infty\}$ ist $\mu$-Nullmenge.
	\item Ist $f \geq 0$ und $\int f d\mu = 0 \Rightarrow \{f > 0\}$ ist $\mu$-Nullmenge. 
\end{enumerate}
\end{Lemma}


\end{document}
