\documentclass[11pt]{memoir}

\usepackage[utf8]{inputenc}
\usepackage[ngerman]{babel}
\usepackage{mathrsfs} %hübsche Buchstaben
\usepackage{amssymb}
\usepackage{ntheorem}
\everymath{\displaystyle}
\usepackage{amsmath, amssymb} %Damit geht dann cases
%\usepackage{array}
\usepackage{mathabx} %für #
%\usepackage{amsfonts} Wie bekommt man Underrightarrow{bla}?
%\usepackage{fontenc} % Für "
\usepackage{ulsy} %Für Widerspruchspfeil
\usepackage{mathtools} %Für Rightarrow mit Wort darüberunderbrace
\usepackage{extpfeil} %für = mit Wort drüber
\usepackage{fdsymbol} %varrightwavearrow - blockiert Underbrace: muss nach mathtools kommen, sonst hässliche Balken.
\usepackage{braket} % Für schönere Mengen. Oder so. Hier ist z. B. \Set drin und die vertikalen Balken sind schön.


%
%13.11 Freitag Younis
%16.11 Montag Younis done
%20.11 Freitag T
%11.12 von Michelle abgetext, fast nicht lesbar: KONTROLLE
%21.12 von Michelle

\theoremstyle{changebreak}
\theoremseparator{\medskip}

\newtheorem{Definition}{Definition}[chapter]
\newtheorem{Bemerkung}{Bemerkung}[chapter]
\newtheorem{Beispiel}{Beispiel}[chapter]
\newtheorem{Lemma}{Lemma}[chapter]
\newtheorem{Satz}{Satz}[chapter]

\newcommand{\cara}{Carathéodory-Fortsetzung}
\newcommand\overequal[1]{\mathrel{\overset{\makebox[0pt]{\mbox{\normalfont\tiny\sffamily $ #1 $}}}{=}}}
\newcommand{\quotes}[1]{``#1''}
\newcommand{\umgf}{$m$-dimensionale Untermannigfaltigkeit des $\mathbb R^n$}

\begin{document}

\title{\textbf{Analysis III}\\ Skript zur Vorlesung von Prof. Dr. Tobias Lamm}
\author{Tamar Mirbach\\ Younis Bensalah}
\date{Wintersemester 2015/16\\ Karlsruher Institut für Technologie}

\maketitle

\chapter{Maße und messbare Funktionen}
\section{$\sigma$-Algebren und Maße}


\emph{Notation}: $X$ Menge, \\ $P(X) =$ \{Teilmengen von $X$\}  {} Potenzmenge \\
$A$ $\subset$  $P(X)$ Mengensystem
%Def. I.1
\begin{Definition}
Ein Mengensystem $\mathscr{A} \subset P(X)$ heißt \underline{$\sigma$-Algebra}, falls
\begin{enumerate}
	\item $X \in \mathscr{A}$
	\item $A \in \mathscr{A} \Rightarrow X \backslash A \in \mathscr{A}$
	\item $A_i \in \mathscr{A}$, $i \in$ $\mathbb{N} \Rightarrow \bigcup_{i = 1}^{\infty} A_i$
\end{enumerate}
Das System $(X, \mathscr{A})$ heißt \underline{messbarer Raum}.
\end{Definition}

%Bem.
\begin{Bemerkung}
\begin{enumerate}
	\item $A_i \in \mathscr{A}$, $i \in \mathbb{N} \Rightarrow \bigcap_{i = 1}^{\infty} A_i$, denn:
	$\bigcap_{i=1}^{\infty} A_i  = X \backslash (\bigcup_{i=1}^{\infty} (X \backslash A_i))$
	\item $\emptyset \in \mathscr{A}$
	\item $A, B \in \mathscr{A} \Rightarrow A \backslash B \in \mathscr{A}$, denn $A \backslash B = A \cap (X \backslash B)$
\end{enumerate}
\end{Bemerkung}

%Bsp.
\begin{Beispiel}
\begin{enumerate}
	\item $P(X)$ trivial
	\item $\{ \emptyset, X \}${} trivial
\end{enumerate}
\end{Beispiel}

%Satz I.1
\begin{Satz}
Jeder Durchschnitt von (endlich oder $\infty$-vielen) $\sigma$-Algebren auf derselben Menge $X$ ist eine $\sigma$-Algebra auf $X$.
\end{Satz}

%Def I.2
\begin{Definition}
Für ein Mengensystem $E \subset P(X)$ heißt \\
$\sigma(E):=\bigcap \{ \mathscr{A}: \mathscr{A}$ ist $\sigma$-Algebra in $X$ und $E$ $\subset \mathscr{A} \}$ die \underline{von $E$ erzeugte $\sigma$-Algebra}.
\end{Definition}

%Bem.
\begin{Bemerkung}
\begin{enumerate}
	\item $P(X)$ $\sigma$-Algebra mit $E \subset P(X)$
	\item Satz 1 $\Rightarrow \sigma(E)$ ist $\sigma$-Algebra
	\item $\mathscr{A}$ $\sigma$-Algebra mit $E \subset \mathscr{A} \Rightarrow \sigma(E) \subset 		\mathscr{A}$ "kleinste"{} $\sigma$-Algebra, die $E$ enthält.
\end{enumerate}
\end{Bemerkung}

%Bsp.
\begin{Beispiel}
\begin{enumerate}
	\item $E \subset X$ $\Rightarrow $ $\sigma(E)$ = \textbraceleft $X$, $E$, $\emptyset$, $X \setminus E$\textbraceright
	\item $(X, d)$ sei metrischer Raum. $\mathscr{O}:=$ \textbraceleft offene Teilmenge von $X$ \\
	$\sigma(\mathscr{O})=:\mathscr{B(O)}$ \underline{Borel $\sigma$-Algebra}. Elemente darin heißen \underline{Borelmengen} \\
	$X = \mathbb{R}^{n}: \mathscr{B(O)} = \mathscr{B}^{n}$ \\
	$\overline{\mathbb{R}} := \mathbb{R}$  $\cup$ \{t$\pm\infty$\} {}($a < b$ Anordnung)
\end{enumerate}
\end{Beispiel}

%Def I.3
\begin{Definition}
Eine Folge $(s_{k}) \subset \overline{\mathbb{R}}$ \underline{konvergiert} gegen $s \in \mathbb{R}$, falls eine der folgenden Alternativen gilt:
\begin{itemize}
	\item  $s \in \mathbb{R}$ und $\forall \epsilon > 0$ gilt $s_{k} \in (s-\epsilon, s+\epsilon)$ $\forall k$ groß genug
	\item $s = \infty$ und $\forall r \in \mathbb{R}: s_{k} \in (r, \infty]$ $\forall k$ groß genug
	\item $s = -\infty$ und $\forall r \in \mathbb{R}: s_k \in (-\infty, r)$ $\forall k$ groß genug
\end{itemize}
\end{Definition}

%Bem.
\begin{Bemerkung}
\begin{enumerate}
	\item $(a_n)$ monoton wachsende Folge
	$\Rightarrow \lim a_n \in \mathbb{\overline{R}}$
	\item $\sum\limits_{n=1}^{\infty} a_n \in \mathbb{\overline{R}}$, $a_n \geq 0$ $ \forall n \in \mathbb{N}$
	\item offene Teilmengen von $\overline{\mathbb{R}}$: \\
	$U \subset \overline{\mathbb{R}} $ offen $\Leftrightarrow U \cap \mathbb{R}$ offen und falls $\{+\infty\}$ oder $\{-\infty\}$ in $M$ liegt, existiert ein $a \in \mathbb{R}$ mit $(a,{} \infty ]\subset U$ bzw. $[-\infty, a) \subset U$
\end{enumerate}
\end{Bemerkung}
 % Hier fehlt noch eine kleine Tabelle

 %Def I.4
\begin{Definition}
$\mathscr{A} \subset P(X)$ $\sigma$-Algebra. Eine nicht negative Mengenfunktion $\mu: \mathscr{A} \rightarrow [0, \infty]$ heißt \underline{Maß} auf $\mathscr{A}$, wenn
\begin{enumerate}
	\item $\mu(\infty) = 0$
	\item für paarweise disjunkte Mengen $A_i \in \mathscr{A}, i \in \mathbb{N}$ gilt: \\
	$\mu(\bigcup\limits_{i=1}^{\infty}A_i) = \sum\limits_{i=1}^{\infty} \mu(A_i)$ \underline{$\sigma$-Additivität}
\end{enumerate}
Das Tripel $(X, \mathscr{A}, \mu)$ wird als Maßraum bezeichnet.
\end{Definition}

%Bem
\begin{Bemerkung}
Monotonie: $A \subset B \Rightarrow \mu(A) \leq \mu(B)$\\
$\mu(B) = \mu(A \cup B\backslash A) = \mu(A) + \underbrace{\mu(B\backslash A)}_{\geq 0}\geq \mu(A)$
\end{Bemerkung}

%Def I.5
\begin{Definition}
$(X, \mathscr{A}, \mu)$ sei Maßraum: $\mu$ heißt \underline{endlich}, wenn $\mu(A)$ \textless {} $ \infty $ $\forall A \in \mathscr{A}$, $\mu$ heißt \underline{$\sigma$-endlich}, falls eine Folge $(X_i) \subset \mathscr{A}$ mit $\mu(X_i)$ \textless {} $ \infty $ existiert, sodass $X = \bigcup\limits_{i=1}^{\infty} X_i$. \\
Falls $\mu(X) = 1$, so wird $\mu$ \underline{Wahrscheinlichkeitsmaß} genannt.
\end{Definition}


%Bsp.
\begin{Beispiel}
\begin{enumerate}
	\item $X$ Menge, $\mathscr{A} = P(X), x \in X$ \underline{Diracmaß}
	\begin{equation}
		\delta_{x} (A) =
		\begin{cases}
			1 & \text{falls } x \in A\\
			0 & \text{sonst}
		\end{cases}
	\end{equation}
	\begin{itemize}
		\item $\delta_x ( \emptyset) = 0$
		\item Sei $A = \bigcup\limits_{i=1}^{\infty} A_i$ und $A_i$ sind paarweise disjunkt.
		\begin{enumerate}
			\item $x \in A$, $\delta_x(A) = 1 = \sum\limits_{i=1}^{\infty} \delta_x(A_i)$, $x \in A_i$ für genau ein i
			\item $x \notin A \Rightarrow \delta_x(A) = 0 = \sum\limits_{i=1}^{\infty}A_i)$
					\end{enumerate}
		Weil $\delta_x(X) = 1$, ist $\delta_x$ ein Wahrscheinlichkeitsmaß.
	\end{itemize}
	\item $X$ beliebig, $\mathscr{A} = P(X)$: \underline{Zählmaß} card: $P(X) \rightarrow [0, +\infty]$
	\begin{equation}
		card(A) :=
		\begin{cases}
			\varhash \text{ Elemente von } A & A \text{ endlich} \\
			\infty & \text{sonst}
		\end{cases}
	\end{equation}
	$card(\emptyset) = 0$ \\
	$A = \bigcup\limits_{i=1}^{\infty} A_i$ paarweise disjunkt.
	\begin{enumerate}
		\item $A$ endlich $\Rightarrow$ endliche Vereinigung von endlichen Mengen
		\item $A$ nicht endlich
		\begin{enumerate}
			\item ein $A_i$ ist nicht endlich
			\item $\forall A_i$ endlich
		\end{enumerate}
		card endlich $\Leftrightarrow X$ endlich \\
		card ist $\sigma$-endlich $\Leftrightarrow X$ abzählbar
	\end{enumerate}
\end{enumerate}
\end{Beispiel}

%Satz I.2
\begin{Satz}
$(X, \mathscr{A}, \mu)$ sei Maßraum. Dann gelten für $A_i \in \mathscr{A}, i \in \mathbb{N}$ die Aussagen
\begin{enumerate}
	\item $A_1 \subset A_2 \subset ...  A_i \subset A_{i+1} \subset .. $\\
	$\Rightarrow \mu(\bigcup\limits_{i=1}^{\infty} A_i) = \lim_{i \rightarrow \infty} \mu(A_i)$
	\item Aus $A_1 \supset A_2 \supset ... \supset A_i \supset A_{i+1} \supset ...$ mit $\mu(A_1)$ \textless  {} $ \infty$ folgt\\
	$\mu(\bigcap\limits_{i=1}^{\infty} A_i) = \lim_{i \rightarrow \infty} \mu(A_i)$
	\item $\mu(\bigcup\limits_{i=1}^{\infty} A_i) \leq \sum\limits_{i=1}^{\infty} \mu(A_i)$ \\
	($A_i$ müssen hier nicht paarweise disjunkt sein)
\end{enumerate}
\end{Satz}

%Bem
\begin{Bemerkung}
\begin{enumerate}
	\item \emph{1.} heißt \underline{Stetigkeit von unten} \\
	\emph{2.} heißt \underline{Stetigkeit von oben} \\
	\emph{3.} \underline{$\sigma$-Subadditivität} des Maßes
	\item  $\mu(A_1$ \textless $\infty)$ in \emph{2.} kann durch $\mu(A_k) $ \textless  $\infty$ für ein $k \in \mathbb{N}$ ersetzt werden.
\end{enumerate}
\end{Bemerkung}

%Bsp.
\begin{Beispiel}
$A_k := \{k, k+1, ... \} \subset \mathbb{N}$, card($A_k$) \textgreater $ \infty$ \\
card($\bigcap\limits_{k=1}^{\infty} A_k$) $= 0 \ne 0 \lim_{k \rightarrow \infty} $card$ (A_k) = \infty$
\end{Beispiel}

%Def I.6
\begin{Definition}
Sei $(X, \mathscr{A}, \mu)$ Maßraum. Jede Menge $A \in \mathscr{A}$ mit $\mu(A) = 0$ heißt \underline{$\mu$-Nullmenge}. Das Maß heißt \underline{vollständig}, wenn $N \subset A $ für ein $A \subset \mathscr{A}$ mit $\mu(A) = 0 \Rightarrow N \in \mathscr{A}$ und $\mu(N) = 0$.
\end{Definition}

%Bsp.
\begin{Beispiel}
(das blöde Maß): \\
$\mu(A) = 0$ $ \forall A \in \mathscr{A}$ und $\mathscr{A} \ne P(X)$ ist \underline{nicht vollständig}
\end{Beispiel}

%Bem.
\begin{Bemerkung}
Jedes Maß kann vervollständigt werden (siehe Blatt $2$)
\end{Bemerkung}


\section{Messbare Funktionen}

%Def. I.7
\begin{Definition}
Seien $(X, \mathscr{A}), (Y, \mathscr{C})$ messbare Räume.\\
 Eine Abbildung $f: X \rightarrow Y$ heißt \underline{$\mathscr{A}-\mathscr{C}$-messbar}, wenn $f^{-1}(\mathscr{C}) \subset \mathscr{A}$, d.h. $f^{-1}(C) \in \mathscr{A}$ $ \forall C \in \mathscr{C}$
\end{Definition}

%Bem.
\begin{Bemerkung}
$f$ heißt kurz messbar (bzw. $\mathscr{A}$-messbar), wenn an $\mathscr{A}, \mathscr{C}$ (bzw. $\mathscr{C}$) kein Zweifel besteht.
\end{Bemerkung}

%Bsp
\begin{Beispiel}

\begin{enumerate}
	\item $(X, \mathscr{A}), (Y, \mathscr{C})$ beliebig. \\
	$f: X \rightarrow Y, f(x) = y_0 \in Y$ \\
	$\forall x \in X, c \in \mathscr{C}$:
	\begin{equation}
		f^{-1}(c) =
		\begin{cases}
			x & y_0 \in C \\
			\emptyset & y_0 \notin C
		\end{cases}
	\end{equation}
	\item Für $E \subset X$ beliebig heißt $\chi_E: X \rightarrow \mathbb{R}$, \\
	\begin{equation}
		\chi_E(x) =
		\begin{cases}
			1 & x \in E \\
			0 & \text{sonst}
		\end{cases}
	\end{equation}
	\underline{charakteristische Funktion} von $E$ auf $\mathbb{R}$. \\
	Sei $\mathscr{B}^1$ als $\sigma$-Algebra gegeben. \\
	$\chi_E$ ist $\mathscr{A}$-messbar $\Leftrightarrow$ $\{1\} = \bigcap\limits_k \left(1- \dfrac{1}{k}, 1 + \dfrac{1}{k}\right), E \in \mathscr{A}$
	\item $(X, \mathscr{A}), (Y, \mathscr{C}), (Z, \mathscr{D})$ messbare Räume \\
	$f: X \rightarrow Y$ $ \mathscr{A}-\mathscr{C}$-messbar \\
	$g: Y \rightarrow Z$ $ \mathscr{C}-\mathscr{D}$-messbar \\
	$\xRightarrow{Beh.}$ $  g \circ f: X \rightarrow Z$ $ \mathscr{A}-\mathscr{D}$-messbar, denn \\
	$(g \circ f)^{-1}(\mathscr{D}) = f^{-1}(g^{-1}(\mathscr{D})) \subset f^{-1}(\mathscr{D}) \subset \mathscr{A}$
\end{enumerate}
\end{Beispiel}

% Lemma I.1
\begin{Lemma}
$(X, \mathscr{A}), (Y, \mathscr{C})$ messbare Räume und $f: X \rightarrow Y$ Abbildung. Für beliebige Mengensysteme $\xi \in \mathscr{C}$ gilt $f^{-1}(\sigma(\xi)) = \sigma(f^{-1}(\xi))$
\end{Lemma}

% Lemma I.2
\begin{Lemma}
Seien $(X, \mathscr{A}), (Y, \mathscr{C})$ messbare Räume und $\mathscr{C} = \sigma(\xi)$ für ein Mengensystem $\xi \subset P(Y)$. Jede Abbildung $f: X \rightarrow Y$ mit $f^{-1}(\xi) \subset \mathscr{A}$ ist $\mathscr{A}-\mathscr{C}$-messbar.
\end{Lemma}

%Bsp
\begin{Beispiel}
\begin{enumerate}
	\item Jedes $f: \mathbb{R}^n \rightarrow \mathbb{R}^m$ stetig ist $\mathscr{B}^n-\mathscr{B}^m-$ messbar, denn $f^{-1}(\text{offen})$ ist offen. \\
	\emph{Notation}: $f$ ist \underline{Borel-messbar}
	\item $X \ne \emptyset$ Menge, $(Y, \mathscr{C})$ messbarer Raum, $f: X \rightarrow Y$ Abbildung. \\
	Blatt $\varrightwavearrow$ $f^{-1}(\mathscr{C})$ $\sigma$-Algebra. Tatsächlich ist es die kleinste $\sigma$-Algebra, die $f: X \rightarrow Y$ messbar macht. $f^{-1}(\mathscr{C})$ heißt die \underline{durch $f$ und $(Y, \mathscr{C})$ induzierte $\sigma$-Algebra}
\end{enumerate}
\end{Beispiel}

%Definition I.8
\begin{Definition}
Sei $(X, \mathscr{A})$ messbarer Raum und $D \in \mathscr{A}$. Eine Funktion $f: D \rightarrow \overline{\mathbb{R}}$ heißt \underline{numerische Funktion}.
\end{Definition}

%Bem
\begin{Bemerkung}
Numerische Funktionen sind $\mathscr{A}$-messbar (auf $D$), wenn $f^{-1}(\overline{\mathbb{B}^1}) \subset \mathscr{A}|_D:= \{A \cap D |A \in \mathscr{A}\}$ \\
($\mathscr A | _D$ ist $\sigma$-Algebra, siehe Übung).
\end{Bemerkung}

%Lemma I.3
\begin{Lemma}
$(X, \mathscr A)$ messbarer Raum, $D \in \mathscr A$, $f: D \rightarrow \overline {\mathbb R}$. Dann sind äquivalent:
\begin{enumerate}
	\item $f$ ist $\mathscr A$-messbar
	\item $\forall U \subset \mathbb R$ offen ist $f^{-1}(U) \in \mathscr A |_D$ und $f^{-1}(\{\infty\}), f^{-1}(\{-\infty\}) \in \mathscr A |_D$
	\item $\{f \leq s\} := \{x \in D: f(x) \leq x\} \in \mathscr A |_D$ $ \forall s \in \mathbb R$
	\item $\{f < s\} := \{x \in D: f(x) < x\} \in \mathscr A |_D$ $ \forall s \in \mathbb R$
	\item $\{f \geq s\} := \{x \in D: f(x) \geq x\} \in \mathscr A |_D$ $ \forall s \in \mathbb R$
	\item $\{f > s\} := \{x \in D: f(x) > x\} \in \mathscr A |_D$ $ \forall s \in \mathbb R$
\end{enumerate}
\end{Lemma}

%Lemma I.4
\begin{Lemma}
$(X, \mathscr A)$ messbarer Raum. $D \in \mathscr a$ und $f, g: D \rightarrow \overline{\mathbb R }$ $\mathscr A$-messbar \\
$ \left( f^{-1}(\overline{\mathscr{B}^1}) \subset \mathscr A |_D, g^{-1}(\overline{\mathscr{B}^1}) \subset \mathscr A |_D\right)$. \\ Dann sind die Mengen $\{f < g\}:= \{x \in D: f(x) < g(x)\}$ und $\{f \leq g\}$ Elemente aus $\mathscr A |_D$.
\end{Lemma}

%Satz I.3
\begin{Satz}
$(X, \mathscr A)$ messbarer Raum. $D \in \mathscr A$, $f_k: D \rightarrow \overline{\mathbb R}$ Folge von $\mathscr A$-messbaren numerischen Funktionen. dann sind $\inf\limits_k f_k$, $\sup\limits_k f_k$, $\liminf\limits_{k\rightarrow \infty} f_k$, $\limsup\limits_{k \rightarrow \infty} f_k$ auch $\mathscr A$-messbar. \\
(Hierbei ist $(\liminf f_k)(x) = \liminf f_k(x)$, ebenso $\limsup$)
\end{Satz}

%Satz I.4
\begin{Satz}
$(X, \mathscr A)$ messbarer Raum, $D \in \mathscr A$, $f, g: D \rightarrow \overline{\mathbb R }$ $\mathscr A$-messbar und $\alpha \in \mathbb R$. Dann sind $f+g, \alpha \cdotp f, f^\pm, \max{\{f, g\}}, \min{\{f, g\}}, \|f\|, f \cdotp g, \dfrac{f}{g}$ $\mathscr A$-messbar. \\
$\left( f^+ = \max{\{f, 0\}} \geq 0, f^- = \max{\{-f, 0\}} \geq 0, f = f^+ - f^-, \|f\| = f^+ + f^- \right)$
\end{Satz}

%Wieso x in M in A?? Was ist dann M?
%Bem
\begin{Bemerkung}
$(X, \mathscr A, \mu)$ Maßraum. Man sagt, die Aussage $A[x]$ ist wahr für \underline{$\mu$-fast alle $x \in M \in \mathscr A$}, falls eine $\mu$-Mullmenge $N$ existiert mit $\{x \in M: A[x]  \text{ falsch}\} \subset N$.\\
$f, g: X \rightarrow \overline{\mathbb R}$ Aussage  \quotes{$f(x) \leq g(x)$ für $\mu$-fast alle $x \in X$}  bedeutet: \\
$\exists N \subset \mathscr A$ mit $\mu(N) = 0$, sodass $\forall x \in X\backslash N: f(x) \leq g(x)$ \\
Eine Funktion $h$ ist \quotes{\underline{$\mu$-fast überall auf $X$}} definiert, wenn $h$ aus $D \in \mathscr A$ definiert ist und $\mu(X \backslash D) = 0$. \\
Eine Folge von Funktionen $f_k: D \rightarrow \overline{\mathbb R}$ konvergiert punktweise $\mu$-fast überall gegen $f: D \rightarrow \overline{\mathbb R}$, falls eine Menge $M \in \mathscr A$ existiert mit $\mu(N) = 0$ und $\lim\limits_ {k \rightarrow \infty} f_k(x) = f(x)$ $\forall x \in D \backslash N$
\end{Bemerkung}

%Definition I.9
\begin{Definition}
$(X, \mathscr A, \mu)$ Maßraum. Eine auf $D \in \mathscr A$ definierte numerische Funktion $f$ heißt \underline{$\mu$ messbar} (auf $X$), wenn $\mu(X \backslash D) = 0$ und $f$ $\mathscr A$-messbar ist.
\end{Definition}

%bem
\begin{Bemerkung}
Die Relation $f = g$ $\mu$-fast überall ist eine Äquivalenzrelation. Sei $D \in \mathscr A, f: D \rightarrow \overline{\mathbb R} \mu$-messbar. Dann gibt es eine $\mathscr A$-messbare Funkion $g: X \rightarrow \overline{\mathbb R}$ mit $f = g$ auf $D$. \\
\emph{Z.B.} \\
\begin{equation}
	g(x) =
	\begin{cases}
		f(x) & , x \in D \\
		0 & \text{, sonst}
	\end{cases}
\end{equation} \\
$\rightarrow$ Satz $1.3$ und $1.4$ gelten auch für $\mu$-messbare Funktionen (man muss zusätzlich fordern, dass $f, g, f \cdotp g, \dfrac{f}{g}$ $ \mu$-fast überall definiert sind)
\end{Bemerkung}


%Lemma I.5
\begin{Lemma}
$(X, \mathscr A, \mu)$ vollständiger Maßraum und $f$ $\mu$-messbar auf $X$. Dann ist auch jede Funktion $\tilde f$ mit $\tilde f = f$ $\mu-$fast überall auf $X$ $\mu$-messbar.
\end{Lemma}


%Satz I.5
\begin{Satz}
$(X, \mathscr A, \mu)$ vollständiger Maßraum, $f_k, k \in \mathbb N$ $\mu$-messbar. Falls $f_k$ punktweise $\mu$-fast überall gegen $f$ konvergiert, so ist $f$ $\mu$-messbar.
\end{Satz}

%Satz I.6
\begin{Satz}{\emph{Egorov:}} \\
$(X, \mathscr A, \mu)$ Maßraum. $D \in \mathscr A$ Menge mit $\mu(D) \less \infty$ und $f_k, f$ $ \mu|_D$-messbar, $\mu|_D$ - fast überall endliche Funktionen mit $f_k \rightarrow f$ $\mu|_D$- fast überall. \\
Dann existiert $\forall \epsilon \> 0$ eine Menge $B \subset D, B \in \mathscr A$ mit \\
\begin{enumerate}
	\item $\mu(D\backslash B) \less \epsilon$
	\item $f_n \rightarrow f$ gleichmäßig auf $B$.
\end{enumerate}
\end{Satz}

\section{Äußere Maße}

%Def I.10
\begin{Definition}
Sei $X$ eine Menge. Eine Funktion $\mu: P(X) \rightarrow [0, \infty]$ mit $\mu(\emptyset) = 0$ und $A \subset \bigcup\limits_{i=1}^\infty A_i \Rightarrow \mu(A) \leq \sum\limits_{i=1}^\infty \mu(A_i)$ heißt \underline{äußeres Maß} auf $X$.
\end{Definition}

%Bem
\begin{Bemerkung}
\begin{enumerate}
	\item Äußeres Maß ist immer auf $P(X)$ definiert.
	\item $\mu$ Maß auf $P(X) \Rightarrow$ $\mu$ \underline{äußeres Maß}.
	\item Begriffe $\sigma$-Additiv, $\sigma$-endlich, endlich, monoton, Nullmengen sind wie bei Maßen definiert.
	\item Jedes äußere Maß ist monoton. $A \subset B \Rightarrow \mu(A) \leq \mu(B)$
\end{enumerate}
\end{Bemerkung}

%Def I.11
\begin{Definition}
Sei $\mu$ äußeres Maß auf $X$. Die Menge $A \subset X$ heißt \underline{$\mu$-messbar}, falls für alle $S \subset X$: \\
$\mu(S) \geq \mu(S \cap A) + \mu(S \backslash A)$ \\
$\mathscr M(\mu):= \{ A \subset X | A$ $ \mu-messbar \}$
\end{Definition}

%Bem
\begin{Bemerkung}
Es gilt $(S \cap A) \cup (S \backslash A) = S$. Also $\mu(S) \leq \mu(S \cap A) + \mu(S \backslash A)$ \\
$A$ $\mu$-messbar $\Leftrightarrow$ $\mu(S) = \mu(S \cap A) + \mu(S \backslash A)$ $\forall S \subset X$
\end{Bemerkung}

%Satz I.7
\begin{Satz}
Sei $\mathscr Q$ ein System von Teilmengen einer Menge $X$, das die leere Menge enthält und sei $\lambda: \mathscr Q \rightarrow [0, \infty]$ mit $\lambda(\emptyset) = 0$.\\
Definiere $\mu: P(X) \rightarrow [0, \infty]$ für beliebige $E \subset X$ durch \\
 $\mu(E) := \inf \Set{ \sum\limits_{i=1}^\infty \lambda(P_i) | P_i \in \mathscr Q \forall i, E \subset \bigcup\limits_{i=1}^\infty P_i }$ \\
Dann ist $\mu$ äußeres Maß. $(\inf \emptyset = + \infty)$
\end{Satz}

%Satz I.8
\begin{Satz}
$\mu: P(X) \rightarrow [0, \infty]$ äußeres Maß auf $X$. Für $M \subset X$ erhält man durch $\mu|_M : P(X) \rightarrow [0, \infty], \mu|_M := \mu(A \cap M)$ ein \underline{äußeres Maß} (Einschränkung von $\mu$ auf $M$) auf $X$. Es gilt: $A$ $ \mu-$messbar $\Rightarrow$  $A$ $ \mu|_M $-messbar
\end{Satz}

%Satz I.9
\begin{Satz}
Sei $\mu$ äußeres Maß auf $X$. Dann gilt:
\begin{enumerate}
	\item $N$ $\mu$-Nullmenge $\Rightarrow N$ $\mu$-messbar
	\item $N_i, i \in \mathbb N,$ $\mu$-messbar $\Rightarrow \bigcup\limits_{x=1}^\infty N_i$ Nullmengen
\end{enumerate}
\end{Satz}


%Lemma I.6
\begin{Lemma}
$\mu$ äußeres Maß auf $X$, $A_i \in \mathscr M(\mu), i = 1, ... k$ seien paarweise disjunkt. Dann gilt $\forall S \subset X$: \\
$\mu\left(S \cap \bigcup\limits_{i=1}^\infty A_i\right) = \sum\limits_{i=1}^\infty \mu(S \cap A_i)$
\end{Lemma}

%Satz I.10
\begin{Satz}
Sei $\mu: P(X) \rightarrow [0, \infty]$ äußeres Maß. Dann ist $\mathscr M(\mu)$ eine $\sigma$-Algebra und $\mu$ ist ein vollständiges Maß auf $\mathscr M(\mu)$.
\end{Satz}

%Lemma I.7
\begin{Lemma}
Sei $\mu$ äußeres Maß und seien $A_i \in \mathscr M(\mu), i \in \mathbb N$. Dann gelten:
\begin{enumerate}
	\item aus $A_1 \subset A_2 \subset ...$ folgt $\mu\left(\bigcup\limits_{i=1}^\infty A_i\right) = \lim_{i \rightarrow \infty} \mu(A_i)$
	\item aus $A_1 \supset A_2 \supset ...$ folgt $\mu\left(\bigcap\limits_{i=1}^\infty A_i\right) = \lim_{i \rightarrow \infty} \mu(A_i)$, falls $\mu(A_1) \less \infty$
\end{enumerate}
\end{Lemma}


%hö, das sollte aber I.5 sein. Was ist wo I.4??
\section{Fortsetzungssatz von Carathéodory}

%Def I.12
\begin{Definition}
Ein Mengensystem $\mathscr A \in P(X)$ heißt \underline{$\cap$-stabil}, \underline{ $\cup$-stabil }, \underline{$\backslash$-stabil}, falls mit $A, B \in \mathscr A$ auch $A \cap B \in \mathscr A, A \cup B \in \mathscr A, A\backslash B \in \mathscr A$.
\end{Definition}

%Bem.
\begin{Bemerkung}
$\cap$-stabil $(\cup$-stabil$)$ $\Rightarrow$ $\cap$-Stabilität $($bzw. $\cup)$ von endlichen Durchschnitten.
\end{Bemerkung}

%Def I.13.
\begin{Definition}
Ein Mengensystem $\mathscr R \in P(X)$ heißt \underline{Ring} über $X$, falls
\begin{enumerate}
	\item $\emptyset \in \mathscr R$
	\item $A, B \in \mathscr R \Rightarrow A\backslash B \in \mathscr R$
	\item $A, B \in \mathscr R \Rightarrow A \cup B \in \mathscr R$
\end{enumerate}
Ist $X \in \mathscr R$, so heißt $\mathscr R$ \underline{Algebra}.
\end{Definition}


%Bsp.
\begin{Beispiel}
\begin{enumerate}
	\item Für $A \subset X$ ist $\{\emptyset, A\}$ ein Ring. Für $A \ne X$ ist $\{\emptyset, A\}$ keine Algebra. $P(X)$ ist Algebra
	\item $\{$endliche Teilmengen von $X\}$ Ring über $X$, ebenso $\{ $abzählbare Teilmengen von $X\}$
\end{enumerate}
\end{Beispiel}

%Bem.
\begin{Bemerkung}
\begin{enumerate}
	\item $A, B \in \mathscr R$ (Ring) $\Rightarrow A \cap B = A \backslash (A \backslash B) \in \mathscr R$
	 \item Satz 1.1 ist auch für Ringe bzw. Algebren richtig; Satz 1.1 $\rightarrow$ erzeugte $\sigma$-Algebra. Also können wir auch erzeugte Ringe und erzeugte Algebren definieren.
\end{enumerate}
\end{Bemerkung}


%Def I.14
\begin{Definition}
Sei $\mathscr R \subset P(X)$ ein Ring. Eine Funktion $\lambda: \mathscr R \rightarrow [0, \infty]$ heißt \underline{Prämaß} auf $\mathscr R$, falls
\begin{enumerate}
	\item $\lambda(\emptyset) = 0$
	\item für paarweise disjunkte $A_i \in \mathscr R, i \in \mathbb N$ mit $\bigcup\limits_{i=1}^\infty A_i \in \mathscr R$: \\
	$\lambda \left(\bigcup\limits_{i=1}^\infty A_i\right) = \sum\limits_{i=1}^\infty \lambda(A_i)$
\end{enumerate}
\end{Definition}

%Bem.
\begin{Bemerkung}
Die Begriffe $\sigma$-subadditiv, $\sigma$-endlich, endlich, monoton, Nullmenge werden wie bei Maßen definiert
\end{Bemerkung}

%Bsp.
\begin{Beispiel}
\begin{enumerate}
	\item $\mathscr R$ Ring über $X$,
	\begin{equation}
	\lambda(A) =
		\begin{cases}
			0 & A = \emptyset \\
			\infty & \text{sonst}
		\end{cases}
	\end{equation}
	\item $\mathscr R = \{$endliche Teilmengen vom $X\}$, $\lambda = $ $card|_{\mathscr R}$ ist Prämaß
	\item alle äußeren Maße sind Prämaße, wenn man sie einschränkt auf $\mathscr R = \mathscr M(\mu)$
\end{enumerate}
\end{Beispiel}

%Def I.15
\begin{Definition}
Sei $\lambda$ Prämaß auf Ring $\mathscr R \subset P(X)$. Ein äußeres Maß $\mu$ auf $X$ (bzw. ein Maß $\mu$ auf $\mathscr A$) heißt \underline{Fortsetzung} von $\lambda$, falls gilt:
\begin{enumerate}
	\item $\mu|_\mathscr R = \lambda$, d.h. $\mu(A) = \lambda(A)$ $\forall A \in \mathscr R$
	\item $\mathscr R \in \mathscr M(\mu) ($bzw. $\mathscr R \subset \mathscr A)$
\end{enumerate}
\end{Definition}


%Satz I.11, VL 06.11.15
\begin{Satz}
$\lambda: R \rightarrow [0, \infty], {}R \subset P(X)$ (Prämaß auf Ring), $\forall {}E \subset X:$ \\
\begin{center}
$\mu(E):= inf\Set{ \sum\limits_{i=1}^{\infty} \lambda(A_i) | E \subset \bigcup\limits_{i=1}^{\infty} A_i, A_i \in \mathbb{R}}$
\end{center}
ist Fortsetzung von $\lambda$. $(\mu$ ist das durch $\lambda$ induzierte äußere Maß$)$
\end{Satz}

%Lemma I.6
\begin{Lemma}
Sei $\mu$ \cara{} des Prämaß $\lambda$ auf $R$. Sei $\tilde{\mu}$ ein Maß auf $\sigma(R)$ mit $\tilde{\mu} = \lambda$ auf $R$. Dann gilt $\tilde{\mu}(R) \leq \mu(E)$ $\forall E \in \sigma(R)$.
\end{Lemma}

%Satz I.12
\begin{Satz} \emph{Hopf-Fortsetzung} \\
Sei $\lambda: R \rightarrow [0, \infty]$ Prämaß auf Ring $R \subset P(X)$. Dann existiert ein Maß $\mu$ auf $\sigma(R)$ mit $\mu =\lambda$ auf $R$. Diese Fortsetzung ist \underline{eindeutig}, falls $\lambda$ $\sigma$-endlich ist.
\end{Satz}

\emph{Notation:}
Sei $\mu$ Maß auf $X$ mit der Eigenschaft: $\forall D \subset X: \exists E \in \mathscr{M}(\mu)$ mit $D \subset E$ und $\mu(D) = \mu(E)$. Dann heißt $\mu$ \underline{reguläres} äußeres Maß.

\begin{Beispiel}
Blödes Beispiel
\end{Beispiel}

%Satz I.13
\begin{Satz}
Sei $\mu$ \cara{} des Prämaßes $\lambda$ auf $R$. Dann existiert zu jedem $D \subset X$ ein $E \in \sigma(R) \subset \mathscr M(\mu)$ mit $E \supset D$ und $\mu(D) = \mu(E)$
\end{Satz}

%Satz I.14
\begin{Satz}
$\lambda$  $\sigma$-endliches Prämaß auf Ring $\mathscr R$ und $\mu$ sei \cara{} von $\lambda$. Dann ist $\mu|_{\mathscr M(\mu)}$ die \underline{Vervollständigung} von $\mu|_{\sigma(\mathscr R)}$. Insbesondere gibt es genau eine Fortsetzung von $\lambda$ zu einem Maß auf $\mathscr M(\mu)$.
\end{Satz}

%Lemma I.9
\begin{Lemma}
Sei $\lambda: \mathscr R \rightarrow [0, \infty]$ Prämaß auf Ring $\mathscr R$ und $\lambda$ sei $\sigma$-endlich. $\mu$ sei die \cara{} von $\lambda$. $D \subset X$ ist genau dann $\mu$-messbar, falls eine der beiden folgenden Bedingungen gilt:
\begin{enumerate}
	\item $\exists E \in \sigma(\mathscr R)$ mit $E \supset D$ und $\mu(E \backslash D) = 0$
	\item $\exists C \in \sigma(\mathscr R)$ mit $C \subset D$ und $\mu(D \backslash C) = 0$
\end{enumerate}
\end{Lemma}

%Michelle 09.11.15
Prämaß auf Ring kann zu einem \underline{Maß} fortgesetzt werden: \\
\underline{Ziel}: Volumen (Flächeninhalt) einer Menge $\Omega \subset \mathbb R^n$ zu berechnen ($\rightarrow$ Maß einer Menge)


%I.5 Halbringe und Inhalte
\section{Halbringe und Inhalte}
%Def. I.16
\begin{Definition}
Ein Mengensystem $\mathscr Q \subset P(X)$ heißt \underline{Halbring} über $X$, falls
\begin{enumerate}
	\item $\emptyset \in \mathscr Q$
	\item $P, Q \in \mathscr Q \Rightarrow P \cap Q \in \mathscr Q$
	\item $P, Q \in \mathscr Q \Rightarrow P \backslash Q = \bigcup\limits_{i=1}^k P_i$ mit $P_i \in \mathscr Q$ paarweise disjunkt.
\end{enumerate}
\end{Definition}

%Bsp.
\begin{Beispiel}
\begin{enumerate}
	\item $X$ beliebige Menge. $Q := \{\emptyset\} \cup \{\{a\}:  a \in X\}$ ist Halbring
	\item $I \subset \mathbb R$ heißt \underline{Intervall}, falls $a, b \in \mathbb R$ existieren mit $a \leq b$ und $(a, b) \subset \overline{I} \subset [a, b]$. Quader in $\mathbb R^n$ $ Q:= I_1 \times ... \times I_n$,  $I_1, ... I_n$ Intervalle.
\end{enumerate}
\end{Beispiel}

%Satz I.15
\begin{Satz}
$Q^n := \{Q \in \mathbb R^n: Q $ Quader$\}$ ist Halbring über $\mathbb R^n$
\end{Satz}

%Satz I.16
\begin{Satz}
$\mathscr Q$ Halbring über $X$ und $\mathscr F := \Set{F = \bigcup\limits_{i=1}^k P_i, k \in \mathbb N, P_i \in \mathscr Q \quad \forall 1 \leq i \leq k}$. Dann ist $\mathscr F$ der von $\mathscr Q$ erzeugte Ring.
\end{Satz}

%Bsp.
\begin{Beispiel}
\begin{enumerate}
	\item $\mathscr Q = \{\emptyset \} \cup \{\{a\}: a \in X\} \rightarrow \mathscr F = \{$ endliche Teilmengen von $X\}$
	\item $\mathscr Q^n \rightarrow$ E erzeugter Ring $\mathscr F^n:= \{$endliche Vereinigung von Quadern$\}$: Figuren
\end{enumerate}
\end{Beispiel}

%Lemma I.10
\begin{Lemma}
Sei $\mathscr Q$ Halbring über $X$ und $\mathscr F$ der von $\mathscr Q$ erzeugte Ring. Dann gilt: $\sigma(\mathscr Q) = \sigma(\mathscr F)$, $(\sigma(.)$: erzeugte $\sigma$-Algebra$)$
\end{Lemma}

%Lemma I.11
\begin{Lemma}
Sei $\mathscr Q$ Halbring über $X$ und $\mathscr F$ der von $\mathscr Q$ erzeugte Ring. Zu jedem $F \in \mathscr F$ existieren $P_1, ... P_k \in \mathscr Q$ \underline{paarweise disjunkt} mit $F = \bigcup\limits_{i=1}^k P_i$
\end{Lemma}

%Def I.17
\begin{Definition}
Sei $\mathscr Q \subset P(X)$ Halbring über $X$. Eine Funktione $\lambda: \mathscr Q \rightarrow [0, \infty ]$ heißt \underline{Inhalt} auf $\mathscr Q$, falls
\begin{enumerate}
	\item $\lambda(\emptyset) = 0$
	\item $\lambda\left(\bigcup\limits_{i=1}^n	A_i\right) = \sum\limits_{i=1}^n \lambda(A_i)$ für paarweise disjunkte $A_i \in \mathscr Q$ mit $\bigcup\limits_{i=1}^n A_i \in \mathscr Q$
\end{enumerate}
Ein Inhalt heißt \underline{Prämaß} auf $\mathscr Q$, falls $\lambda$ $\sigma$-additiv auf $\mathscr Q$ ist, d.h. für paarweise disjunkte $A_i, i \in \mathbb N$ gilt: $\lambda\left(\bigcup\limits_{i\in \mathbb N} A_i \right) = \sum\limits_{i=1}^\infty \lambda(A_i)$, falls $\bigcup\limits_{i \in \mathbb N} A_i \in \mathscr Q$
\end{Definition}

%Bem.
\begin{Bemerkung}
\begin{enumerate}
	\item Ist $\mathscr Q$ Ring, so stimmt die Definition von Prämaß auf \emph{Def. 1.14} überein
	\item Die Begriffe $\sigma$-subadditiv, $\sigma$-endlich, monoton, Nullmenge übertragen sich direkt auf Inhalte.
\end{enumerate}
\end{Bemerkung}

%Satz I.17
\begin{Satz}
Sei $\lambda$ Inhalt auf Halbring $\mathscr Q$ über $X$ und sei $\mathscr F$ der von $\mathscr Q$ erzeugte Ring. Dann gibt es genau einen Inhalt $\overline{\lambda}: \mathscr F \rightarrow [0, \infty]$ mit $\overline{\lambda}(Q) = \lambda(Q)$ $\forall Q \in \mathscr Q$
\end{Satz}

%Fehlt hier noch was?? Die Vorlesung nach dem 09.11. war über was:
%13.11.15

%Lemma I.12
\begin{Lemma}
Ein Inhalt $\lambda$ auf dem Halbring $\mathscr Q$ über $X$ ist monoton und subadditiv.
\end{Lemma}

Auf $\mathscr Q^n = \{Q = I_1 \times ... \times I_n: I_i \in J, 1 \leq i \leq n, $ Quader in $\mathbb R^n \}$ .
Halbring, falls $Q = I_1 \times ... \times I_n$ und $(a_i, b_i) \subset I_i \subset [a_i, b_i]$:
\begin{center}
	$vol^n(Q) := \prod\limits_{i=1}^n (b_i - a_i)$
\end{center}

%Satz I.18
\begin{Satz}
$vol^n(\cdotp)$ ist ein Inhalt auf $\mathscr Q^n$
\end{Satz}

%Satz I.19
\begin{Satz}
$\lambda: \mathscr Q \rightarrow [0, \infty]$ Prämaß auf Halbring $\mathscr Q \subset P(X)$, $F$ sei der von $\mathscr Q$ erzeugte Ring und $\overline{\lambda}: F \rightarrow [0, \infty]$ der eindeutig bestimmte Inhalt auf $F$ mit $\overline{\lambda}|_\mathscr Q = \lambda \; ($Satz I.17$)$
Dann ist $\overline{\lambda}$ ein Prämaß auf $F$.
\end{Satz}



%Was tust du da? Das ist sehr komisch. Bzw. ich weiß nicht, wo das hässliche F herkommt und was noch zur Bemerkung gehört, bzw. wofür die Gänsefüßchen stehen.
%Bem
\begin{Bemerkung}
$\lambda: \mathscr Q \rightarrow [0, \infty]$ Prämaß auf Halbring $\mathscr Q \subset P(X)$ und $\overline{\lambda} : F \rightarrow [0, \infty]$ Prämaß auf dem von $\mathscr Q$ erzeugten Ring $F$ mit $\overline{\lambda}|_\mathscr Q = \lambda$ \\

$\mathscr  F =$ Menge aller Figuren. Oder so. \\

Satz I.17: $\overline{\lambda}(F) = \sum\limits_{i=1}^n \lambda(Q_i)$ mit $F \in \mathscr F$ und $F = \bigcup\limits_{i=1}^n Q_i \in \mathscr Q$

\end{Bemerkung}

Satz I.17 $\rightarrow$ für zwei äußere Maße $\mu, \overline{\mu}$ gilt: $\mu = \overline{\mu}$


%Satz I.20
\begin{Satz}
Sei $\lambda: \mathscr Q \rightarrow [0, \infty]$ Prämaß auf Halbring $\mathscr Q$ über $X$. Sei $\mu: P(X) \rightarrow [0, \infty]$ das aus $\lambda$ konstruierte äußere Maß $($Satz I.7$)$ \\
D.h.:
\begin{center}
	$\mu(E) := \inf\Set{\sum\limits_{i=1}^\infty \lambda(A_i) \, |\, A_i \inf\mathscr Q, \, E \subset \bigcup\limits_{i=1}^\infty A_i }$
\end{center}
Dann ist $\mu$ eine Fortsetzung von $\lambda$.
\end{Satz}


%Satz I.21
\begin{Satz}
Für einen Inhalt $\lambda$ auf einem Ring $R$ auf $A_i \in P,\, i \in \mathbb N$ betrachte die Aussagen:
\begin{enumerate}
	\item $\lambda$ ist Prämaß auf $R$
	\item $A_1 \subset A_2 \subset ... \subset A_i \subset A_{i+1} \subset ...$
	\begin{center}
		$\Rightarrow \lambda\left( \bigcup_{i \in \mathbb N} a_i \right) = \lim_{i \rightarrow \infty} \lambda(A_i)$
	\end{center}
	\item $A_1 \supset ... \supset A_i \supset A_{i+1} \supset ...$ mit $\lambda(A_1) < \infty$
	\begin{center}
		$\Rightarrow \lambda\left(\bigcap_{i \in \mathbb B} A_i \right) = \lim_{i\rightarrow \infty} \lambda(A_i)$
	\end{center}
	\item $A_1 \supset ... \supset A_i \supset A_{i+1} \supset ...$ mit $\lambda(A_1) <  \infty$ und $\bigcap\limits_{i=1}^\infty A_i = \emptyset$
	\begin{center}
		$\Rightarrow \lim \lambda(A_i) = 0$
	\end{center}
\end{enumerate}
Dann gilt: $1. \Leftrightarrow 2. \Rightarrow 3. \Rightarrow 4.$ \\
Ist $\lambda$ endlich, d.h. $\lambda(A) < \infty \, \forall A \in R$, dann sind $1. - 4.$ äquivalent.
\end{Satz}


\newpage
%Kapitel 2

\chapter{Das Lebesgue -Maß und -Integral}
\section{Das Lebesgue-Maß}

Halbring $\mathscr{Q}^n$ Inhalt.\\
$vol^n : \mathscr{Q}^n \rightarrow [0, \infty]$


	Prämaß auf Halbring\\
	$\Rightarrow$ \cara{} ist äußeres Maß auf $P(X)$.

\begin{Lemma}
	$vol^n : \mathscr{Q}^n \rightarrow [0, \infty]$ ist ein Prämaß.
\end{Lemma}

\begin{Definition}
	Das \underline{(äußere) Lebesguemaß} einer Menge $E \subset \mathbb{R}^n$ ist definiert durch\\
	$ \mathscr{L}^n(E) := inf \Set{ \sum_{i = 1}^{\infty} vol^n(P_i) | P_i \in \mathscr{Q}^n, E \subset \bigcup\limits_{i = 1}^\infty P_i }$
\end{Definition}

\begin{Satz}
	Für $\mathscr{L}^n$ auf $\mathbb{R}^n$ gilt:
	\begin{enumerate}
		\item $\mathscr{B}^n \subset \mathscr{M}(\mathscr{L}^n)$ $($Alle Borelmengen sind $\mathscr{L}^n$-messbar.$)$
		\item Zu $E \subset \mathbb{R}^n$ existiert $B \in \mathscr{B}$ mit $E \subset B$
			und $\mathscr{L}^n(B) = \mathscr{L}^n(E)$
		\item $\mathscr{L}^n(K) < \infty$ \, $\forall K \subset \mathbb{R}^n kompakt.$
	\end{enumerate}
\end{Satz}

\begin{Lemma}
	Für $E \subset \mathbb{R}^n$ gilt:
	\begin{enumerate}
		\item $\mathscr{L}^n(E) = inf\Set{\mathscr{L}^n(U) | \text{U offen mit E} \subset{U}}$
		\item $\mathscr{L}^n(E) = sup\Set{\mathscr{L}^n(K) | \text{K kompakt mit K} \subset{E}}$,
			falls $E \in \mathscr{M}(\mathscr{L}^n)$
	\end{enumerate}
\end{Lemma}

\begin{Satz}
	$D \subset \mathbb{R}^n$ ist genau dann $\mathscr{L}^n$-messbar, wenn eine der folgenden Bedingungen gilt:
	\begin{enumerate}
		\item $\exists B \in \mathscr{B}^n$ mit $B \ni D$ und $\mathscr{L}^n(B \backslash D) = 0$
		\item $\exists C \in \mathscr{B}^n$ mit $C \subset D$ und $\mathscr{L}^n(D \backslash C) = 0$
	\end{enumerate}
	\textit{Zusatz:}\\
	Es kann $B = \bigcap\limits_{i = 1}^\infty U_i$ mit $U_i$ offen, $C = \bigcup\limits_{j = 1}^\infty A_j$ mit $A_j$
	abgeschlossen gewählt werden.
\end{Satz}

%20.11.15
%Satz II.3: Luisin
\begin{Satz}
\emph{Luisin}: \\
Sei $A \subset \mathbb R^n$ offen mit $\mathscr L^n(A) < + \infty$. $f$ sei $\mathscr L^n|_A$-messbar mit Werten in $\mathbb R$. \\
Für alle $\epsilon > 0$ existiert $\kappa = \kappa_\epsilon \subset A$ kompakt mit
\begin{enumerate}
	\item $\mathscr L^n(A \backslash \kappa) < \epsilon$
	\item $f|_\kappa$ ist stetig.
\end{enumerate}
\end{Satz}
%Krasser Scheiß, ist der Beweis lang.



%Def II.2
\begin{Definition}
Ein äußeres Maß $\mu$ auf $\mathbb R^n$ heißt \underline{Borelmaß}, falls gilt:
\begin{enumerate}
	\item $B \in \mathscr B^n \Rightarrow B \, \mu$-messbar $(B \in \mathscr M(\mu))$
	\item $\mu(\kappa) < \infty \, \forall \kappa \subset \mathbb R^n$ kompakt
\end{enumerate}
\end{Definition}

%Bsp.
\begin{Beispiel}
$\mathscr L^n$ $($Satz II.1$) \varrightwavearrow \, \mathscr L^n|_E$, $E \subset \mathbb R^n$ beliebig
\end{Beispiel}

\emph{Notation}: \\
Ein Maß $\mu$ auf $\mathbb R^n$ heißt \underline{translationsinvariant}, wenn mit $E + a = \{x +a | x \in E\}$ gilt:
\begin{center}
	$\mu(E+a) = \mu(E) \, \forall E \subset \mathbb R^n, a \in \mathbb R^n$
\end{center}

%Bsp.
\begin{Beispiel}
$\mathscr L^n$, denn $vol^n$ ist translationsinvariant und der Rest folgt aus der Definition von $\mathscr L^n$. \\
\begin{center}
	$\mathscr L^n(E) = \inf\Set{ \sum vol^n(P_i) | P_i \in \mathscr Q, \,\bigcup\limits_{i=1}^\infty P_i \supset E }$
\end{center}
\end{Beispiel}

%noch ein Verweis auf das Hausdorffmaß? VL 20.11.15, Seite 5. Auch auf der Seite davor schon

%Lemma II.3
\begin{Lemma}
Sei $\mu$ tranlationsinvariantes Borelmaß auf $\mathbb R^n$
\begin{center}
	$\Rightarrow H = \Set{ x \in \mathbb R^n | x_i = c \text{ für ein } 1\leq i \leq n}$, $c \in \mathbb R$, ist $\mu$-Nullmenge.
\end{center}
\end{Lemma}

%Ist der Index hier wirklich ein c? VL 20.11.15, immer noch Seite 5
%Bsp.
\begin{Beispiel}
$\mathscr L^2(\{x_c = 0\}) = 0$
\end{Beispiel}

%Satz II.4
\begin{Satz}
Sei $\mu$ tranlationsinvariantes Borelmaß auf $\mathbb R^n$. Dann gilt:
\begin{center}
	$\mu(E) = \theta \mathscr L^n(E)$ $ \forall E \subset \mathbb R^n$
\end{center}
ist $\mathscr L^n$-messbar, $\theta = \mu([0, 1]^n)$
\end{Satz}

%Bsp.
%ui, sind hier viele Beispiele
\begin{Beispiel}
$\mathscr H^n = \mathscr L^n$ $(($sehr$)$ bisschen schwierig$)$
\end{Beispiel}

%Lemma II.4
\begin{Lemma}
Sei $U \subset \mathbb R^n$ offen unf $f: U \rightarrow \mathbb R^n$ Lipschitz-stetig mit Konstante $L$ bzgl $\|x\|_\infty := \max_{1 \leq i \leq n} |x_i |$. Dann gilt:
\begin{center}
	$\mathscr L^n(f(E)) \leq L^n \mathscr L^n (E)$ $\forall E \subset U$
\end{center}
\end{Lemma}

%Bsp. schon wieder eins
\begin{Beispiel}
$\mathscr L^n(E) = 0 \Rightarrow \mathscr L^n(f(E)) = 0$
\end{Beispiel}
%Oh, der 20.11.15 ist schon fertig und ich bin noch gar nicht in Freiburg :(


%Ich brauche Beschäftigung: VL 23.11.15: ab Satz II.5
%31.12.15 spät abends. Noch ist der folgende Satz II.5 im pdf als II.2 gezählt

%Satz II.5
\begin{Satz}
Sei $U \subseteq \mathbb R^n$ offen und $f: U \rightarrow \mathbb R^n$ lokal Lipschitz $($z.B. $f \in \mathscr C^1(U, \mathbb R^n))$, so gilt:
\begin{enumerate}
	\item $N \subset U$ $\mathscr L^n$-Nullmenge $\Rightarrow$ $f(N)$ $\mathscr L^n$-Nullmenge
	\item $E \subset U$ $\mathscr L^n$-messbar $\Rightarrow$ $f(E)$ $\mathscr L^n$-messbar.
\end{enumerate}
\end{Satz}


%Satz II.6
\begin{Satz}
Für $S \in O(n)$ und $a \in \mathbb R^n$ gilt:
$\mathscr L^n (S(E) +a) = \mathscr L^n(E)$ $\forall E \subset \mathbb R^n$
\end{Satz}


%Lemma II.5
\begin{Lemma}
$\forall S \in GL(n)$ $\exists$ Diagonalmatrix $\Lambda$ mit Einträgen $\lambda_i > 0$ und $T_1, T_2 \in O(n)$, sodass $S = T_1\cdotp\Lambda\cdotp T_2$ $($Polarzerlegung$)$
\end{Lemma}


%Satz II.7
\begin{Satz}
\emph{Lineare Transformationsformel} \\
Für eine lineare Abbildung $S: \mathbb R^n \rightarrow \mathbb R^n$ gilt: $\mathscr L^n(S(E)) = |\det S| \mathscr L^n(E)$ $\forall E \subset \mathbb R^n$
\end{Satz}


%YES! Erste Matrix! Sogar mit Punkten!! :D
%Bsp Ellipsoid
\begin{Beispiel}
\emph{Ellipsoid} \\
$\lambda_1, ..., \lambda_n = 0$, $E = \Set{ x \in \mathbb R^n | \left( \dfrac{x_1}{\lambda_1} \right)^2 + ... + \left (\dfrac{x_n}{\lambda_n}\right)^2 < 1 }$ \\
Jetzt gilt: $E = \Lambda(\mathscr B_1(0))$, $\Lambda =$
$\begin{pmatrix}
      \lambda_1 & 	 &	&	& 0 \\
      		 &  	\ddots \\
		 &	& \ddots \\
		 &	&	& \ddots \\
		 0 &	&	&	&	\lambda_n
\end{pmatrix}$
$\Rightarrow \mathscr L^n(E) = (\lambda_1 \cdotp ... \cdotp \lambda_n)\mathscr L^n(\mathscr B_1(0))$
\end{Beispiel}


%Bsp Vitali
\begin{Beispiel}
\emph{Vitali 1905} \\
$\exists S \subset [0, 1]$, die nicht $\mathscr L^1$-messbar ist. Auf $[0, 1]$ gibt es die Äquivalenzrelation
\begin{center}
	$x \sim y \Leftrightarrow x - y \in \mathbb Q$
\end{center}
Auswahlaxiom: $\exists$ Repräsentantensystem $S \subset [0, 1]$ für $\sim$, d.h. $\forall y \in [0, 1]$ $\exists! x \in S: x \sim y$ \\
Sei jetzt $q_1, q_2, ...$ eine Abzählung von $\mathbb Q \cap [-1, 1]$. Dann gilt: $(q_j + S) \cap (q_k +S)  \emptyset $ $\forall j \ne k$ \\
$($Sonst existieren $x_1, x_2 \in S$ mit $q_j +x_1 = q_k + x_2 \Rightarrow x_1-x_2 = q_k -q_j \in \mathbb Q \Rightarrow x_1 \sim x_2  \Rightarrow x_1 = x_2 \Rightarrow q_j = q_k $\blitzd $ )$ \\

\emph{Behauptung:} $[0, 1]$ $\underbrace{\subset}_{1)}$ $\bigcup\limits_{k=1}^\infty
 (q_k + S) \subset [-1, 2]$ \\
 Zu $1)$: zu $y \in (0, 1)$ existiert $x \in S: y - x = q \in \mathbb Q \cap [-1, 1] \Rightarrow$ $\exists k: y \in q_k + S$ \\
\emph{Annahme:} $S$ ist $\mathscr L^1$-messbar mit $\mathscr L^1(S) = \mathscr L^1(q + S)$ $\forall q \in \mathbb Q$ folgt: $\sum\limits_{k=1}^\infty \mathscr L^1(S) = \sum\limits_{k=1}^\infty \mathscr L^1(q_k + S) \underbrace{=}_ {\text{Satz I.10}} \mathscr L^1\left(\bigcup\limits_{k=1}^\infty (q_k + S) \right) \in (1, 3]$ \blitzd
\end{Beispiel}


\section{Lebesgue-Integral}
%Def II.3
\begin{Definition}
$(X, \mathscr A)$ messbarer Raum. Eine Funktion $\phi: X \rightarrow \mathbb R$ heißt \underline{$\mathscr A$-Treppenfunktion} $(\mathscr A$-TF$)$, falls sie $\mathscr A$-messbar ist und nur endlich vielen FUnktonswerte annimmt. Nach Satz I.4 ist die Menge $\tau_ \mathscr A$ der $\mathscr A$-TF ein $\mathbb R$-Vektorraum. \\
Nun setzte $\tau_\mathscr A^+ := \left\{\phi\in \tau_\mathscr A: \phi \geq\right\}$
\end{Definition}


%Bsp
\begin{Beispiel}
$E \subset X$ \\
\begin{equation}
	\chi_E (x) :=
	\begin{cases}
		1 &, x \in E \\
		0 & , x \notin E
	\end{cases}
	\in \tau_\mathscr A^+ \Leftrightarrow E \in \mathscr A
\end{equation}
\end{Beispiel}

%Satz II.8
\begin{Satz}
Zu jeder $\mathscr A$-messbaren Funktion $f: [0, \infty]$ exisitert eine Folge $f_k \in \tau_\mathscr A^+$ mit $f_0 \leq f_1 \geq ...$ und
\begin{center}
	$\lim_{k \rightarrow \infty} f_k(x) = f(x) \, \forall x\in X$
\end{center}
\end{Satz}
%27.11.15

%der erste Absatz ist nicht wirklich beschriftet, sondern einfach so da, ich tex ihn mal einfach ab, habe das geschwungene tau aber noch nie in Ana gesehen^^


%Absatz


\emph{komische Vorbemerkung?}
$(X, \mathscr A)$ messbarer Raum \\
$\rightarrow \mathscr A-TF$ \\
$\rightarrow \tau_\mathscr A ^+ = \left\{\phi \in \tau_\mathscr A | \phi \geq 0\right\}$ \\
Jetzt $(X, \mathscr A, \mu)$ Maßraum. Neue Notation: \underline{$\tau^+ (\mu)$} statt $\tau_\mathscr A ^+$ \\
Definiere für $\phi \in \tau+(\mu)$ mit Wertemenge $\{s_1, ... , s_l\} \subset [0, \infty)$ das \emph{Integral}: \\
$I(\phi) = \sum\limits_{i=1}^l s_i \mu(\{\phi = s_i\}) \in [0, \infty]$


%Lemma II.6
\begin{Lemma}
Für $\phi, \psi \in \tau	^+(\mu)$ und $\alpha, \beta \in [0, \infty)$ gelten:
\begin{enumerate}
	\item $I(\alpha\phi + \beta\psi) = \alpha I(\phi) + \beta I(\psi)$
	\item $\phi \leq \psi \Rightarrow I(\phi) \leq I(\psi)$
\end{enumerate}
\end{Lemma}


%Def II.4
\begin{Definition}
Sei $(X, \mathscr A, \mu)$ ein Maßraum und $f: X \rightarrow \overline{\mathbb R} \mathscr A$-messbar. Ist $f: X \rightarrow [0, \infty]$, so setzen wir:
$\int f d\mu := \sup\Set{I(\phi) | \phi \in \tau^+(\mu), \phi \leq f}$ $(\phi$ heißt \underline{Unterfunktion} in dieser Definition$)$ \\

Ist $f: X \rightarrow \overline{\mathbb R}$ und sind die Integrale von $f^+ = \max{f, 0}$ und $f^- = \max{-f, 0}$ wie eben definiert und nicht beide unendlich, so setzen wir: \\
$\int f d\mu := \int f^+ d\mu - \int f^- d\mu \in [-\infty, \infty]$ \\
$(f = f^+ - f^-)$ \\
$($Das \underline{Lebesgue-Integral}!$)$
\end{Definition}

%Lemma II.7
\begin{Lemma}
Für $f \in \tau^+ (\mu)$ gilt: $\int f d\mu = I(f)$
\end{Lemma}

%Bsp.
\begin{Beispiel}
$\mathscr L^1$ auf $\mathbb R$; $\mathscr A = \mathscr M(\mathscr L^1), \chi_\mathbb Q \in \tau^+ (\mathscr L^1)$ \\
$\mathbb Q \in \mathscr B^1$ $($Blatt 2$)$ \\
$\int \chi_\mathscr Q d\mathscr L^1 = 0 \cdotp \mathscr L^1(\mathbb R \backslash \mathbb Q) + 1 \cdotp \mathscr L^1(\mathbb Q) = 0$
\end{Beispiel}

%Def II.5
\begin{Definition}
Eine Funktion $f: X \rightarrow \overline{\mathbb R}; ((X, \mathscr A, \mu)$ Maßraum$)$ heißt \underline{integrierbar bzgl. $\mu$}, wenn sie \underline{$\mathscr A$-messbar} ist und wenn gilt: \\
$\int f d\mu \in \mathbb R$ $\Leftrightarrow$ $\int f^+ d\mu + \int f^- d\mu < \infty$
\end{Definition}

%Satz II.9
\begin{Satz}
Sei $(X, \mathscr A, \mu)$ Maßraum und $f, g: X \rightarrow \overline{\mathbb R}$ seien $\mathscr A$-messbar. Ist $f \leq g$ $\mu$-fast unendlich und $\int f d\mu > -\infty$, so existiert auch $\int g d\mu$ und es gilt: \\
$\int f d\mu \leq \int g d\mu$ \\
Analog:  Ist $f \geq g$ $\mu$-fast unendlich und $\int f d\mu < \infty$, so existiert auch $\int g d\mu$ und es gilt: \\
$\int f d\mu \geq \int g d\mu$
\end{Satz}

%Bem.
\begin{Bemerkung}
Sei $\mu$ vollständiges Maß, $f$ $\mu$-messbar und $f = g$ $\mu$-fast überall \\
$\xRightarrow{\text{Lemma I.5}} g$ $\mu$-messbar \\
$\Rightarrow \int f d\mu = \int g d\mu$, falls einer der beiden existiert.
\end{Bemerkung}

%Lemma II.8: Tschebyscheff-Ungleichung
\begin{Lemma}
\emph{Tschebyscheff-Ungleichung} \\
Für eine $\mathscr A$-messbare Funktion $f: X \rightarrow [0, \infty]$ mit $\int f d\mu < \infty$ gilt: \\
\begin{equation}
	\mu(\{f \geq s\}) \leq
	\begin{cases}
		\dfrac{1}{s}\int f d\mu & s \in (0, \infty) \\
		0 & s = \infty
	\end{cases}
\end{equation}
\end{Lemma}

%Lemma II.9
\begin{Lemma}
$f: X \rightarrow \overline{\mathbb R}$ sei $\mathscr A$-messbar.
\begin{enumerate}
	\item Ist $\int f d\mu < \infty \Rightarrow \{f = \infty\}$ ist $\mu$-Nullmenge.
	\item Ist $f \geq 0$ und $\int f d\mu = 0 \Rightarrow \{f > 0\}$ ist $\mu$-Nullmenge.
\end{enumerate}
\end{Lemma}


%VL 30.11.15


%Satz II.10 Satz von der monotonen Konvergenz von Beppo Levi
\begin{Satz}
\emph{Satz von der monotonen Konvergenz von Beppo Levi} \\
Sei $(X, \mathscr A, \mu)$ Maßraum und $f_k: X \rightarrow [0, \infty]$ sei Folge $\mathscr A$-messbarer Funktionen mit $f_1 \leq f_2 \leq ...$ Definiere $f: [0, \infty]$ durch $f(x) = \lim_{k \rightarrow \infty} f_k (x)$. Dann gilt:
\begin{center}
	$\int f d\mu = \lim_{k\rightarrow \infty} \int f_k d\mu$
\end{center}
\end{Satz}


%Bsp.
\begin{Beispiel}
$f_{\epsilon} (x) = \dfrac{1}{2\epsilon} \chi_{[-\epsilon, \epsilon]}(x)$
$(\mathbb R, \mathscr M(\mathscr L^1), \mathscr L^1)$ Maßraum, $\epsilon > 0$, $f_\epsilon \, \mathscr M(\mathscr L^1)$-messbar.
\begin{equation}
	\lim_{\epsilon \searrow 0} f_\epsilon (x) = f_0 (x) =
	\begin{cases}
		0 & x \ne 0 \\
		\infty & x = 0
	\end{cases}
\end{equation}
$0 = \int f_0 \, d\mathscr L^1 \ne \lim_{\epsilon \rightarrow 0} \int f_\epsilon \, d\mathscr L^1$ $\overequal{\lim_{\epsilon \searrow 0}}$ $\dfrac{1}{2\epsilon} \mathscr L^1([-\epsilon, \epsilon]) =1$
\end{Beispiel}


%Satz II.11
\begin{Satz}
$(X, \mathscr A, \mu)$ Maßraum. Für $\mathscr A$-messbare Funktionen $f, g: X \rightarrow \overline{\mathbb R} $ und $\alpha, \beta \in \mathbb R$ sei $\alpha\int f \, d\mu + \beta \int g \, d\mu$ in $\overline{\mathbb R}$ definiert. Dann ist $\alpha f + \beta g$ außerhalb einer $\mu$-Nullmenge $N$ definiert und mit $\alpha f + \beta g = 0$ auf $N$ gilt: \\
\begin{center}
	$\int (\alpha f + \beta g) \, d\mu = \alpha \int f \, d\mu + \beta \int g \, d\mu$
\end{center}
\end{Satz}

%Def II.6
\begin{Definition}
$(X, \mathscr A, \mu)$ Maßraum und $E \subset X$ sei $\mathscr A$-messbar. Dann setzten wir $($falls das rechte Integral existiert$)$:
\begin{center}
	$\int_E f \, d\mu := \int f \chi_E \, d\mu$
\end{center}
$f$ heißt \underline{auf $E$ integrierbar}, falls $f \chi_E$ integrierbar ist.
\end{Definition}

%Bem.
\begin{Bemerkung}
$(f \chi_E)^\pm = f^\pm \chi_E \leq f^\pm$. $f$ integrierbar $\Rightarrow$ $f$ integrierbar auf $E$. \\
$f \geq 0$ $\Rightarrow$ $\int_E f \, d\mu$ existiert $\forall E \, \mathscr A$-messbar
\end{Bemerkung}

%Behauptung, zu der er dann noch lange Gedanken/Beweise aufgeschrieben hat. Aufschrieb aus verständlichen Gründen unvollständig, dreimal darfst du raten, wer da die Vorlesung gehalten hat.

\emph{Behauptung}: \\
$\int_{\mathbb R^n \backslash \mathscr B_1 (0)} f \, d\mathscr L^n < \infty$ $\Leftrightarrow$ $\alpha > n$
%Das war dann der 30.11.




%VL 04.12.15

%Bsp.
%Viel Spaß, sollte das mal jemand Korrektur lesen müssen. Es hat auch super viel Spaß gemacht, das zu schreiben. 04.01.16 - Freude! :)
\begin{Beispiel}
$f: \mathbb R^n \rightarrow \mathbb R$, $f(x) = |x|^{-\alpha}, \alpha \in \mathbb R$ \\
\emph{Behauptung:} $\int\limits_{\mathbb R^n\backslash \mathscr B_1(0)} f d\mathscr L^n < \infty$ $\Leftrightarrow$ $\alpha > n$ \\
%das hier macht nicht so wirklich Sinn. Wieso über B oder ohne gleich??
$\int\limits_{\mathscr B_1(0)} f d\mathscr L^n < \infty$ $\Leftrightarrow$ $\alpha > n$ \\
$A_k := \left\{x \in \mathbb R^n | 2^k \leq |x| < 2^{k+1}\right\}, k \in \mathbb Z$ \\
$\bigcup\limits_{k \in \mathbb Z} A_k = \mathbb R^n\backslash\{0\}$ \\
$g = \sum\limits_{k \in \mathbb Z} 2^{-k\alpha}\chi_{A_k}$ \\
Es gilt: $2^{-\alpha}g \leq f \leq g$ $\forall \alpha \geq 0$ \\
Auf $\mathbb R^n\backslash\{0\}: 2^{-\alpha}g \geq f\geq g$ $\forall \alpha < 0$ \\
Sei $x \in A_k$ für ein $k \in \mathbb Z \Rightarrow 2^k \leq |x| \leq 2^{k+1}$ und $g(x) = 2^{-k\alpha}$ \\
$\alpha \geq 0 \Rightarrow 2^{k\alpha} \leq |x|^\alpha \leq 2^{(k+1)\alpha} \Rightarrow |x|^{-\alpha} \geq 2^{-k\alpha - \alpha}, |x|^{-\alpha} \leq 2^{-k\alpha} = g(x) = 2^{-\alpha} g(x)$ \\
Es reicht, die Behauptung für $g$ zu zeigen! \\
Satz II.7: $\mathscr L^n(A_k) = 2^{kn}\underbrace{\mathscr L^n(A_0)}_{=: \nu_n \in (0, \infty)}$ \\
$A_k = 2^k A_0$ \\
Es ist weiter $\sum\limits_{k=0}^l 2^{-k\alpha}\chi_{A_k}$ $\underrightarrow{k \rightarrow \infty}$ $\sum\limits_{k=0}^\infty 2^{-k\alpha}\chi_{A_k} = g\chi_{\mathbb R^n \backslash \mathscr B_1(0)}$ punktweise und monoton. \\
Satz II.10 \\
$\rightarrow \int\limits_{\mathbb R^n\backslash \mathscr B_1(0)} g d\mathscr L^n = \int g\chi_{\mathbb R^n\backslash B_1(0)} d\mathscr L^n = \int \sum\limits_{k=1}^\infty 2^{-k\alpha}\chi_{A_k} d\mathscr L^n = \int \lim_{l \rightarrow \infty} \sum\limits_{k=0}^l 2^{-k\alpha} \chi_{A_k} d\mathscr L^n$
$\overequal{\text{Satz II.10}} \, \, \, \, \lim_{l \rightarrow \infty} \sum\limits_{k=0}^l \int 2^{-k\alpha}\chi_{A_k} d\mathscr L^n = \sum\limits_{k=0}^\infty 2^{-k\alpha} \int \chi_{A_k} d\mathscr L^n = \sum\limits_{k=0}^\infty 2^{-k\alpha} \mathscr L^n (A_k)$ \\
\begin{equation}
= \sum\limits_{k=0}^\infty  \nu_n 2^{k(n-\alpha)} =
\begin{cases}
	\nu_n \cdotp \dfrac{1}{(1-2^{n-\alpha})} & , \alpha > n \\
	\infty & , \text{sonst}
\end{cases}
\end{equation}
\begin{equation}
\overline{g}(x) :=
\begin{cases}
	g(x) & \forall x \in \mathbb R^n \backslash \{0\} \\
	\infty & x = 0
\end{cases}
\end{equation}
$\xRightarrow{\alpha < 0} 2^{-\alpha}\overline{g} \geq f \geq g$ $\forall x \in \mathbb R^n$ \\
$\alpha > 0: 2^{-\alpha}g \leq f \leq \overline g$ \\
$\mathscr L^n$ ist ein \underline{vollständiges Maß} \\
$\xRightarrow{\text{Bem. letzten Freitag}}$ $\int g \chi_{\mathscr B_1 (0)} d\mathscr L^n = \int \overline g \chi_{\mathscr B_1(0)} d\mathscr L^n$
\end{Beispiel}

%Satz II.12
\begin{Satz}
Sei $f: X \rightarrow \overline{\mathbb R}$ messbar, $(X, \mathscr A, \mu)$ Maßraum. Dann gelten:
\begin{enumerate}
	\item $f$ integrierbar $\Leftrightarrow$ $|f|$ integrierbar.
	\item $\left| \int f d\mu \right| \leq \int |f| d\mu$, falls $\int f d\mu$ existiert.
	\item Ist $g: X \rightarrow [0, \infty]$ $\mathscr A$-messbar mit $|f| \leq g$ $\mu$-fast überall und $\int g d\mu < \infty$ $\Rightarrow f$ ist integrierbar.
\end{enumerate}
\end{Satz}

%Bsp.
\begin{Beispiel}
$f: \mathbb R^n \rightarrow \mathbb R$ $\mathscr M(\mathscr L^n)$-messbar und es existiert $c \in [0, \infty)$ mit $|f(x)| \leq c|x|^{-\alpha}$ $\mathscr L^n$-fast überall in $\mathscr B_\epsilon (0)$ mit $\alpha < n$ \\
Bzw. $|f(x)| \leq c|x|^{-\alpha}$ $\mathscr L^n$-fast überall in $\mathbb R^n \backslash \mathscr B_\mathbb R (0)$ mit $\alpha > n$ $\Rightarrow f$ ist auf $\mathscr B_\epsilon (0)$ bzw. $\mathbb R^n \backslash \mathscr B_\mathscr R (0)$ integrierbar $\mathscr L^n$.
\end{Beispiel}


%II.3 Konvergenzsätze und L^p-Räume
\section{Konvergenzsätze und $L^p$-Räume}

%Satz II.13: Lemma von Fatou
\begin{Satz}
\emph{Lemma von Fatou} \\
$(X, \mathscr A, \mu)$ Maßraum. Sei $f_k: X \rightarrow [0, \infty]$ Folge von $\mathscr A$-messbaren Funktionen. Für $f: X \rightarrow \overline{\mathbb R}, f(x) = \liminf_{k \rightarrow \infty} f_k (x)$ gilt: \\
\begin{center}
$\int f d\mu \leq \liminf_{k \rightarrow \infty} \int f_k d\mu$
\end{center}

Bsp. $f_\epsilon = \dfrac{1}{2\epsilon} \chi_{[-\epsilon, \epsilon]}$
\end{Satz}

%Satz II.14 Satz über majorisierte Konvergenz, Satz von Lebesgue
\begin{Satz}
\emph{Satz über majorisierte Konvergenz, Satz von Lebesgue} \\
Sei $(X, \mathscr A, \mu)$ Maßruam und $f_1, f_2, ...$ Folge $\mathscr A$-messbarer Funktionen mit \\$f(x) := \lim_{k \rightarrow \infty} f_k (x)$ für $\mu$-fast alle $x \in X$. \\
Es gebe eine integrierbare Funktion $($bzgl. $\mu)$ $g: X \rightarrow [0, \infty]$ mit \\
$\sup_{k \in \mathbb N} |f_k (x)| \leq g(x)$ für $\mu$-fast alle $x \in X$. Dann ist $f$ integrierbar bzgl $\mu$ mit
\begin{center}
$\int f d\mu = \lim_{k \rightarrow \infty} \int f_k d\mu$.
\end{center}
\end{Satz}

%Bem.
\begin{Bemerkung}
Es gilt sogar: $\|f - f_k\|_{L^1(\mu)} = \int |f - f_k|d\mu$ $\underrightarrow{k \rightarrow \infty}$ $0$
\end{Bemerkung}

\emph{Riemann-Integral}
Welche Funktionen sind $R$-integrierbar? \\
$f: I:=[a, b] \rightarrow \mathbb R$ \\
Sei $Z$ eine Unterteilung von $I$, d.h. $a=x_0\leq x_1 \leq ... \leq x_N = b$ in Teilintervalle $(x_{i-1}, x_i) \ni \xi_i$ \\
$f$ ist $R$-integrierbar $\Leftrightarrow$ \\
$S_{Z, \xi} (f) := \sum\limits_{i=1}^N f(\xi_i)(x_i - x_{i-1})$ konvergiert für $\Delta(Z) \rightarrow 0$ \\


%Def. Das ist eine Definition ohne Nummer? Macht die die Nummerierung dann kaputt? Oder hat er die Nummer da nur vergessen? Ich denke, es ist nur eine Unterdef, also ohne die normale Nummerierung.
\emph{Definition} \\
$\overline{s_Z}(f) = \sum\limits_{i=1}^N (\sup f)(x_i  - x_{i-1})$ \underline{Obersumme} \\
%in deinem Aufschrieb ist da noch ein [x_{i-1}, x_i) schräg unter dem sup. Was macht das da? 04.12.15, Seite k oben
\underline{$s_Z$}$(f) = \sum\limits_{i=1}^N (\inf f)(x_i - x_{i-1})$ \underline{Untersumme} \\
Es gilt:
$\int\limits_a^b f(x) dx = s \Leftrightarrow \sup_z \underline{s_Z}(f) = \inf \overline{s_Z}(f) = s$ \\
$($Mehr Details heute 15:45 bei unserem Lieblingsübungsleiter$)$

%Satz II.15
\begin{Satz}
Sei $f: I \rightarrow \mathbb R$ beschränkte Funktion auf $I = [a, b]$. Dann gilt: \\
$f R$-integrierbar $\Leftrightarrow \mathscr L^1(\Set{x \in I | f \text{ ist nicht stetig in } x }) = 0$ \\
In diesem Fall ist $f$ auch Lebesgue-integrierbar und die Integrale stimmen überein.
\end{Satz}

%Juchu, das war die VL vom 04.12.

%Jetzt der 07.12. Die erste Seite ist eine Zusammenfassung bisheriger Sätze

%Satz II.16 Stetigkeit von Parameterintegralen
\begin{Satz}
\emph{Stetigkeit von Parameterintegralen} \\
$X$ ein metrischer Raum, $\mu$ ein Maß auf $Y$, $f: X \times Y \rightarrow \mathbb R$ mit den Eigenschaften:
\begin{enumerate}
	\item $\forall x \in X:$ $f(x, \cdotp)$ integrierbar
	\item $f(\cdotp, y)$ ist in $x_0$ für $\mu$-fast alle $y$ stetig.
	\item $\exists g: Y \rightarrow \mathbb R^+$ integrierbar, $|f(x, y)| \leq g(y)$ $\forall y \in Y \backslash Nx$, $Nx$ $\mu$-Nullmenge
\end{enumerate}
$\Rightarrow F: X \rightarrow \mathbb R$, $F(x):= \int f(x, y) d\mu$ ist stetig in $x_0$.
\end{Satz}

%Satz II.17 Differentation unter den Integralzeichen
\begin{Satz}
\emph{Differentation unter den Integralzeichen} \\
$I$ offenes Integral, $\mu$-Maß auf $Y$, $f: I \times Y \rightarrow \mathbb R$ mit
\begin{enumerate}
	\item $f(x, \cdotp)$ integrierbar für alle $x \in I$
	\item $f(\cdotp, y)$ ist in $x_0 \in I$ differenzierbar für $\mu$-fast alle $y$
	\item $\dfrac{|f(x, y) - f(x_0, y)|}{|x - x_0|} \leq g(y)$ für alle $y \in Y \backslash Nx, \mu(Nx)= 0$ und $f$ ist $\mu$-integrierbar \\
\end{enumerate}
$\Rightarrow F: I \rightarrow \mathbb R, F(x):= \int f(x, y)d\mu(y)$ ist in $x_0$ differenzierbar mit $ F'(x) = \int \dfrac{\partial f}{\partial x}(x_0, y) d\mu(y)$

\end{Satz}

%Lemma II.10
\begin{Lemma}
Sei $U \subset \mathbb R^n$ offen, $\mu$ Maß auf $Y$, $f: U \times Y \rightarrow \mathbb R$ mit
\begin{enumerate}
	\item $f(x, \cdotp)$ ist $\mu$-integrierbar für alle $x$
	\item $f(\cdotp, y)$ ist in $\mathscr C^1(U)$ für $\mu$-fast alle $y$.
	%Was tut das D da? ist das ein Differentialsymbol? VL 07.12. Seite f
	\item $|D_x f(x, y)| \leq g(y)$ für alle $x$ und $g$ integrierbar.
\end{enumerate}
$\Rightarrow F: U \rightarrow \mathbb R, F(x):= \int f(x, y) d\mu(y)$ ist $\mathscr C^1(U)$ mit $\dfrac{\partial F}{\partial x}(x) = \int \dfrac{\partial f}{\partial x}(x, y) d\mu(y)$
\end{Lemma}

%Bsp.
\begin{Beispiel}
$\phi \in \mathscr C_c^\infty(\mathbb R^n, \mathbb R)$ $($d.h. $spt(\phi) =\Set{x| \phi(x) \ne 0}$ ist kompakt, $spt$ steht wohl für support.$)$  \\
%Was auch immer das ist. Also 'support'
$\forall h$ Lebesgue-integrierbar ist \\
$\phi \Asterisk h(x) = \int \phi(x-y)f(y) d\mathscr L^1(y)$ $(\Asterisk$ ist \emph{Faltung}$)$ ist \underline{glatt}
\end{Beispiel}

%Def II.7 L^p-Raum

%juchuu, hier beginnt, das Chaos mit den Lp-Räumen und dem Lebesgueding
\begin{Definition}
\emph{$L^p$-Raum} \\
Sei $f: X \rightarrow \overline{\mathbb R}$ $\mu$-messbar, $1 \leq p \leq \infty$ \\
\begin{equation}
\|f\|_{L^p (\mu)} =
\begin{cases}
	\left( \int |f|^p d\mu\right) ^{\dfrac{1}{p}} & , \text{für } 1 \leq p < \infty \\
	\inf \{s > 0 | \mu(\{|f| > s\}) = \infty \} & , p = \infty
\end{cases}
\end{equation}
$L^p(\mu):= \Set{f \mu\text{-messbar mit }\|f\|_{L^p (\mu)} < \infty}$ auf $L^p$ definieren die Äquivalenzrelation
\begin{center}
	$f \sim g \Leftrightarrow f(x) = g(x)$ für $\mu$-fast alle $x$
\end{center}
Definiere $L^p(\mu) := L^p(\mu) /_\sim$ \\
Häufig schreiben wir $\|f\|_{L^p}$ anstelle von $\|f\|_{L^p (\mu)}$
\end{Definition}

%Def II.8
\begin{Definition}
$E \subset X$ $\mu$-messbar, $f: X \rightarrow \overline{\mathbb R}$ $\mu$-messbar, definiere die Fortsetzung
\begin{equation}
f_0(x) =
\begin{cases}
	f(x) & x \in E \\
	0 & \text{sonst}
\end{cases}
\end{equation}
$L^p(E) := \{f: E \rightarrow \overline{\mathbb R}, f_0 \in L^p(x)\}$ und $L^p(E, \mu) = L^p(E, \mu)/_\sim$
\end{Definition}

%Bem.
%ist es komisch, dass ich den Sinn dieser Bemerkung nicht verstehe? Gehört in der ersten Gleichung villeicht das ^p weg?
\begin{Bemerkung}
$L(E, \mu) = L^p(\mu|_E)$ und $L^p(E, \mu) = L^p(\mu|_E)$
\end{Bemerkung}

%Satz II.18
\begin{Satz}
Für $1 \leq p \leq \infty$ ist $(L^p(\mu),\, \|\cdotp\|_{L^p(\mu)})$ ein normierter Vektorraum. Insbesondere gilt: für $\lambda \in \mathbb R$, $f, g \in L^p(\mu)$:
\begin{enumerate}
	\item $\|f\|_{L^p} = 0 \Rightarrow f=0$ $\mu$-fast überall
	\item $f \in L^p(\mu)$ und $\|\lambda f\|_{L^p} = |\lambda|\|f\|_{L^p}$
	\item $f +g \in L^p(\mu)$ und $\|f + g\|_{L^p} \leq \|f\|_{L^p} + \|g\|_{L^p}$
\end{enumerate}

%Lemma II.11 Youngsche Ungleichung
\begin{Lemma}
\emph{Youngsche Ungleichung} \\
Für $1 < p, q < \infty$ mit $\dfrac{1}{p}+\dfrac{1}{q} =1$ gilt $\forall x, y \geq 0$
\begin{center}
	$xy \leq \dfrac{1}{p}x^p + \dfrac{1}{q}y^q$
\end{center}
\end{Lemma}
\end{Satz}

%Satz II.19 Hölder Ungleichung
\begin{Satz}
\emph{Höldersche Ungleichung} \\
$f, g$ $\mu$-messbar, $1 \leq p, q \leq \infty$ mit $\dfrac{1}{p}+\dfrac{1}{q} =1$ \\
\begin{center}
	$\Rightarrow \left| \int f, g, d\mu\right| \leq \|f\|_{L^p}\|g\|_{L^p}$
\end{center}
\end{Satz}


%Satz II.20 Minkovski Ungleichung
\begin{Satz}
\emph{Minkovski-Ungleichung} \\
$f, g \in L^p(\mu), 1 \leq p \leq \infty$ dann gilt:
\begin{center}
	$\|f+g\|_{L^p} \leq \|f\|_{L^p} + \|g\|_{L^p}$
\end{center}
\end{Satz}

%Lemma II.12
\begin{Lemma}
Sei $1 \leq p \leq \infty$, $f_k := \sum\limits_{i=1}^k u_i$, $u_i \in L^p(\mu)$, falls $\sum\limits_{i=1}^\infty \|u_i\|_{L^p} < \infty$ dann gilt:
\begin{enumerate}
	\item $\lim_{k \rightarrow \infty} f_k (x) = f(x)$ existiert und $\forall x \in X\backslash N$ mit $\mu(N) =0$
	\item Mit $f(x) = 0$ $\forall x \in N$ folgt $f \in L^p(\mu)$
	\item $\|f-f_k\|_{L^p} \rightarrow 0$
\end{enumerate}
\end{Lemma}

%Jup, der 07.12. ist dann auch soweit fertig.




%Freitag, 11.12.15 fertig, aber
%UNBEDINGT NOCHMAL KONTROLLIEREN! DÜRFTEN VIELE FEHLER DRIN SEIN
%erstes
%Satz II.21 Riesz-Fischer
\begin{Satz}
\emph{Riesz-Fischer} \\
$\left(L^p(\mu), \|\cdotp\|_{L^p}\right)$ ist vollständig, d.h. ein Banachraum.
\end{Satz}

%Lemma II.13
\begin{Lemma}
Konvergiert $\{f_k\}_k \subset L^p (\mu)$ gegen $f \in L^p$, dann existiert eine Teilfolge $\{f_{k_j}\}_{j \in \mathbb N}$, sodass $f_k \rightarrow f$ punktweise $\mu$-fast überall.
\end{Lemma}

%Bsp
\begin{Beispiel}
für $n = 2^k + j$ definiere $f_n(x) = \chi_{\left[\dfrac{j}{2^k}, \dfrac{j+1}{2^k}\right]}$, $0 \leq j \leq 2^k$ \\
Für $p \rightarrow \infty: \|f_n\|_{L^p(\mathscr L^1)} = L^1\left(\left[\dfrac{j}{1^k}, \dfrac{j+1}{2^k} \right] ^{\dfrac{1}{p}} \right) = 2^{-\dfrac{k}{p}} \rightarrow 0$, d.h. $f_n \rightarrow 0$ in $L^p(\mathscr L^1)$ \\
Also $\forall x \in [0, 1]: \limsup_{n \rightarrow \infty} f_n (x) = 1$, d.h. $f_n$ konvergiert nicht punktweise fast überall. \\
mögliche Teilfolge $\{f_{2^k}\}_{k \in \mathbb N}$ konvergiert punktweise fast überall.
\end{Beispiel}

%Satz II.22: Dichtheit C^0_c (x) in L^p (\Omega)
\begin{Satz}
\emph{Dichtheit $C^0_c (x)$ in $L^p(\Omega)$}: \\
Sei $\Omega \subset \mathbb R^n$ offen und $1 \leq p < \infty$. Dann gibt es zu jedem $f \in L^p$ eine Folge $f_k \in C^0_c (\Omega)$ mit $\|f_k - f\|_{L^p} \rightarrow 0$\\
$(\mu$ ist Radon-Maß, z.B. Lebesgue-Maß$)$
\end{Satz}

%Def II.9
\begin{Definition}
Der \underline{Träger} einer Funktion $f: \Omega \rightarrow \mathbb R$ ist $spt(f) = \overline{\{x \in \Omega: f(x) \ne 0\}} $
und $C_c^0 (\Omega) = \{g \in C^0(\Omega): sprt(y) \, \text{kompakt} \subset \Omega\}$
\end{Definition}


%Bem.
\begin{Bemerkung}
Falls $\mu$ ein Radonmaß, folgt aus $Satz \, II.22$ und $Satz \,  II.21$, dass $\overline{C^0_c (\Omega)}^{\|\cdotp\|_{L^p}} = L^p (\mu)$ für $p < \infty$
\end{Bemerkung}

%Satz II.23: Konvergenzsatz von Vitali
\begin{Satz}
\emph{Konvergenzsatz von Vitali} \\
Sei $\mu$ ein Maß auf $A \subset P(X)$ und $1 \leq p < \infty$ \\
$\{f_n\}_{n \in \mathbb N} \subset L^p(\mu)$ mit $f_n \rightarrow f$ $\mu$-fast überall, dann ist äquivalent:
\begin{enumerate}
	\item $f \subset L^p(\mu)$ und $\lim_{k \rightarrow \infty} \|f_k -f \|_{L^p (\mu)} = 0$
	\item $\{f_k\}_{k \in \mathbb N}$ ist gleichgradig integrierbar mit $\lambda(A) = \lim_{n \rightarrow \infty} \int_A |f_n|^p \, d\mu$ gilt:
	\begin{enumerate}
		\item $\forall \epsilon > 0$ $\exists \delta > 0$, sodass $\lambda(A) < \epsilon$, falls $\mu(A) < \delta$
		\item $\forall \epsilon > 0$ $\exists E \in \mathscr A$ mit $\mu(E) < \infty$ und $\lambda(X \backslash E) < \epsilon$
	\end{enumerate}
\end{enumerate}

\end{Satz}


%Montag, 14.12.15 fertig




\newpage
%Ich fange mal noch mit Kapitel III an
%KAPITEL 3
\chapter{Der Satz von Fubini und die Transformationsformel}
Ana I: $f: [a, b] \times [c, d] \rightarrow \mathbb R$ stetig \\
$\underrightarrow{Fubini} \int\limits_a^b \left( \int\limits_c^d f(x, y) dy\right) dx = \int\limits_c^d \left( \int\limits_a^b f(x, y) dx \right) dy$

%Def III.1
\begin{Definition}
Seien $\alpha_i$ äußere Maße auf $X_i, i = 1, ... n$. Eine Menge $P \subset X_1 \times ... \times X_n$ heißt \underline{Produktmenge}, wenn es $\mathscr M(\alpha_i)$ messbare Mengen $A_i \subset X_i$ gibt mit $P = A_1 \times ... \times A_n$
\end{Definition}

%Lemma III.1
\begin{Lemma}
$\alpha_i$ äußere Maße auf $X_i, 1 \leq i \leq n$. Dann gilt:
\begin{enumerate}
	\item Das System $\mathscr P$ der Produktmengen ist ein Halbring.
	\item Die Funktion $\lambda: \mathscr P \rightarrow [0, \infty], \, \lambda(A_1 \times ... \times A_n) = \prod\limits_{i=1}^n \alpha_i(A_i)$ ist ein Prämaß.
\end{enumerate}
\end{Lemma}

%Def III.2
\begin{Definition}
Seien $\alpha_i$ äußere Maße auf $X_i, 1 \leq i \leq n$. Das \underline{Produktmaß} $\alpha_1 \times ... \times \alpha_n$ einer Menge $E \subset X_1 \times ... \times X_n$ ist definiert durch
\begin{center}
	$(\alpha_1 \times ... \times \alpha_n)(E) = \inf\Set{ \sum\limits_{j=1}^\infty \lambda(P_j) | P_j \in \mathscr P, E \subset \bigcup\limits_{j=1}^\infty P_j }$

\end{center}
\end{Definition}

\emph{bekannt}:
$\mathbb R^n = \mathbb R^{n_1} \times \mathbb R^{n_2}$, $n_1 + n_2 = n$

%Lemma III.2
\begin{Lemma}
Es gilt
\begin{center}
	$\mathscr L^n = \mathscr L^{n_1} \times \mathscr L^{n_2}$
\end{center}
\end{Lemma}

%Satz III.1: Fubini
%wuhu, Monstersatz

\begin{Satz}
\emph{Satz von Fubini} \\
Es seien $\mu, \nu$ äußere Maße auf $X$ bzw. $Y$. Dann gilt:
\begin{enumerate}
	\item $\forall S \subset X \times Y$ $\exists R \in \mathscr M(\mu \times \nu)$ mit $S \subset R$ und
	\begin{center}
		$(\mu \times \nu)(S) = (\mu \times \nu)(R)$
	\end{center}
	\item Für $A \in \mathscr M(\mu), B \in \mathscr M(\nu)$ ist $A \times B \in \mathscr M(\mu \times \nu)$ und
	\begin{center}
		$(\mu \times \nu)(A \times B) = \mu(A)\nu(B)$
	\end{center}
	\item Für $S \subset X \times Y$, $S \in \mathscr M(\mu \times \nu)$ und $S$ sei $\sigma$-endlich bezüglich $\mu \times \nu$
	\par
	$\left(\text{d.h.} \exists S_n \subset X \times Y \text{ mit }\bigcup S_n = S\text{ und }(\mu \times \nu)(S_n) < \infty \,\forall n\right)$.
	\par
	Dann gilt:
	\par
	$S_y := \{x: (x, y) \in S\}$ ist für $\nu$-fast alle $y \in Y$ in $\mathscr M(\mu)$, \\
	$(y \mapsto \mu(S_y))$ ist $\mathscr M(\nu)$-messbar. \\
	\par
	$S^x := \{y : (x, y) \in S\}$ ist für $\mu$-fast alle $x \in X$ in $\mathscr M(\nu)$, \\
	$(x \mapsto \nu(S^x))$ ist $\mathscr M(\mu)$-messbar und
	\begin{center}
		$(\mu \times \nu)(S) = \int \mu(S_y)\, d\nu(y) = \int \nu(S^x) \, d\mu(x)$ \\
	\end{center}
	\emph{Prinzip von \underline{Cavalieri}}

	\item Für $f: X \times Y \rightarrow [0, \infty]$ $\mathscr M(\mu \times \nu)$-messbar und $\{f > 0\}$ sei $\sigma$-endlich bezüglich $\mu \times \nu$ gilt:
	\par
	$(x \mapsto f(x, y))$ ist für $\nu$-fast alle $y \in Y$ $\mathscr M(\mu)$-messbar.
	\par
	$\left(y \mapsto \int f(x, y) \, d\mu(x)\right) $ ist $\mathscr M(\nu)$-messbar,
	\par
	$(y \mapsto f(x, y))$ ist für $\mu$-fast alle $x \in X$ $\mathscr M(\nu)$-messbar,
	\par
	$\left( x \mapsto \int f(x, y) \, d\nu(y)\right)$ ist $\mathscr M(\mu)$-messbar mit
	\begin{center}
		$\int f \, d(\mu \times \nu) = \int \left( \int f(x, y) \, d\mu(x) \right) \, d\nu(y) = \int \left( \int f(x, y) \, d\nu(y) \right) \, d\mu(x)$ \\
	\end{center}
	\emph{\underline{Tonelli}}

	\item Für $f \in L^1(\mu \times \nu)$ gilt:
	\par
	$(x \mapsto f(x, y)) \in L^1(\mu)$ für $\nu$-fast alle $y \in Y$,
	\par
	$(y \mapsto \int f(x, y)\, d\mu(x)) \in L^1(\nu)$,
	\par
	$(y \mapsto f(x, y)) \in L^1(\nu)$ für $\mu$-fast alle $x \in X$,
	\par
	$(x \mapsto \int f(x, y) \, d\nu(y)) \in L^1(\mu)$ und
	\begin{center}
		$\int f \, d(\mu \times \nu) = \int \int f(x, y) \, d\mu(x)d\nu(y) = \int \int f(x, y) \, d\nu(y)d\mu(x)$
	\end{center}
	\emph{\underline{Fubini}}
\end{enumerate}
\end{Satz}

%Bsp.
\begin{Beispiel}
Sei $\alpha_n = \mathscr L^n(\mathscr B_1(0))$,
$\mathscr L^n = \mathscr L^{n-1} \times \mathscr L^1$\\
\par
$\mathscr B_1(0) = \{z \in \mathbb R^n,|z| < 1\} = \{z = (x, y) \in \mathbb R^{n-1} \times \mathbb R: |x|^2 + y^2 <1\} $
\par\bigskip
$(\mathscr B_1(0))_y = \{x \in \mathbb R^{n-1}: |x| < \sqrt{1-y^2},\,  y \in (-1, 1]\} = \sqrt{1-y^2}\mathscr B_1^{n-1} (0) $
\par\bigskip
$\alpha_n = \mathscr L^n(\mathscr B^1(0)) = (\mathscr L^{n-1} \times \mathscr L^1)(\mathscr B_1(0))$
\par\bigskip
$ = \int\limits_{-1}^1 \mathscr L^{n-1}\left(\{x \in \mathbb R^{n-1}: |x| < \sqrt{1 -y^2}\}\right)\, d\mathscr L^1(y) $
\par\bigskip
$=\int\limits_{-1}^1 (1- y^2)^{\dfrac{n-1}{2}} \alpha_{n-1}\, d\mathscr L^1(y) = \alpha_{n-1} \int\limits_{-1}^1 (1 - y^2)^{\dfrac{n-1}{2}} \, dy $
\par\bigskip
$\overequal{y = \cos{\theta}}  \alpha_{n-1} \int\limits_0^\pi \sin^n \theta \, d\theta $
\par\bigskip
$=: \alpha_{n-1} A_n$  \\
\par\bigskip
$A_n = \int\limits_0^\pi \sin^n \theta\, d\theta$
\par\bigskip
$\rightarrow A_n = \dfrac{n-1}{n} A_{n-2}$ $\forall n \geq 2$,
\par\bigskip
 $A_0 = \pi, A_1 = 2..$
 \par\bigskip
$\Rightarrow A_{2k} = \dfrac{2k -1}{2k} \cdotp ... \cdotp \dfrac{1}{2} A_0 = \pi \prod\limits_{j=1}^k \dfrac{2j -1}{2j}$
\par\bigskip
$A_{2k+1} = \dfrac{2k}{2k+1} \cdotp ... \cdotp \dfrac{2}{3} A_1 = 2 \prod\limits_{j=1}^k \dfrac{2j}{2j+1}$
$\Rightarrow A_{2k+1} \cdotp A_{2k} = \dfrac{2\pi}{2k +1}$
\par\bigskip
$A_{2k} A_{2k-1} = \dfrac{2\pi}{2k} \dfrac{2k-1}{2k-1} = \dfrac{\pi}{k}$
\par\bigskip
$\alpha_{2k} = (A_{2k} \cdotp A_{2k-1})\cdotp ... \cdotp (A_2 \cdotp A_1) \alpha_0 = \dfrac{\pi^k}{k!}\underbrace{\alpha_0}_{=1} = \dfrac{\pi^k}{k!}$
\par\bigskip
$\alpha_{2k+1} = (A_{2k-1}\cdotp A_{2k}) ... (A_3 \cdotp A_2) \underbrace{\alpha_1}_{=2} = 2 \prod\limits_{j=1}^k \dfrac{2\pi}{2j+1} = \dfrac{\pi^k}{\left(k + \dfrac{1}{2}\right)\left(k- \dfrac{1}{2}\right)... \dfrac{1}{2}}$
\par\bigskip
$\alpha_3 = \dfrac{\pi}{\dfrac{3}{4}} = \dfrac{4\pi}{3}$
\par\bigskip
\emph{Bemerkung}: \\

$\alpha_n \rightarrow 0$ mit $n \rightarrow \infty$
\end{Beispiel}

%Das war der 18.12.15, der ansonsten nur aus dem Beweis des Satzen von Fubini bestand.

%21.12.15 fehlt noch
%von Michelle

%Bsp.
\begin{Beispiel}
$\mu$ $\sigma$-endliches äußeres Maß auf $X$ und $f: X \rightarrow [0, \infty]\, \mu$-messbar. $\phi: [0, \infty] \rightarrow [0, \infty]$ stetig mit $\phi(0) = 0$ und $\phi\lvert_{[0, \infty]}$ stetig differenzierbar mit $\phi'(t) >0$ \\
\emph{Behauptung}:
$\int_t \phi(f(x)) \, d\mu(x) = \int\limits_0^\infty \phi'(t)\mu(\{x \in X: f(x) >t\} \, dt$\\
\emph{Anwendung der Behauptung}:  \\
$\phi(t) = |t|^p, \, p>1, \, \phi'(t) = p \cdotp t^p \varrightwavearrow \int(f(x))^p \, d\mu(x) = \int\limits_0^\infty p\cdotp t^{p-1} \mu(\{f > t\}) \, dt$ \\
Betrachte $E = \{(x, t) \in X \times [0, \infty]: t < f(x)\}$ Produktmaß $\mu \times \mathscr L^1$. \\
Die Funktionen
\begin{itemize}
	\item $(x, t) \mapsto t$
	\item $(x, t) \mapsto f(x)$
	\item $(x, t) \mapsto \phi'(t)$
\end{itemize}
sind $\mu \times \mathscr L^1$-messbar. \\
\par
$\Rightarrow E \in \mathscr M(\mu \times \mathscr L^1) \varrightwavearrow (x, t) \mapsto \phi'(t) \chi_E(x, t)$ ist $\mu \times \mathscr L^1$-messbar \\
\par
$\Rightarrow \int_x \phi(f(x)) \, d\mu(x) = \int_x \left( \int\limits_0^{f(x)}\phi'(t)\, dt\right)d\mu(x) = \int_x\int\limits_0^\infty \phi'(t) \chi_E(x, t)\, dt \,d\mu(x)$
\par
$\overequal{Fubini}\, \int \phi'(t) \chi_E(x, t) \,d(\mu \times \mathscr L^1)(x, t) \,\overequal{Fubini} \,\int\limits_0^\infty \int_x \phi'(t)\chi_E(x, t) \,d\mu(x) \,d\mathscr L^1(x)$
\par
$ = \int\limits_0^\infty \phi'(t) \mu(\{x \in X: f(x) >t\})\, dt$ \\
$($Es fehlt die Überdeckung von kompakten Intervallen, damit Riemann = Lebesgue$)$
\end{Beispiel}

%Satz III.2
\begin{Satz}
$f, g\in \mathscr C^1(\mathbb R^1)$ und $(\partial_j f)g, f(\partial_j g), fg$ sind integrierbar bezüglich $\mathscr L^n$. Dann gilt:
\begin{center}
	$\int_{\mathbb R^n} (\partial_j f)g \, dx = - \int_{\mathbb R^n} f(\partial_j y) \, dx$
	\par\bigskip
	\emph{Notation}: $d\mathscr L^n = dx$
	\par\bigskip
	$\left(\int\limits_a^b f' \cdotp g \,dx = -\int\limits_a^b f\cdotp g' \, cx + [f\cdotp g]_{x=a}^{x=b} \right)$
\end{center}
\end{Satz}

\newpage

%neues Jahr, Freude:
%08.01.16 bei unserem Lieblingsübungsleiter -.-

\section{Transformationssatz}
%aber das ist vermutlich gar kein neuer Absatz, die Überschrift ist nur ungünstig groß?

%Ich habe mit Grapher das mit den Zeichnungen mal ausprobiert, das ist aber noch nicht so ganz tauglich. Also mache ich erstmal ohne weiter.

$\Phi: \mathbb R^n \rightarrow \mathbb R^n$ $\mathscr C^1$ Diffeomorphismus, $L^n(\Phi(A))$

%Def III.3
\begin{Definition}
Eine Abbildung $\Phi: U \rightarrow V$, $U, V \subset \mathbb R^n$ offen heißt \underline{$\mathscr C^1$-Diffeomorphismus}, falls:
\begin{enumerate}
	\item $\Phi$ bijektiv
	\item $\Phi, \Phi^{-1}$ sind stetig differenzierbar.
\end{enumerate}
\end{Definition}

%Bsp: Polarkoordinaten
\begin{Beispiel}
\emph{Polarkoordinaten} \\
$\Phi: (0, \infty) \times (0, 2\pi) \rightarrow \mathbb R^2 \backslash \{x, 0)| x \geq 0\}$ \\
$\Phi(r, \theta) = (r \cos \theta, r \sin \theta)$ \\
\begin{equation}
	\Phi^{-1}(x, y) =
	\begin{cases}
		(r, \arccos\left(\dfrac{x}{r}\right)) & , y \geq 0 \\
		(r, 2\pi - \arccos\left(\dfrac{x}{r}\right)) & , y < 0
	\end{cases}
\end{equation}
mit $r = \sqrt{|x|^2 + y^2} $\\
$\Phi^{-1} (x, y) = \left(r, \dfrac{\pi}{2} + \arccos\left(\dfrac{y}{r}\right)\right)$ für $x < 0$

%Und noch ein schönes Bild.
\end{Beispiel}

%Bsp
\begin{Beispiel}
Für $n = 1$: \\
falls $\Phi: I \rightarrow J$, $I, J \subset \mathbb R$ offene Intervalle \\
$\Phi$ $\mathscr C^1$ und streng monoton. \\
\begin{center}
	$\Rightarrow \int\limits_a^b f\left(\Phi(s)\right) \Phi^{-1}(s) \, ds = \int\limits_{\Phi(a)}^{\Phi(b)} f(s) \, ds$
\end{center}
für $f$ stetig und $[a, b] \subset I$. \\

\emph{1. Fall}: $\Phi(a) < \Phi(b)$ \\
$\alpha = \min{\{\Phi(a), \Phi(b)\}}, \beta = \max{\{\Phi(a), \Phi(b)\}} $ \\
$\Rightarrow \alpha = \Phi(a)$ und $\beta = \Phi(b)$ und $\Phi'(s) \geq 0$ $\forall s$ \\

\emph{2. Fall}: $\Phi(b) > \Phi(a)$ \\
$\Rightarrow \alpha= \Phi(b)$ ung $\beta = \Phi(a)$ und $\Phi'(s) \leq 0$ $\forall s$
\begin{center}
	$\Rightarrow \int\limits_a^b f(\Phi(s))|\Phi'(s)|\, ds = \int\limits_a^b f(s)\, ds$
\end{center}
\end{Beispiel}

\emph{Notation}: \\
für $x \in \mathbb R^n, f > 0$ \\
\begin{center}
	$Q(x, y) := \{y \in \mathbb R^n: \|y-x\|_\infty \leq f\}$
\end{center}
wobei $\|y-x\|_\infty = \max_{i = 1, .. n} |x^{i} - y^{i}|$


%Lemma III.4
\begin{Lemma}
Sei $U \subset \mathbb R^n$ offen, $x_0 \in U$. \\
$\Phi: U \rightarrow \mathbb R^n$ differenzierbar in $x_0$ und $D\phi(x) \in GL_n(\mathbb R^n)$. \\
Sei $Q_j = Q(x_i, l_j)$ gegeben mit $x_i \rightarrow x_0$ und $l_j \rightarrow 0$ für $x_0 \in Q_j$ $\forall j$
\begin{center}
	$\Rightarrow  \limsup_{j \rightarrow \infty} \dfrac{L^n(\phi(Q_j))}{L^n(Q_j)} \leq \left| \det (D\phi(x_0))\right|$
\end{center}
\end{Lemma}


%Satz III.3 Tranformationsformel
\begin{Satz}
\emph{Transformationsformel} \\
Seien $U, V \subset \mathbb R^n$ offen, $\Phi: U \rightarrow V$ $\mathscr C^1$ Diffeomorphismen. \\
$A \subset U $ $\mathscr L^n$-messbar, dann ist $\Phi(A) $ $\mathscr L^n$-messbar und \\
\begin{center}
	$\mathscr L^n(\Phi(A)) = \int |\det (D\Phi(x))| \, dx$\\
	$(dx$ anstelle von $d\mathscr L^n(x))$
\end{center}
Weiter gilt für $\mathscr L^n$-messbare Funktion $f: V \rightarrow \overline{\mathbb R}$:
\begin{center}
	$\int_{V = \Phi(U)} f(y) \, dy = \int_U f(\Phi(x))|\det (D\Phi(x))|\, dx$,
\end{center}
falls eines der Integrale definiert ist.
%Ab hier von Lamm am 11.01.16
Sei $\Phi: U \rightarrow V$, $U, V \subset \mathbb R^n$ offen, $\mathscr C^k$-Diffeomorphismus $(k \in \mathbb N)$: \\


\emph{Gramsche Matrix}: \\
$g \in (g_{ij})$,
\begin{center}
	$g(x) = D\Phi(x)^T D \Phi(x)$ \\
	bzw. $g_{ij}(x) = \langle \partial_i \Phi(x), \partial_j \Phi(x) \rangle$
\end{center}
$\varrightwavearrow g$ symmetrisch und strikt positiv definit:
\begin{center}
	$\langle g(x)v, v\rangle = |D\Phi(x)v|^2 > 0$ $\forall x \in U, v \in \mathbb R^n\backslash \{0\}$
\end{center}
$\varrightwavearrow g^{(x)}$ ist invertierbar;
$(g^{ij}$ meint Inverses$)$ \\
$\varrightwavearrow g^{ij}(x) := (g(x)^{-1})_{ij} \Rightarrow \sum\limits_{j=1}^n g^{ij} g_{jk} = \delta_{jk}$ \\
$\det g(x) = \det(D\Phi(x)^T D \Phi(x)) = |\det(D \Phi(x))|^2$

\begin{center}
	$\int_V f(y) \, dy = \int_U f(\Phi(x)) \sqrt{\det g(x)} \, dx$
\end{center}

\end{Satz}

%Bsp 1 Hirsch
\begin{Beispiel}
\emph{Gauß-Integral}:
$f(x, y) = e^{-(x^2 + y^2)}$ \\
$\int_{\mathbb R^2} f(x, y) \, \mathscr L^2(x, y) = \int_\mathbb R \int_\mathbb R e^{-x^2} e^{-y^2} \, dx\, dy = \left(\int \mathbb R e^{-x^2} \, dx \right)^2$ \\
Da $\det(D\phi(r, \theta)) = r$ $($s. Bsp.$)$ \\
$\Rightarrow \int_\mathbb R f(x, y) \, d\mathscr L^2(x,y) = \int_{(0, \infty) \times (0, 2\pi)} f(\phi(r, \theta))r \, dr\, d\theta = \int\limits_0^\infty \int\limits_0^{2\phi} e^{-r^2} \, d\theta\, r \, dr = 2\pi \int\limits_0^\infty e^{-r^2}r\, dr = -\pi e^{-r^2} |_0^\infty = \pi \Rightarrow \int_\mathbb R e^{-x^2} \, dx = \sqrt{\pi}$
\end{Beispiel}

%Younis, das kann ich nicht lesen. Letzte Seite der VL am 08.01
%Bsp. 2 Hirsch
\begin{Beispiel}
Sei $s \in Q_n(\mathbb R)$ und $\phi(x) = sx$ \\
$\Rightarrow D\phi(x) = s$ \\
$\Rightarrow \mathscr L^n(\phi(A)) = |\det(s)|\mathscr L^n(A)$ \\
$\int $
\end{Beispiel}

%abgesehen davon, dass ich mir an vielen Stellen nicht so ganz sicher bin, was du da gemeint hast, ist der 08.01. dann auch fertig

%ab hier mit 11.01. weiter
$\phi: U \rightarrow V$ Diffeomorphismus, $x \in U, y \in V$, $v: V \rightarrow \mathbb R$ Funktion, \\
$v = v(y) \varrightwavearrow u:= v \circ \phi: U \rightarrow \mathbb R, \, u(x) = v(\phi(x))$

%Bsp. Kugelkoordinaten
\begin{Beispiel}
\emph{Kugelkoordinaten - Polarkoordinaten in $\mathbb R^3$} \\
$\phi: \underbrace{(0, \infty) \times (0, \pi), \times (0, 2\pi)}_{=: U} \rightarrow \mathbb R^3 \backslash \{x, 0, z): x \geq 0\}$ \\
$\phi(r, \theta, \phi) = (r \sin\theta\cos\phi, r\sin\theta\sin\phi, r\cos\theta)$ \\
$\varrightwavearrow V = v(x, y, z) \varrightwavearrow u(r, \theta, \phi) = v(r\sin\theta\cos\phi, r\sin\theta\sin\phi, r\cos\theta)$
\end{Beispiel}


\chapter{Das Flächenmaß auf Untermannigfaltigkeiten und der Satz von Gauß}

%Def IV.1
\begin{Definition}
Sei $U \subset \mathbb R^n$ offen. Eine Abbildung $f \in \mathscr C^1(U, \mathbb R^{n+k})$ heißt \underline{Immersion}, falls gilt:
\begin{center}
	$ rang D f(x) = n \Leftrightarrow ker D f(x) = \{0\}$ $\forall x \in U$
\end{center}
\end{Definition}

%Bsp.
\begin{Beispiel}
$\gamma: [0, 1] \rightarrow \mathbb R^2$ mit $\gamma'(t) \ne 0$ $\forall t \in [0,1]$
\end{Beispiel}

%Bem.
\begin{Bemerkung}
Die Vektoren $\partial_1 f(x), ..., \partial_n f(x)$ bilden eine Basis von $Bild\,(Df(x))$ \\
Analog zu vorher: Definiere die Gramersche Matrix $($ bzw. induzierte Matrix$): g(x) = \underbrace{Df(x)^T}_{\in \mathbb R^{n \times (n+k)}}\cdotp \underbrace{Df(x)}_{\in \mathbb R^{(n+k) \times n}} \in \mathbb R^{n\times n}$ \\
$g= (g_{ij}), g_{ij}(x) = <\partial_i f(x), \partial_j f(x)>$ \\
Tatsächlich ist $(g_{ij})$ für eine beliebige Abbildung $f \in \mathscr C^1(U, \mathbb R^{n+k})$ definiert und positiv semi-definit. $<g(x)v, v> = |Df(x)v|^2 \geq 0$ \\
$g$ ist strikt positiv definit und damit invertierbar, falls
\begin{center}
	$|Df(x)v|^2 > 0$ $\forall v \in \mathbb R^n \backslash \{0\} \Leftrightarrow f$ Immersion
\end{center}
Es gilt: $ker g(x) = ker Df(x)$
\end{Bemerkung}

%Def. IV.2
\begin{Definition}
Sei $\phi \in \mathscr C^1(U, \mathbb R^n), U \subseteq \mathbb R^n$, Immersion mit induzierter Matrix $g$ und $B \subseteq U$ sei $\mathscr L^n$-messbar. Der $m$-dimensionale Flächeninhalt von $\phi$ auf $B$ ist
\begin{center}
	$A_B(\phi) := \int_B J \phi(x) \, d \mathscr L^n(x)$
\end{center}
mit $J\phi(x) := \sqrt{\det g(x)}$ $($Jacobische$)$
\end{Definition}

%Formatierung
%Klausurrelevante Beispiele
\begin{Beispiel}
\begin{enumerate}
	\item $\phi: I \rightarrow \mathbb R^n, I = (a, b)$ reguläre Kurve, $\phi \in \mathscr C^1(I, \mathbb R^n)$ $\varrightwavearrow \phi'(t) \ne 0 \, \forall t\in I \varrightwavearrow \phi$ Immersion \\
	$g(t) = <\phi'(t), \phi'(t) > = |\phi'(t)|^2$ \\
	$J\phi(t) = |\phi'(t)|$ \\
	$A_I(\phi) = \int_I |\phi'(t)|\, dt = L(\phi)$: Bodenlänge = eindimensionaler Flächeninhalt

	\item $U \subseteq \mathbb R^2: \phi: U \rightarrow \mathbb R^3$ Immersion \\
	$\varrightwavearrow J_x \phi, J_y \phi$ linear unabhängig auf $U$ $($reguläre Fläche$)$\\
	$D\phi = (J_x \phi, J_y \phi)$ \\
	$g= (g_{ij}) =$
	%Da fehlen Betragsstriche?
	$\begin{pmatrix}
      		|Dx\phi|^2 & 	<J_x\phi, J_y\phi \\
      		<J_x\phi, J_y\phi> &  	|D_y\phi|^2 \\
	\end{pmatrix}$ \\
	$\varrightwavearrow J\phi = \sqrt{|\partial_x \phi|^2 |\partial_y \phi|^2 - <\partial_x\phi, \partial_y\phi>^2}$ $(= |\partial_x\phi \times \partial_y \phi|: $Vektor-Produkt$)$ \\
	$U:= (0, \pi) \times (0, 2\pi), \phi(\theta, \Phi) = (\sin\theta\cos\Phi, \sin\theta\sin\Phi, \cos\theta)$ \\
	$\varrightwavearrow J\phi(\theta, \Phi) = \sin\theta, (g_{ij}) = $
	$\begin{pmatrix}
		1 & 0 \\
		0 & \sin^2 \theta \\
	\end{pmatrix}$

	%Bildchen

	$A_U(\phi) = \int_U \sin\theta\,d\mathscr L^2(\theta, \Phi) \overequal{\text{Fubini}} \int\limits_0^\pi \left( \int\limits_0^{2\pi} \sin\theta \, d\Phi\right) \,d\theta = 4 \pi$
	\
	item $\phi \in \mathscr C^1(U, \mathbb R^n), U \subseteq \mathbb R^n, \phi$ ist Immersion $\Leftrightarrow \det D\phi(x) \ne 0 \, \forall x \in U \Leftrightarrow \phi$ ist lokaler Diffeomorphismus. \\
	$J(\phi)= |\det D\phi|$ ist zusätzlcih $\phi: U \rightarrow \mathscr C(U)$ injektiv \\
	$\Rightarrow A_E (\phi) = \int_E |\det D\phi|\, d\mathscr L^n \overequal{\text{Satz 3.3}} \mathscr L^n(\phi(E)) \, \forall E \subseteq U \, \mathscr L^n$-messbar

	\item $U \subseteq \mathbb R^n, u \in \mathscr C^1(U, \mathbb R^k), m+k =n,\, \phi(x) := (x, u(x)), \phi: U \rightarrow \mathbb R^n, \phi \in \mathscr C^1(U, \mathbb R^n)$ Immersion \\
	$g_{ij}(x) = <\partial_i \phi(x), \partial_j \phi(x)> = <(e_i, \partial_i u(x)), (e_j, \partial_j u(x))> =
	\delta_{ij} + <\partial_i u(x), \partial_j u(x)> $ \\
	$g = (D\phi)^T D\phi = \underbrace{\mathbb E_{\mathbb R^m}}_{\text{Einheitsmatrix im }\mathbb R^m} + (Du)^TDu$  \\%mathbb E soll Einheitsmatrix, also mathbb 1 sein
	$\Rightarrow A_E(\phi) = \int_E \sqrt{(\det(\mathbb E_{\mathbb R^m} +(Du)^TDu)}\,d\mathscr L^m \, \forall E \subseteq U\, \mathscr L^m$-messbar
\end{enumerate}
\end{Beispiel}

\emph{Spezialfall: $k=1$} \\
\emph{Behauptung}: Zu jedem Punkt $x \in U$ existiert eine Orthonormalbasis $v_1, ... , v_m$ mit:
\begin{center}
	$Du(x) v_1 = |Du(x)|, Du(x) v_j = 0\, \forall j \geq 2$
\end{center}


%Satz IV.1
\begin{Satz}
Sei $\Phi \in \mathscr C^1(U, \tilde U)$ Diffeomorphismus. $U, \tilde U \subseteq \mathbb R^m$ und $\phi \in \mathscr C^1(\tilde U, \mathbb R^n)$ sei Immersion. Dann gilt:
\begin{center}
	$A_{\Phi(B)}(\phi) = A_B (\phi \circ \Phi) \, \forall B \subseteq U \, \mathscr L^n$-messbar
\end{center}
%Skript Seite 102 + Beweis (Sina), Bildchen vorhanden
\end{Satz}

%Satz IV:2
\begin{Satz}
$\phi_1, \phi_2$ wie oben, $M$ sein $m$-dimensionale $\mathscr C^1$-Untermannigfaltigkeit des $\mathbb R^n$. Dann sind $U_1 ', U_2 '$ offen in $\mathbb R^m$ und $\phi_2^{-1}\circ \phi_1$ ist $\mathscr C^1$-Diffeomorphismus.
\end{Satz}

%Satz IV.3
\begin{Satz}
Jede $m$-dimensionale $\mathscr C^1$-Untermannigfaltigkeit des $\mathbb R^n$ kann als abzählbare Vereinigung $M = \bigcup\limits_{i=1}^\infty K_i$ von relativ kompakten Mengen $K_i \subseteq \mathbb R^n$ geschrieben werden.
\end{Satz}

%Def. IV.3
\begin{Definition}
\emph{injektive Immersion} \\
Sei $M \subseteq \mathbb R^n$ $m$-dimensionale $\mathscr C^1$-Untermannigfaltigkeit. Eine \underline{lokale Parametrisierung} von $M$ ist ein $\mathscr C^1$-Homomorphismus. \\
$f: U \rightarrow f(U) \subseteq M \subseteq \mathbb R^2$ mit Rang $Df(x) = m \, \forall x \in U$, wobei $U \subseteq \mathbb R^m$ offen ist.
\end{Definition}

%Bem
\begin{Bemerkung}
\begin{enumerate}
	\item $f: U \rightarrow M \, \mathscr C^1$-Untermannigfaltigkeit injektiver Immersion $\Rightarrow f:U \rightarrow f(U) \, \mathscr C^1$-Homomorphismus

	\item $f: U \rightarrow ?$ injektive Immersion $\nRightarrow f: U \rightarrow f(U) \, \mathscr C^1$-Homomorphismus.
\end{enumerate}
\end{Bemerkung}

%Lemma IV.1
\begin{Lemma}
Für jedes $m$-dimensionale $\mathscr C^1$-Untermannigfaltigkeit des $\mathbb R^n$ gibt es lokale Parametrisierung $f_i: U \rightarrow M, i \in \mathbb N$, sodass
\begin{center}
	$M = \bigcup\limits_{i=1}^\infty f(U)$
\end{center}
\end{Lemma}


%Satz IV.4
\begin{Satz}
Sei $M \subset \mathbb R^n$ \umgf. \,$ B \subset M$ heißt \underline{messbar}, falls $\phi^{-1}(B) \, L^m$-messbar ist für alle lokalen Parametrisierungen \\
 $\phi: \underbrace{U}_{\subset \mathbb R^m \text{ offen}} \rightarrow M$  \\
 Das System $M$ alles messbaren Teilmengen von $M$ ist eine $\sigma$-Algebra und enthält alle Borelmengen. Auf $M$ existiert genau ein Maß $\mu_M$, sodass für jede lokale Parametrisierung $\phi: U \rightarrow M$ und jede messbare Menge $B \subset \phi(U)$ gilt:
 \begin{center}
 	$\mu_M(B) = \int_{\phi^{-1}(B)} \partial \phi(X) \, d L^m(x)$
 \end{center}
\end{Satz}

%Satz IV.5
\begin{Satz}
Sei $M \subset \mathbb R^n$ \umgf. $M \subset \bigcup\limits_{i=1}^\infty M_i$ disjunkt, $M_i \subset V_i, \, \phi_i: U_i \rightarrow V_i = \phi_i(U_i)$ lokale Parametrisierung. Für eine $M$-messbare Funktion $u: M \rightarrow \overline{\mathbb R}$ gilt:
\begin{center}
	$\int_M u \, d\mu_M = \sum\limits_{i=1}^\infty \int_{\phi_i^{-1}(M_i)} u(\phi_i(x))J\phi_i(x) \, dL^m(x)$
\end{center}
\end{Satz}

%Satz IV.6
\begin{Satz}
Sei $\lambda > 0, \, Q \in \mathscr O(n), a \in \mathbb R^n$ und $\tau: \mathbb R^n \rightarrow \mathbb R^n$, $\tau(p) = \lambda Q(p+a)$ \\
Ist $M \subset \mathbb R^n$ \umgf $\Rightarrow \tau(M) =: N$ ist \umgf und für $\mu_m, \mu_N$ gilt:
\begin{enumerate}
	\item $A \subset M$ messbar $\Rightarrow \tau(A) \supset N$ messbar und $\mu_N (\tau(A)) = \lambda^m \mu_M (A)$
	\item Ist $u: N \rightarrow \overline{\mathbb R}$ $\mu_M$-messbar $\Rightarrow u \circ \tau: M \rightarrow \overline{\mathbb R}$ $\mu_M$-messbar, und falls entweder $u \geq 0$ oder $u \mu_N$-intgrierbar ist, gilt:
	\begin{center}
		$\int_N u(q) \,d\mu_N(q) = \lambda^m \int_M u(\tau(p))\, \mu_M(p)$
	\end{center}
\end{enumerate}
\end{Satz}

%Satz IV.7 Zwiebelformel
\begin{Satz}
\emph{Zwiebelformel}: \\
Für $u \in L^n(\mathbb R^{n+1})$ ist $u|_{JB_r} \in L^1(\mu_{JB_r})$ für fast alle $r >0$ mit $JB_r:= JB_r(0)$ und es gilt:
\begin{center}
	$\int{\mathbb R^{n+1}} u(p) \, dL^{n+1}(p) = \int\limits_0^\infty \int_{JB_r} u(p) \,d\mu_{JB_r} \,dL^1(r) = \int\limits_0^\infty r^n \int_{S^n} u(rw)\,d\mu_{S^n}(u) \,dL^1(r)$
\end{center}
\end{Satz}

%Bsp
\begin{Beispiel}
$u = \chi_{B_1}(0), \omega_n = \mu_{S^n} (S^n) $\\
$L_{n+1} = \mathscr L^{n+1}(B_1(0)) = \int_{\mathbb R^{n+1}} \chi_{B_1}(0) \,dL^{n+1} = \int\limits_0^1 \mu_{JB_r}(JF_r) \,dL^1(r) = \int\limits_0^1 r^n \omega_n \, dL^1(r)   = \dfrac{\omega_n}{n+1}$ \\
$\omega_1 = 2\pi, \, \alpha_2 = \pi; \, \omega_2 = 4\pi \, \Rightarrow \alpha_3 = \dfrac{4}{3}\pi$
\end{Beispiel}




\end{document}
