\documentclass[11pt]{memoir}

\usepackage[utf8]{inputenc}
\usepackage[ngerman]{babel}
\usepackage{mathrsfs}
\usepackage{amssymb}
\usepackage[standard]{ntheorem}

\begin{document}

\title{Analysis III}
\author{Prof. Dr. Tobias Lamm}
\date{Wintersemester 2015/16}

\maketitle

\chapter{Maße und messbare Funktionen}
\section{$\sigma$-Algebren und Maße}
Notation: $X$ Menge, \\ $P(X) =$ \textbraceleft Teilmengen von $X$\textbraceright  {} Potenzmenge \\
$A$ $\subset$  $P(X)$ Mengensystem
\begin{Definition}
Ein Mengensystem $\mathscr{A} \subset P(X)$ heißt \underline{$\sigma$-Algebra}, falls
\begin{enumerate}
	\item $X \in \mathscr{A}$
	\item $A \in \mathscr{A} \Rightarrow X \backslash A \in \mathscr{A}$
	\item $A_i \in \mathscr{A}$, $i \in$ $\mathbb{N} \Rightarrow \bigcup_{i = 1}^{\infty} A_i$
\end{enumerate}
Das System $(X, \mathscr{A})$ heißt \underline{messbarer Raum}.
\end{Definition}

\begin{Bemerkung}
\begin{enumerate}
	\item $A_i \in \mathscr{A}$, $i \in \mathbb{N} \Rightarrow \bigcap_{i = 1}^{\infty} A_i$, denn:
	$\bigcap_{i=1}^{\infty} A_i = X\backslash (\bigcup_{i=1}^{\infty} (X \backslash A_i))$
	\item $\emptyset \in \mathscr{A}$
	\item $A, B \in \mathscr{A} \Rightarrow A \backslash B \in \mathscr{A}$, denn $A \backslash B = A \cap (X \backslash B)$
\end{enumerate}
\end{Bemerkung}

\begin{Beispiel}
\begin{enumerate}
	\item $P(X)$ trivial
	\item \textbraceleft$ \emptyset, X $\textbraceright{} trivial
\end{enumerate}
\end{Beispiel}

\end{document}
