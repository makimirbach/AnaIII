\documentclass[11pt]{article}

\usepackage[utf8]{inputenc}
\usepackage[ngerman]{babel}
\usepackage{mathrsfs} %hübsche Buchstaben
\usepackage{amssymb}
\usepackage{ntheorem}
\everymath{\displaystyle}
\usepackage{amsmath, amssymb} %Damit geht dann cases
\usepackage{ulsy} %Für Widerspruchspfeil
\usepackage{mathtools} %Für Rightarrow mit Wort darüberunderbrace
\usepackage{extpfeil} %für = mit Wort drüber
\usepackage{fdsymbol} %varrightwavearrow - blockiert Underbrace: muss nach mathtools kommen, sonst hässliche Balken.
\usepackage{braket} % Für schönere Mengen. Oder so. Hier ist z. B. \Set drin und die vertikalen Balken sind schön.
\usepackage{scalerel, stackengine}
\stackMath

%really wide hat
\newcommand\rwhat[1]{
\savestack{\tmpbox}{\stretchto{%
  \scaleto{%
    \scalerel*[\widthof{\ensuremath{#1}}]{\kern-.6pt\bigwedge\kern-.6pt}%
    {\rule[-\textheight/2]{1ex}{\textheight}}%WIDTH-LIMITED BIG WEDGE
  }{\textheight}% 
}{0.5ex}}%
\stackon[1pt]{#1}{\tmpbox}%
}
\parskip 1ex
\newcommand\overequal[1]{\mathrel{\overset{\makebox[0pt]{\mbox{\normalfont\tiny\sffamily $ #1 $}}}{=}}}
\newcommand{\dom}{\partial\Omega}
\newcommand{\pax}{\partial x}
\newcommand{\pay}{\partial y}
\newcommand{\pau}{\partial u}
\newcommand{\pav}{\partial v}
\newcommand{\pap}{\partial\phi}
\newcommand{\bpi}{\dfrac{1}{(2\pi)^{n/2}}}
\begin{document}


\title{\textbf{Analysis III}\\ Übung am 12.02.16}
\author{Tamar Mirbach}
\date{Wintersemester 2015/16\\ Karlsruher Institut für Technologie}

\maketitle

\emph{Blatt 14}

\section{Aufgabe 1}
$\Omega \subset \mathbb R^n$ beschränkt mit $C^1$-Rand. Für $\lambda \in \mathbb R, u \in C^2(\overline{\Omega}), u \ne 0$:
\begin{align}
-\Delta u = \lambda u & \quad \text{ in } \Omega \\
u = 0 & \quad \text{ auf } \dom
\end{align}
\emph{Beweis}: \\
Greensche Formel: \\
$$\int_\Omega \Delta v u + \bigtriangledown u \cdotp \bigtriangledown = \int_{\partial \Omega} \dfrac{\pau}{\pav} \cdotp u$$
$$ \Rightarrow -\lambda \int u^2 + \int \|Du\|^2 = 0$$
$$\Rightarrow \int_\Omega \|Du \|^2 = \lambda \int_\Omega u^2$$
$$\Rightarrow \lambda > 0, $$ da $\int_\Omega \|Du\|^2 \ne 0$ \\

\underline{2. Fall}:\\
$\dfrac{\pau}{\pav} = 0$ auf $\dom$ aus Greenscher Formel:
$$\int_\Omega \|Du\|^2 = \lambda \int_\Omega U^2$$%U??
$($falls LS $ = 0 \Rightarrow u \equiv $konstant$ \Rightarrow u \equiv $ konstant $ \ne 0$ wäre eine Lösung mit $\lambda = 0)$ \\
falls $u \ne $konstant 
$$\Rightarrow \int_\Omega \|Du\|^2 > 0 \Rightarrow \lambda > 0$$


\section{Aufgabe 2}
$A \in \mathbb R^{n \times n}$ symmetrisch, $n \geq 2$
$$\int_{S^{n-1}} \langle Ax, x \rangle \; d\mu_{S^{n-1}}$$

\par\bigskip

$$u(x) = \,1/2\, \langle Ax, x \rangle = \sum\limits_{i, j =1}^n \,1/2\, a_{ij} x^i x^j ,\qquad A = (a_{ij})_{ij}$$
$$Du(x) = Ax, \qquad \Delta u(x) = \sum\limits_{i=1}^n \partial_{ii}u(x) $$
$$\dfrac{\pau}{\pav} (x) = \left\langle Du(x), \dfrac{x}{|x|}\right\rangle = \left\langle Ax, \dfrac{x}{|x|} \right \rangle = \langle Ax, x \rangle \qquad \forall x \in S^{n-1}$$


$$\int_{S^{n-1}} \langle Ax, x \rangle \; d\mu_{S^{n-1}} = \int_{\partial B_1} \dfrac{\pau}{\pav} \; d\mu_{S^{n-1}}  = \int_{B_1} \Delta u(x) \; dx = \text{tr}(A) \cdotp L^n(B_1) = \alpha_n \cdotp \text{tr}(A)$$


\section{Aufgabe 3}
\emph{Beispiel einer Faltung}: \\
\begin{align}
f(x) = \chi_{[0,1]}(x) \\
\chi_{[0,1]} (x-y) = \chi_{[x-1, x]}(y) \\
\chi_A (y) \chi_B (y) = \chi_{A \cap B}(y) & \qquad \forall A, B \subset \mathbb R \text{ messbar}
\end{align}

$$f \ast  f(x) = \int \chi_{[0,1]}(x-y) \cdotp \chi_{[0,1]} (y) \; dy = \int \chi_{[x-1, x]}(y) \chi_{[0,1]}(y) \; dy$$
\begin{equation}
= L^1\left([x-1, x] \cap [0, 1]\right) = 
\begin{cases}
	0 & \text{, für } x > 0 \\
	x & \text{ , für } 0 < x < 1\\
	2-x & \text{, für } 1 < x < 2 \\
	0 & \text{, für } x > 2
\end{cases}
\end{equation}
$f \ast  f(x)$ ist stetig, nicht differenzierbar in $\Set{0, 1, 2}$ 
$$h(y) = f \ast f(y) $$
$$ k(x):= f \ast f \ast  f (x) = \int \chi_{[x-1, x]}(y) h(y) \; dy = \int\limits_{x-1}^x h(y) \; dy$$
falls $x \notin [0, 3] \Rightarrow k(x) = 0$
\begin{itemize}
	\item \underline{1. Fall} : \\
	$x \in [0, 1]$
	$$k(x) = \int\limits_0^x y \; dy = \,1/2\, x^2$$
	\item \underline{2. Fall} : \\
	$$k(x) = \int\limits_{x-1}^1 h(y) \; dy + \int\limits_1^x h(y) \; dy = \,1/2\, \left(1-(x-1)^2\right)-\,1/2\, (2-y)^2 \vert_1^x$$ 
	$$ = \,1/2\, \left(1-(x-1)^2 -(2-x)^2 + 1\right) = \,1/2\, \left(2-x^2+ 2x - 1-4 + 4x - x^2\right) $$
	\par
	$$ = \,1/2\,(-3 + 6x - 2x^2)$$
	\item \underline{3. Fall}: \\
	$x \in [2,3]$
	$$k(x) = \int\limits_{x-1}^2 h(y) \; dy = -\,1/2\, (2-x)^2 \vert_{x-1}^2 = \,1/2\, (3-x)^2$$
	$k(x)$ ist differenzierbar mit $k'(x_0) = 0, \;k'(0),\; k'(1) = 1,\; k'(2) = -1$
\end{itemize}

\section{Aufgabe 4}
$\Omega \subset \mathbb R^2$ offen, beschränkt mit $C^1$-Rand. $\nu: \dom \rightarrow \mathbb R^2$ äußere Normale, d.h.
\begin{equation}
\tau = \zeta \nu, \qquad \zeta = 
\begin{pmatrix}
	0 & -1 \\
	1 & 0
\end{pmatrix} 
\end{equation}
$$\Rightarrow F \in C^1(\overline\Omega, \mathbb R^2): \quad \int_\Omega curl(F) \; dL^2 = \int_{\dom} \langle F, \tau \rangle \; d\mu_{\dom}$$
Wir stellen fest: 
\begin{equation}
J^T = 
\begin{pmatrix}
0 & 1 \\
-1 & 0
\end{pmatrix}
\end{equation}
$$div(J^T F(x)) = div((F^2, -F^1)) \text{ für } F = (F^1, F^2)$$
$$ = \partial_1 F_2 - \partial_2 F^1 = curl (F)$$
$$\int_\Omega curl (F) = \int_\Omega div (J^T F) = \int_{\dom} \langle J^T F, \nu \rangle \; d\mu_{\dom} = \int_{\dom} \langle F, J\nu \rangle \; d\mu_{\dom} = \int_{\dom} \langle F, \tau \rangle \; d\mu_{\dom}$$

\section{Spannendes..}
$S = S(\mathbb R^n, \mathbb C) := \Set{f \in C^\infty (\mathbb R^n, \mathbb C) | x^\alpha D^\beta f \text{ beschränkt } \forall \alpha, \beta \in \mathbb N_0^n}$

\subsection{1. Identität}
$$(i \lambda)^\alpha D^\beta \hat f(\lambda) =  \; \rwhat{ D^\alpha \left((-ix)^\beta f\right)}$$
\underline{Beweis}:
da 
$$F(x, \lambda):= \dfrac{f(x) e^{i x \lambda}}{(2\pi)^{n/2}}$$
ist stetig differenzierbar in $x$ und $\lambda$ und
$$\left|x^\alpha D^\beta F(x, \lambda) \right| \in C^1$$
$$\hat f (\lambda) = \dfrac{1}{(2\pi) ^{n/2}} \int_{\mathbb R^n} f(x) e^{-i x \lambda} $$
$$ \Rightarrow\quad  (i \lambda)^\alpha D^\beta \hat f(\lambda) = \dfrac{1}{(2\pi)^{n/2}} \int (i \lambda)^\alpha (-i x)^\beta f(x) e^{- i \lambda x} \; dx $$
$$ = \bpi \int_{\mathbb R^n} (-1)^\alpha D^\alpha \left(e^{-i \lambda x}\right) (-i x)^\beta f(x) \; dx$$
$$= \bpi \int e^{-i \lambda x} \left(D^\alpha(-i x)^\beta f(x)\right) \; dx = \rwhat{ D^\alpha\left((-i x)^\beta f\right)}(\lambda)$$
%Abschnitt 2
\subsection{2. Identität}
$\hat f \in S$. \\
Zu zeigen: $(+ i \lambda)^\alpha D^\beta \hat f(\lambda) $ ist beschränkt.\\
$$(i \lambda)^\alpha D^\beta \hat f(\lambda) = \rwhat{ D^\alpha \left((-i x)^\beta f\right)} (\lambda)$$
Sei $g(x) := D^\alpha\left((-ix)^\beta f(x)\right)$\\
Da $f \in S \Rightarrow g \in S$
$$\left| (i \lambda)^\alpha D^\beta \hat f(\lambda)\right| = \bpi \left| g(x) e^{-i x \lambda}\; dx \right|$$
$$\leq \bpi \int_{\mathbb R^n} \dfrac{1}{\left(1+|x|^2\right)^n} \sup_x \left|\left(1+|x|^2\right)^n g(x) \right| \;dx$$
$$\leq C \sum\limits_{\substack{\mu, \nu \,\in\,\mathbb N_0^n, \\  |\mu| \,\leq\, 2n + \beta, \\ |\nu| \leq |\alpha|}} \|x^\mu D^\nu f\|_\infty$$ 
Aus der Vorlesung wissen wir:
$ g: [-\pi, \pi]$ in $C^\infty$, dann folgt, dass 
$$g(x) = \sum\limits_{k \in \mathbb Z} g(k) e^{ i k x} \quad \text{ mit } \quad g(k) = \dfrac{1}{2\pi} \int\limits_{-\pi}^\pi g(x) e^{-ikx}$$
Das heißt für $g: [-\pi, \pi]^n \rightarrow \mathbb C \quad C^\infty$, dass gilt:
$$g(x) = g(x_1, ..., x_n) = \sum\limits_{k \in \mathbb Z^n} g(k) e^{ikx}$$
$$k \cdot x = \sum\limits_{i=1}^n k_i x_i$$
$$g(k)= \dfrac{1}{(2\pi)^n} \int_{[-\pi, \pi]^n} g(x) e^{-ikx} \; dx$$
Sei nun $f \in C_c^\infty (\mathbb R^n, \mathbb C)$ mit supp$(f) \subset B_{R_0}$ und $N \in \mathbb N$ mit 
$$\sqrt{n} \;R_0 < N \quad \Rightarrow \quad  B_{R_0} \subset [-N, N]^n$$
$$f_N(x) = f(Nx)$$
$$f(Nx) = f_N(x) = \sum_{k \in \mathbb Z^n} \hat f_n (k) e^{i k/N \; x^N}$$
$$\hat f_N (k) = \bpi \int_{[-\pi, \pi]^n} f_N(x) e^{-i k x} \;dx $$  
%
$$ \overequal{Nx = z }\quad \dfrac{1}{(2\pi)^n N^n} \int_{[-\pi N, \pi N]^n} f(z) e^{-ikz / N} \; dz = \bpi 1/N^n \hat f \left( \dfrac{k}{N} \right)$$
$$f(z) = f_N\left( \dfrac{z}{N} \right) = \sum_{k \in \mathbb Z^n} \hat f_N(k) e^{ikz/N} = \bpi \sum_{k \in \mathbb Z^n} \dfrac{\hat f \left( \dfrac{k}{N} \right) e^{ikz/N}}{N^n}$$
$S_N(z) = \bpi \sum_{k \in \mathbb Z^n} \dfrac{\hat f\left( \dfrac{k}{N} \right) e^{ikz/N}}{N^n}$ für $g \in S$ und $a, b \in \mathbb R^n$
$$\left|g(a) e^{iaz}- g(b) e^{ibz}\right| \leq \|Dg\|_\infty + |z| \|g\|_\infty \text{ auf } |a-b|$$ bzw. was macht das da??\\
%
\emph{Behauptung}: \\
$$\lim_{N \rightarrow \infty} S_N(z) = \bpi \int_{\mathbb R^n} \hat f(\lambda) e^{i \lambda z} \, d\lambda $$
$$\Rightarrow \check{ \hat  f}(z) = f(z)$$
$$\left|S_N(z) - \bpi \int_{\mathbb R^n} \hat f(\lambda) ^{i \lambda z}\; d\lambda \right| = \left| \bpi \sum_{k \in \mathbb Z^n} \int_{k/N + [0, 1/N]^n} \hat f\left( \dfrac{k}{N} \right) ^{ikz/N} - \hat f(\lambda) e^{i \lambda z} \; d\lambda \right|$$
$$ \leq \bpi \sum_{k \in \mathbb Z^n} \int_{k/N + [0, 1/N]^n}\underbrace{ \sup_{\lambda \in k/N + [0, 1/N]^n} \left(|(1+\lambda^2)^n\left(|D\hat f| + z|\hat f|\right)\right)}_{\leq C, \text{ da }\hat f \in S} \; d\lambda \dfrac{\sqrt{n}}{N}$$
$$\leq C/N \int_{\mathbb R^n} (1+\lambda^2)^{-n} \; d\lambda \rightarrow 0 \quad (N \rightarrow \infty)$$

\section{Wichtig sonst:}
$$\| \hat f \|_{L^2} = \|f \|_{L^2} \Rightarrow \langle \hat f , g \rangle = \langle f, \check g\rangle$$
\end{document}